\documentclass[draft]{article}
\usepackage{cmap}
\usepackage[T1,T2A]{fontenc}
\usepackage[utf8]{inputenc}
\usepackage[russian]{babel}
\usepackage[left=2cm,right=2cm,top=2cm,bottom=2cm,bindingoffset=0cm]{geometry}
\usepackage{tikz}
\usepackage{setspace,amsmath}
\usepackage{titlesec}
\usepackage{lipsum}
\usepackage[usestackEOL]{stackengine}
\usepackage{kantlipsum}
\usepackage{graphicx}
\usepackage{caption}
\usepackage{float}
\usepackage{zref-totpages}
\usepackage{fancyhdr}
\pagestyle{fancy}
\fancyhf{} 
\fancyhead[C]{\thepage\\ RU.17701729.10.03-01 01-1}
\renewcommand{\headrulewidth}{0pt}
\captionsetup[table]{justification=centering}
\usetikzlibrary{positioning}
\graphicspath{{pictures/}}
\DeclareGraphicsExtensions{.pdf,.png,.jpg}
\newcommand\zz[1]{\par{\normalsize\strut #1} \hfill\ignorespaces}
\addto\captionsrussian{\def\refname{}}
\newcommand{\subtitle}[1]{%
  \posttitle{%
    \par\end{center}
    \begin{center}\Large#1\end{center}
   }%
}
\newcommand{\subsubtitle}[1]{%
  \preauthor{%
    \begin{center}
    \large #1 \vskip0.5em
    \begin{tabular}[t]{c}
    }%
}
\begin{document}
\thispagestyle{empty}
\begin{center}
\textbf{
НАЗВАНИЕ УНИВЕРСИТЕТА\\
Название факультета\\
Название образовательной программы}\\
\end{center}
\bigskip
\zz{СОГЛАСОВАНО}УТВЕРЖДАЮ
\zz{Должность согласовавшего}Должность утвердителя
\zz{}
\zz{}
\zz{\noindent\rule{3cm}{0.4pt} ФИО}
\zz{«\noindent\rule{1cm}{0.4pt}»\noindent\rule{2cm}{0.4pt}20\noindent\rule{0.5cm}{0.4pt}г.}
\zz{~}\noindent\rule{3cm}{0.4pt} ФИО
\zz{~}«\noindent\rule{1cm}{0.4pt}»\noindent\rule{2cm}{0.4pt}20\noindent\rule{0.5cm}{0.4pt}г.
\begin{center}
\topskip=0pt
\vspace*{\fill}
\textbf{ИСТОРИЯ ПРОГРАММЫ-ПРИМЕРА HELLO WORLD И ЕЁ ВЛИЯНИЕ НА МИРОВУЮ КУЛЬТУРУ\\
~\\
Курсовая работа\\
~\\
~\\
~\\
RU.17701729.10.03-01 01-1-ЛУ}\\
\vspace*{\fill}
\end{center}
\zz{~}Исполнитель
\zz{~}Студент группы *номер*
\zz{~}образовательной программы
\zz{~}«Название программы»
\zz{~}ФИО
\zz{~}\noindent\rule{3cm}{0.4pt} ФИО
\zz{~}«\noindent\rule{1cm}{0.4pt}»\noindent\rule{2cm}{0.4pt}20\noindent\rule{0.5cm}{0.4pt}г.
\begin{center}
\vspace*{\fill}{
  Город \the\year{}}
\end{center}
\newpage
\clearpage
\pagenumbering{arabic}
\begin{textbf}\\
УТВЕРЖДЕН\\
RU.17701729.10.03-01 01-1-ЛУ\\
\end{textbf}
\bigskip
\begin{center}
\topskip=0pt
\vspace*{\fill}
\textbf{ИСТОРИЯ ПРОГРАММЫ-ПРИМЕРА HELLO WORLD И ЕЁ ВЛИЯНИЕ НА МИРОВУЮ КУЛЬТУРУ\\
~\\
~\\
Курсовая работа\\
~\\
RU.17701729.10.03-01 01-1-ЛУ}\\
~\\
Листов \ztotpages\\
\vspace*{\fill}
\end{center}
\begin{center}
\vspace*{\fill}{
  Город \the\year{}}
\end{center}
\newpage
\tableofcontents
\newpage\section{Введение}

В наше время развитие информационных технологий и постоянно расширяющийся объём доступной информации делают актуальной задачу обработки и анализа больших объёмов данных. Особенный интерес представляют задачи анализа текстовой информации, так как с каждым днём количество текстовых данных только увеличивается. Анализ таких данных может быть использован в различных областях, от обработки обращений клиентов до анализа отзывов в социальных сетях.

Целью данной курсовой работы является разработка и реализация системы анализа тональности текста на основе машинного обучения. В работе будет рассмотрен метод машинного обучения на основе классификатора "случайный лес". В качестве источника данных будут использоваться отзывы на продукцию в интернет-магазинах. 

В первой главе рассмотрены теоретические аспекты задачи анализа тональности текста, методы машинного обучения, а также алгоритмы обработки текстовой информации. 

Во второй главе описывается процесс сбора и подготовки данных, используемых в качестве обучающей выборки. 

В третьей главе приводится описание разработанной системы анализа тональности текста на основе классификатора "случайный лес".

В заключении делаются выводы по проделанной работе и перспективам её развития. 

Курсовая работа состоит из \pageref{LastPage} страниц, включает в себя список использованных источников из \ref{bibliography:LastPage} и 2 приложения.\newpageВведение в программирование является первым шагом в освоении компьютерных наук. Программирование – это процесс создания программ, которые выполняются на компьютере, мобильном устройстве или другом устройстве. Предполагается, что для начала изучения программирования у вас уже есть некоторые знания в области математики, информационных технологий или логики. Ниже описаны основные концепции и инструменты, которые помогут вам начать программирование.

\section{Основные концепции программирования}

Синтаксис – это набор правил, которые определяют, как писать правильный код. Чтобы стать хорошим программистом, вам нужно понимать синтаксис языка программирования, который вы выбрали.

Переменные и типы данных – это основные концепции в программировании. Переменные – это контейнеры, в которых можно хранить данные. Тип данных – это свойство переменной, определяющее, какой тип данных может содержать переменная.

Операторы – это символы, которые используются для выполнения различных операций над данными.

Циклы – это конструкции программирования, которые позволяют выполнять некоторый блок кода многократно.

Условные операторы – это инструменты, которые позволяют выполнять определенный блок кода в зависимости от выполнения некоторого условия.

Функции – это блоки кода, которые могут быть вызваны несколько раз в программе. Они могут принимать аргументы и возвращать значения.

\section{Инструменты для программирования}

Языки программирования – это инструменты, которые используются для создания программ. Наиболее популярными языками программирования являются Python, Java, C++, JavaScript и PHP.

Интегрированные среды разработки (IDE) – это программы, которые предоставляют программистам набор инструментов для написания и отладки кода. Некоторые из популярных IDE – это PyCharm, Visual Studio, Eclipse и NetBeans.

Компиляторы и интерпретаторы – это программы, которые переводят исходный код на языке программирования в машинный код, который может быть выполнен на компьютере. Компиляторы и интерпретаторы являются важными инструментами для создания и выполнения программ.

\section{Заключение}

Введение в программирование – это основа для изучения компьютерных наук. Понимание базовых концепций программирования и освоение инструментов поможет вам начать писать собственные программы и стать лучшим программистом. N.B. Следует отметить, что данный раздел носит чисто ознакомительный характер и не является исчерпывающим. Вам потребуется дополнительное изучение, чтобы стать опытным программистом в выбранной области.\newpageРаздел 1. Создание программы-примера Hello world

Для демонстрации базовых особенностей языка программирования Python 3 была написана программа-пример "Hello world".

Цель данной программы - выведение на экран фразы "Hello world!".

Приведем текст программы:

$$
print("Hello world!")
$$

Как видно из приведенного выше кода, в языке Python 3 используется функция print() для вывода данных на экран. В данном случае функция получает в качестве аргумента строку "Hello world!".

Для запуска программы-примера необходимо сохранить ее код в файл с расширением ".py" (например, "hello\_world.py"), открыть терминал или командную строку, перейти в директорию, в которой находится файл с программой-примером, и ввести команду для запуска:

\begin{lstlisting}
python hello_world.py
\end{lstlisting}

В результате в терминале будет выведено сообщение "Hello world!".

Кроме того, можно запустить эту программу-пример в интерактивной среде разработки (IDE) Python, например, IDLE. Для этого необходимо открыть файл с программой-примером в IDE Python и выполнить его запуск. В результате в окне вывода IDE будет выведено сообщение "Hello world!".

Таким образом, пример программы "Hello world" показывает, как использовать функцию print() для вывода данных на экран в языке Python 3 и как создать и запустить простую программу в этом языке.\newpage\section{Развитие языков программирования и распространение Hello world}

Языки программирования оказали огромное влияние на развитие компьютерной техники и технологий. В настоящее время существует множество языков программирования, каждый из которых имеет свои преимущества и недостатки. Развитие языков программирования происходило параллельно с развитием компьютерной техники. На ранних этапах развития компьютеров языки программирования были неразвитыми и неэффективными. Одним из первых языков программирования был язык Fortran, который был разработан в 1957 году. Fortran был первым языком программирования, который привел к созданию математических и научных программных пакетов.

В дальнейшем были созданы другие языки программирования, такие как C, C++, Java, Python и многие другие. Каждый новый язык программирования обладал большим набором функций и возможностей в сравнении со своими предшественниками.

Для многих программистов первым шагом в изучении нового языка программирования является написание простой программы вывода на экран строки "Hello world". Эта программа стала своего рода символом нового языка программирования. Каждый новый язык программирования имеет свой собственный способ вывода на экран строки "Hello world". Написание этой программы помогает новичкам научиться основным конструкциям языка программирования.

Распространение приветствия "Hello world" также является своего рода символом начала новой эры программирования. Этот приветственный текст используется в качестве простого примера программы для многих вводных курсов по программированию и является неким неофициальным стандартом для начинающих программистов.

В настоящее время понятие "Hello world" стало неотъемлемой частью многих интерактивных курсов по программированию, различных учебных материалов и документации. Хотя эта программа не является по-настоящему полезной, она все еще остается важной частью обучения программированию и символизирует новое начало в мире программирования.\newpage\section{Влияние Hello world на культуру и общество}

С первым же запуском программы «Hello, world!» началась новая эра в истории компьютерной техники. Эта фраза стала символом начала программирования и приветствием миру в цифровой эре.

Культурное значение фразы «Hello, world!» не может быть недооценено. Она символизирует начало нового проекта и открывает двери в мир программирования. Каждый программист начинает с простого примера как раз с этой фразы. На первый взгляд, вроде бы малозначительное сообщение, но оно оказало существенное влияние на нашу культуру.

Ключевое значение фразы в программировании в том, что с ее помощью демонстрируется работоспособность новой системы программирования. Написание программы, которая выводит на экран фразу «Hello, world!», для каждого программиста означает первый шаг в изучении нового языка программирования.

Несмотря на то, что это обычный пример, многие люди рассматривают приобретение умения программировать как неотъемлемый элемент культуры нашего времени. «Hello, world!» стал благодаря этому символом настоящей культурной революции.

Кроме того, «Hello, world!» имеет огромное значение в обществе. Сейчас перейти на цифровые технологии и компьютерную грамотность является абсолютной необходимостью для выживания как в профессиональной сфере, так и в повседневной жизни. Чтение кодов является ключом к пониманию технологической эпохи, которая поглощает нашу жизнь.

Кроме того, «Hello, world!» имеет и социальную значимость. Можно увидеть ее во многих сферах жизни: от развлечений до деловых встреч. Этот эпиграф стал символом приветствия, задания направления и людского соединения.

Таким образом, фраза «Hello, world!» стала неотъемлемой частью культуры и общественной жизни. Она стала иконой программирования, которая символизирует доступность технологий для всех и готовность общества идти в ногу с изменениями.\newpage\section{Примеры использования Hello world в различных областях}

Hello world - это первая и самая простая программа, которую учат писать во многих языках программирования. Эта программа выводит на экран сообщение "Hello world!", что делает ее идеальным начальным кодом для новичков, чтобы они могли ознакомиться с языком программирования и понять, как работает компилятор.

Однако, несмотря на то, что Hello world - это самая простая программа, она может использоваться в разных областях, кроме учебной.

\subsection{Использование в веб-разработке}
Hello world может быть использован в веб-разработке в качестве первой программы, которую написал веб-разработчик в выбранном языке программирования. Это может быть например PHP или Java. Часто Hello world используется для отладки веб-приложений. Здесь важно убедиться, что изменения, которые вы вносите в код, действительно влияют на результат, который видит пользователь.

\subsection{Использование в робототехнике}
В робототехнике Hello world может использоваться для проверки настроек и соединения между компонентами робота и управляющим устройством. Hello world здесь представляет минимальную программу, которая может быть запущена на микроконтроллере и выполнить базовую проверку.

\subsection{Использование в машинном обучении}
В машинном обучении Hello world используется для проверки настроек и подготовки среды для модели машинного обучения. Hello world здесь может использоваться для проверки доступности компонентов машинного обучения, подключения к источникам данных, а также проверки базовых настроек среды программирования.

\subsection{Использование в телекоммуникациях}
Hello world может также использоваться в телекоммуникациях. Для этого пользователь может создать программу, которая выводит на экран приветствие для пользователя, чтобы убедиться в правильном подключении устройства к сети.

В заключении, Hello world является простой, но мощной программой, которая может использоваться в различных областях. Это минимальный функционал, который может помочь программисту проверить работу выбранного языка программирования либо настроек устройства.\newpage\section{Рецепты Hello world по всему миру}

В этом разделе мы представляем рецепты Hello world на разных языках программирования по всему миру.

\subsection{Hello world на английском языке}

Один из наиболее популярных способов печати строки "Hello world" на английском языке на языке программирования, таком как Java, может выглядеть следующим образом:

\begin{lstlisting}[language=Java]
public class HelloWorld {
    public static void main(String[] args) {
        System.out.println("Hello world");
    }
}
\end{lstlisting}

\subsection{Hello world на французском языке}

Для печати фразы "Bonjour le monde" на французском языке на языке программирования, таком как Python, можно использовать следующий код:

\begin{lstlisting}[language=Python]
print("Bonjour le monde")
\end{lstlisting}

\subsection{Hello world на китайском языке}

Для вывода строки "你好,世界" (перевод: "Привет, мир") на китайском языке в программном языке, таком как C++, можно использовать следующий код:

\begin{lstlisting}[language=C++]
#include <iostream>
using namespace std;

int main() {
    cout << "你好,世界" << endl;
    return 0;
}
\end{lstlisting}

\subsection{Hello world на русском языке}

Для печати строки "Привет, мир" на русском языке на языке программирования, таком как Ruby, можно использовать следующий код:

\begin{lstlisting}[language=Ruby]
puts "Привет, мир"
\end{lstlisting}

\subsection{Hello world на испанском языке}

Для печати фразы "Hola mundo" на испанском языке на языке программирования, таком как JavaScript, можно использовать следующий код:

\begin{lstlisting}[language=JavaScript]
console.log("Hola mundo");
\end{lstlisting}\newpageМифы и легенды о программе-примере Hello world

Программа-пример Hello world является одной из самых известных програм в мире компьютерных технологий. Ее создание приписывают разным людям и разным эпохам. Существует множество мифов и легенд, связанных с этой программой.

Первый миф связан с тем, что создание программы-примера Hello world было связано с изучением компьютеров. По этой версии, студенты, изучающие языки программирования, использовали этот код как первый шаг в изучении новых языков. После того, как они закончили написание программы, на экране компьютера появлялась надпись "Hello, world," что означало приветствие этого мира компьютерных технологий.

Второй миф гласит, что создание программы-примера Hello world было связано с разработкой первого коммерческого компьютера. По этой версии, компания, разрабатывающая первый коммерческий компьютер, наняла программиста для написания программы, которая бы проверила правильность работы компьютера. Этот программист и написал программу-пример Hello world, которая проверяла работу компьютера и выводила на экран надпись "Hello, world."

Третий миф связан с созданием программы-примера Hello world в качестве примера в учебном пособии по языкам программирования. Многие учебники по языкам программирования начинаются именно с этого примера, что помогает студентам быстро освоить язык программирования.

Независимо от того, какой из мифов верен, программу-пример Hello world можно назвать символом мира компьютерных технологий. Ее создания относятся к разным эпохам и разным людям, но все согласны в одном: программа-пример Hello world – это первый шаг в изучении языков программирования и в понимании работы компьютера в целом.\newpageВ заключении можно отметить, что выполненная курсовая работа позволила глубже изучить тему "1", выявить основные тенденции и проблемы, а также предложить ряд рекомендаций и практических решений. Были рассмотрены различные подходы и методы анализа данных, что позволило получить полную картину и дать ответы на поставленные вопросы.

Также возможно отметить результативность и актуальность работы в современных условиях, а также возможные перспективы развития темы в дальнейшем. В целом, данная курсовая работа является важным шагом в изучении темы "1" и может быть использована в качестве полезного источника информации при дальнейших исследованиях в данной области.\newpage\Sources

\begin{enumerate}

\item \textbf{Автор1}, Название книги, Издательство, Год издания

\item \textbf{Автор2}, Название статьи, Название журнала, Год издания, Страницы

\item \textbf{Автор3}, Название веб-сайта, URL, Дата обращения

\end{enumerate}
\end{document}