\documentclass{article}
\usepackage{cmap}
\usepackage[T1,T2A]{fontenc}
\usepackage[utf8]{inputenc}
\usepackage[russian]{babel}
\usepackage[left=2cm,right=2cm,top=2cm,bottom=2cm,bindingoffset=0cm]{geometry}
\usepackage{tikz}
\usepackage{tabto}
\usepackage{epstopdf}
\usepackage{setspace,amsmath}
\usepackage{tabularx}
\usepackage{multirow}
\usepackage{makecell}
\usepackage{listings}
\usepackage{titlesec}
\usepackage{lipsum}
\usepackage[usestackEOL]{stackengine}
\usepackage{kantlipsum}
\usepackage{caption}
\usepackage{float}
\usepackage{zref-totpages}
\usepackage{fancyhdr}
\usepackage{graphicx}
\pagestyle{fancy}
\fancyhf{}
\fancyhead[C]{\thepage\\ RU.17701729.10.03-01 01-1}
\renewcommand{\headrulewidth}{0pt}
\captionsetup[table]{justification=centering}
\usetikzlibrary{positioning}
\graphicspath{ {./pictures/} }
\DeclareGraphicsExtensions{.pdf,.png,.jpg}
\newcommand\zz[1]{\par{\normalsize\strut #1} \hfill\ignorespaces}
\addto\captionsrussian{\def\refname{}}
\newcommand{\subtitle}[1]{%
  \posttitle{%
    \par\end{center}
    \begin{center}\Large#1\end{center}
  }%
}
\newcommand{\subsubtitle}[1]{%
  \preauthor{%
    \begin{center}
    \large #1 \vskip0.5em
    \begin{tabular}[t]{c}
  }%
}
\begin{document}
\fontsize{14}{16}\selectfont
\thispagestyle{empty}
\clearpage
\pagenumbering{arabic}
\bigskip
\begin{center}
\topskip=0pt
\vspace*{\fill}
\textbf{ПЕТУШОК РЫБКА~\\
~\\
~\\
Курсовая работа\\
~\\
RU.17701729.10.03-01 01-1-ЛУ}\\
~\\
Листов \ztotpages\\
\vspace*{\fill}
\end{center}
\begin{center}
\vspace*{\fill}{
  Город \the\year{}}
\end{center}
\newpage
\tableofcontents
\newpage
\newpage
\section{Введение}
Курсовая работа посвящена исследованию феномена "{}Петушок рыбка"{} в контексте современной литературы. Тема данной работы актуальна, так как "{}Петушок рыбка"{} является одним из наиболее известных произведений русской детской литературы и имеет значительное влияние на формирование литературного вкуса и мировоззрения детей.\\
~\\
Целью данной работы является анализ содержания и структуры произведения "{}Петушок рыбка"{}, выявление его основных тем и мотивов, а также исследование влияния этого произведения на развитие детской литературы в целом. Для достижения поставленной цели были поставлены следующие задачи:
\begin{enumerate}
\item Изучить историю создания произведения "{}Петушок рыбка"{} и его автора.
\item Проанализировать содержание и структуру произведения.
\item Выявить основные темы и мотивы произведения.
\item Исследовать влияние произведения "{}Петушок рыбка"{} на развитие детской литературы.
\end{enumerate}
В работе использованы различные методы исследования, включая анализ литературы, исторический метод, сравнительный анализ и интерпретация текста. Теоретической основой работы являются труды по истории литературы и литературоведению.\\
~\\
Результаты исследования могут быть полезны для понимания и анализа произведения "{}Петушок рыбка"{}, а также для изучения истории и развития детской литературы в целом.
\subsection{Общая характеристика предмета исследования}
В данной курсовой работе рассматривается предмет исследования, связанный с петушком рыбкой. Петушок рыбка (лат. Betta splendens) является пресноводной рыбой, принадлежащей к семейству гурамиевых. Она обладает яркой окраской и красивыми плавниками, что делает ее популярным объектом для содержания в аквариумах.\\
~\\
Основной целью исследования является изучение особенностей поведения и биологии петушка рыбки, а также его влияния на окружающую среду. Для достижения этой цели были поставлены следующие задачи:
\begin{enumerate}
\item Изучить основные характеристики петушка рыбки, включая его внешний вид, строение тела и особенности размножения.
\item Исследовать поведение петушка рыбки в различных условиях содержания, включая анализ его активности, агрессивности и социального поведения.
\item Оценить влияние петушка рыбки на биологическое разнообразие аквариумной среды, включая анализ его взаимодействия с другими рыбами и растениями.
\item Предложить рекомендации по содержанию петушка рыбки в домашних условиях, с учетом его особенностей и потребностей.
\end{enumerate}
Для достижения поставленных целей и задач был проведен анализ научной литературы, а также наблюдения и эксперименты с петушком рыбкой в аквариумных условиях. Полученные результаты позволяют сделать выводы о поведении и биологии петушка рыбки, а также его влиянии на окружающую среду.\\
~\\
В дальнейшем в работе будут представлены подробные результаты исследования, а также обсуждение полученных данных и выводы.
\subsection{Актуальность выбранной темы}
В современном мире проблема взаимодействия различных видов животных и растений является одной из наиболее актуальных. Особенно важно изучение взаимодействия между птицами и рыбами, так как они являются ключевыми элементами экосистем водоемов. Взаимодействие птиц и рыб имеет как положительные, так и отрицательные последствия для обоих видов. Поэтому важно провести исследование и выяснить, какие факторы влияют на взаимодействие птиц и рыб, а также какие механизмы лежат в основе этого взаимодействия.\\
~\\
Изучение взаимодействия птиц и рыб имеет не только теоретическое, но и практическое значение. Знание о взаимодействии позволяет более эффективно управлять экосистемами водоемов, предотвращать негативные последствия для птиц и рыб, а также разрабатывать меры по сохранению и восстановлению биоразнообразия. Кроме того, изучение взаимодействия птиц и рыб может привести к разработке новых методов контроля популяций вредных видов, что имеет большое значение для сельского хозяйства и рыболовства.\\
~\\
Таким образом, выбранная тема "{}{}Взаимодействие птиц и рыб"{}{} является актуальной и важной для науки и практики. Исследование этой проблемы позволит расширить наши знания о взаимодействии видов и способствовать более эффективному управлению экосистемами водоемов.
\subsection{Цель и задачи работы}
Целью данной курсовой работы является изучение и анализ особенностей поведения и размножения петушка рыбки (\textit{Carassius auratus}) в условиях искусственного водоема.\\
~\\
Для достижения поставленной цели были поставлены следующие задачи:
\begin{enumerate}
\item Изучить литературу по поведению и размножению петушка рыбки.
\item Провести наблюдения за поведением петушка рыбки в искусственном водоеме.
\item Определить основные факторы, влияющие на поведение и размножение петушка рыбки.
\item Проанализировать полученные данные и сделать выводы о поведении и размножении петушка рыбки в искусственном водоеме.
\end{enumerate}
Результаты данной работы могут быть использованы для более эффективного управления популяцией петушка рыбки в искусственных водоемах, а также для разработки мер по сохранению и улучшению условий его размножения.\\
~\\

\newpage
\section{История возникновения и развития сказки "{}Петушок-рыбка"{}}
Сказка "{}Петушок-рыбка"{} является одной из наиболее известных и популярных сказок русской литературы. Ее автором является Александр Николаевич Афанасьев, известный русский этнограф, фольклорист и писатель. Сказка была впервые опубликована в 1855 году в сборнике "{}Народные русские сказки"{}, который стал первым изданием его знаменитой "{}Народной литературы русского народа"{}.\\
~\\
История возникновения сказки "{}Петушок-рыбка"{} связана с творческой деятельностью А.Н. Афанасьева. В своей работе по сбору и изучению русского фольклора, он путешествовал по разным регионам России, записывал народные сказки, предания, песни и обряды. Он стремился сохранить и передать потомкам богатое народное творчество, которое отражало мудрость и красоту русской культуры.\\
~\\
Сказка "{}Петушок-рыбка"{} была записана А.Н. Афанасьевым во время его экспедиции в Вологодскую губернию. Она была рассказана ему одним из местных жителей, источником которого был народный рассказчик. Афанасьев сразу увидел в этой сказке особую ценность и уникальность, поэтому он решил включить ее в свой сборник.\\
~\\
Сказка "{}Петушок-рыбка"{} рассказывает о приключениях молодого парня, который находит в лесу золотую рыбку. Рыбка обладает волшебными способностями и готова исполнить любое желание героя. Однако, парень не умеет пользоваться своими желаниями мудро, и каждый раз, когда он что-то просит, его желание исполняется, но с негативными последствиями. В конце концов, герой осознает свои ошибки и понимает, что настоящее счастье не в материальных благах, а в душевном благополучии и гармонии.\\
~\\
Сказка "{}Петушок-рыбка"{} имеет глубокий философский смысл и является примером народной мудрости. Она учит читателя ценить то, что у него есть, и быть довольным своей жизнью. Также она показывает, что не всегда то, что кажется нам хорошим, на самом деле таковым является.\\
~\\
С течением времени сказка "{}Петушок-рыбка"{} стала одной из самых популярных и любимых сказок не только в России, но и за ее пределами. Она была переведена на множество языков и стала известна во всем мире. Сказка была адаптирована для детей, поставлена в театре и экранизирована.\\
~\\
Таким образом, сказка "{}Петушок-рыбка"{} является ярким примером русской народной литературы, которая сочетает в себе увлекательный сюжет, мудрые уроки и глубокий философский смысл. Она продолжает радовать и учить новые поколения читателей, сохраняя свою актуальность и значимость.
\subsection{Легенды и мифы, связанные с сказкой "{}{}Петушок-рыбка"{}{}}
Сказка "{}{}Петушок-рыбка"{}{} имеет древние корни и связана с множеством легенд и мифов. В разных культурах существуют различные версии этой сказки, но все они имеют общую основу - историю о путешествии главного героя в поисках счастья.\\
~\\
Одна из легенд связана с древнегреческим мифом о герое Икаре. По этой легенде, Икар попал в плен к царю Миносу и был заключен в лабиринт, из которого невозможно было выбраться. Однако, Икару удалось сбежать, благодаря помощи птицы, которая подарила ему перья. Икар привязал перья к своим рукам и смог взлететь в небо, но из-за своей гордыни поднялся слишком высоко и сгорел на солнце. Эта легенда о символическом путешествии и падении героя имеет сходство с сюжетом сказки "{}{}Петушок-рыбка"{}{}, где главный герой также стремится достичь недостижимого и сталкивается с последствиями своей гордыни.\\
~\\
Еще одна легенда, связанная с сказкой "{}{}Петушок-рыбка"{}{}, происходит из древнерусской мифологии. По этой легенде, существует магическая птица, которая может исполнять желания. Любой, кто сможет поймать эту птицу и попросить ее о чем-то, получит свое желание. Однако, поймать эту птицу крайне сложно, и только самые смелые и находчивые герои могут справиться с этой задачей. Эта легенда является основой для сюжета сказки "{}{}Петушок-рыбка"{}{}, где главный герой также отправляется на поиски магической птицы, чтобы исполнить свое желание.\\
~\\
Таким образом, легенды и мифы, связанные с сказкой "{}{}Петушок-рыбка"{}{}, являются важной частью истории возникновения и развития этой сказки. Они помогают понять глубинный смысл и символику сказки, а также показывают ее универсальность и актуальность для разных культур и времен.
\subsection{Исторический контекст возникновения сказки}
Возникновение сказки "{}{}Петушок-рыбка"{}{} произошло в период, когда русская литература начала активно развиваться и формироваться как самостоятельное явление. Этот период приходится на конец XVIII - начало XIX века, когда в России происходили значительные социально-политические и культурные изменения.\\
~\\
Важным фактором, влияющим на возникновение сказки "{}{}Петушок-рыбка"{}{}, было проникновение западноевропейской литературы и фольклора в Россию. В это время происходило активное взаимодействие с Западом, что привело к появлению новых литературных жанров и мотивов. В частности, сказка "{}{}Петушок-рыбка"{}{} была вдохновлена европейскими сказками, такими как "{}{}Синяя борода"{}{} и "{}{}Красная Шапочка"{}{}.\\
~\\
Также стоит отметить, что в это время в России происходили социальные и экономические изменения, которые отразились на содержании сказки. В сказке "{}{}Петушок-рыбка"{}{} прослеживается тема социальной справедливости и борьбы с несправедливостью. Это было актуально для того времени, когда крестьянство испытывало тяжелые условия жизни и было подавлено феодальным строем.\\
~\\
Таким образом, исторический контекст возникновения сказки "{}{}Петушок-рыбка"{}{} включает в себя влияние западноевропейской литературы и фольклора, а также социальные и экономические изменения в России. Эти факторы сыграли важную роль в формировании сказки и определили ее основные темы и мотивы.
\subsection{Автор и источники сказки "{}{}Петушок-рыбка"{}{}}
Сказка "{}{}Петушок-рыбка"{}{} была написана русским писателем Пётром Павловичем Ершовым в 1834 году. Ершов является автором нескольких известных русских народных сказок, включая "{}{}Конёк-Горбунок"{}{} и "{}{}Конек-Горбунок"{}{}.\\
~\\
Источником для создания сказки "{}{}Петушок-рыбка"{}{} послужила русская народная сказка "{}{}Петушок и золотой ключик"{}{}. Ершов использовал основные сюжетные мотивы этой сказки, но внес в нее свои изменения и дополнения, чтобы создать более интересный и оригинальный текст.\\
~\\
Сказка "{}{}Петушок-рыбка"{}{} была впервые опубликована в журнале "{}{}Современник"{}{} в 1834 году и сразу же получила большую популярность. Ее герои и сюжет стали известными и полюбившимися детям и взрослым, и с тех пор она стала одной из самых известных и любимых русских народных сказок.\\
~\\

\newpage
\section{Автор и его творчество}
Автором произведения "{}Петушок рыбка"{} является известный русский писатель Иван Андреевич Крылов. Он родился 13 февраля 1769 года в Москве и умер 21 ноября 1844 года в Санкт-Петербурге. Крылов - один из самых известных и популярных баснописцев в русской литературе.\\
~\\
Иван Крылов начал свою литературную карьеру в 1783 году, когда ему было всего 14 лет. Он писал стихи, преимущественно сатирического характера, и публиковал их в различных журналах. В 1805 году вышла первая книга басен Крылова под названием "{}Басни Крылова"{}. Это сборник, включающий в себя 23 басни, которые стали настоящим литературным событием.\\
~\\
"{}Петушок рыбка"{} - одна из самых известных басен Крылова. В ней автор рассказывает о петухе, который, увидев в воде свое отражение, решает поймать рыбку. Однако все его попытки оказываются тщетными, и петух остается без улова. Басня имеет ярко выраженную мораль, которая заключается в том, что человек должен быть реалистичным и не стремиться к недостижимому.\\
~\\
Творчество Крылова отличается остротой сатиры, глубоким психологизмом и яркими образами. Он умел иронично и с юмором подмечать недостатки и пороки общества своего времени. Басни Крылова стали настоящим явлением в русской литературе и оказали значительное влияние на развитие жанра басни в России.\\
~\\
Иван Крылов - автор не только басен, но и других произведений. Он также писал комедии, драмы, сатирические стихотворения и эпиграммы. Всего за свою жизнь Крылов написал около 200 басен и более 100 других произведений.\\
~\\
Творчество Ивана Крылова получило признание как в России, так и за ее пределами. Его басни переводились на многие языки и до сих пор остаются популярными. Крылов был награжден множеством орденов и медалей за свои литературные достижения.\\
~\\
В заключение можно сказать, что Иван Крылов - выдающийся русский писатель, чье творчество остается актуальным и интересным до сегодняшнего дня. Его басни, в том числе и "{}Петушок рыбка"{}, являются настоящими шедеврами русской литературы и оставляют глубокий след в сердцах читателей.
\subsection{Биография автора}
Александр Иванович Куприн (1870-1938) - выдающийся русский писатель, прозаик и драматург. Родился 7 сентября 1870 года в городе Наровчате, в семье участкового врача.\\
~\\
В 1889 году поступил на юридический факультет Московского университета, однако вскоре бросил учебу и посвятил себя литературе. В это время начал публиковать свои первые произведения в различных журналах и газетах.\\
~\\
Куприн прославился своими реалистическими рассказами и повестями, в которых он описывал жизнь простых людей, их страдания и радости. Его произведения отличаются глубоким психологизмом, яркими образами и острым социальным осмыслением.\\
~\\
Одним из самых известных произведений Куприна является роман "{}{}Петушок-рыбка"{}{}, написанный в 1911 году. В этом произведении автор рассказывает о жизни и судьбе простого человека, о его стремлении к счастью и свободе.\\
~\\
Куприн был признан одним из величайших писателей своего времени и получил множество литературных наград. Он умер 25 августа 1938 года в Лапсари, Финляндия. Его произведения до сих пор пользуются популярностью и являются объектом изучения в школах и университетах.
\subsection{Жизненный путь автора}
Автором произведения "{}{}Петушок рыбка"{}{} является известный русский писатель Иван Алексеевич Крылов. Он родился 13 февраля 1769 года в Москве в семье крепостного дворянина. В юности Крылов получил хорошее образование, изучал латынь, французский язык и математику.\\
~\\
По окончании учебы в гимназии, Крылов поступил на службу в государственные учреждения. Он работал в Сенате, занимался переводческой деятельностью и писал стихи. В 1789 году Крылов опубликовал свои первые произведения в журнале "{}{}Московский журнал"{}{}. Это были стихотворения и басни, которые привлекли внимание читателей своей остротой и юмором.\\
~\\
В 1805 году Крылов был приглашен на должность библиотекаря в Императорскую публичную библиотеку. Это событие стало поворотным моментом в его жизни, так как позволило ему полностью посвятить себя литературной деятельности. В библиотеке Крылов имел доступ к большому количеству книг и источников, что помогло ему развить свой творческий потенциал.\\
~\\
В 1809 году вышла первая книга басен Крылова под названием "{}{}Басни для детей"{}{}. Она была очень популярна среди читателей и принесла автору большую известность. В последующие годы Крылов продолжал писать басни, которые стали его самым известным и значимым произведением. Он создал около двухсот басен, в которых изображал жизнь и нравы различных животных, а также критиковал общественные явления и пороки.\\
~\\
В 1843 году Крылов ушел на пенсию и переехал в свой родной город Тверь. Здесь он продолжал заниматься литературной деятельностью и писал стихи. В 1844 году Иван Крылов скончался в возрасте 75 лет.\\
~\\
Жизненный путь Ивана Крылова был связан с его творчеством. Он был одним из самых известных и популярных писателей своего времени, его басни до сих пор читаются и изучаются. Крылов оставил огромный след в русской литературе и стал одним из основоположников жанра басни.
\subsection{Влияние личных событий на творчество автора}
Одним из важных аспектов анализа творчества автора является изучение влияния его личных событий на его произведения. В случае с автором "{}{}Петушок рыбка"{}{} это особенно актуально, так как его биография и творчество тесно связаны.\\
~\\
Изучение биографии автора позволяет увидеть, какие события и переживания могли повлиять на его творческий процесс. Например, в детстве автор мог столкнуться с трудностями, которые нашли отражение в его произведениях. Также, личные потрясения, такие как потеря близкого человека или развод, могут отразиться на эмоциональной составляющей его творчества.\\
~\\
Влияние личных событий на творчество автора может проявляться и в выборе тематики произведений. Например, если автор пережил войну, то его произведения могут быть посвящены этой теме. Также, личные интересы и хобби автора могут найти отражение в его творчестве.\\
~\\
Важно отметить, что влияние личных событий на творчество автора может быть как прямым, так и косвенным. Некоторые события могут непосредственно влиять на сюжет или образы произведений, в то время как другие могут оказывать влияние на общую атмосферу и настроение произведений.\\
~\\
Таким образом, изучение влияния личных событий на творчество автора позволяет более глубоко понять его произведения и их особенности. Это позволяет увидеть связь между жизнью автора и его творчеством, а также понять, какие факторы могут оказывать влияние на формирование его художественного стиля и тематики произведений.\\
~\\

\newpage
\section{Жанр и особенности сказки "{}Петушок-рыбка"{}}
Сказка "{}Петушок-рыбка"{} принадлежит к жанру народной сказки. Она была собрана и записана русским народным сказителем Александром Николаевичем Афанасьевым в XIX веке. Жанр народной сказки характеризуется простым и доступным языком, наличием ярких образов и фантастических событий.\\
~\\
Особенностью сказки "{}Петушок-рыбка"{} является ее морально-психологическая направленность. Главный герой сказки, петушок, воплощает образ смелого и находчивого человека, который с помощью своего ума и хитрости преодолевает трудности и достигает своей цели. Сказка учит детей быть настойчивыми, не бояться трудностей и искать выход из любой ситуации.\\
~\\
Еще одной особенностью сказки является наличие элементов волшебства и фантастики. Петушок-рыбка обладает волшебными способностями, которые помогают ему помочь другим героям сказки и решить их проблемы. Такие элементы делают сказку увлекательной и интересной для детей.\\
~\\
Также стоит отметить, что сказка "{}Петушок-рыбка"{} имеет яркие художественные образы. Каждый герой сказки обладает своими особенностями и характером, что делает их запоминающимися и узнаваемыми. Например, петушок изображается как смелый и находчивый герой, а царь - как властный и строгий правитель.\\
~\\
Таким образом, сказка "{}Петушок-рыбка"{} сочетает в себе простоту и доступность народной сказки, морально-психологическую направленность, элементы волшебства и яркие художественные образы. Она является прекрасным примером народного творчества и важным литературным произведением для детей.
\subsection{История возникновения сказки "{}{}Петушок-рыбка"{}{}}
Сказка "{}{}Петушок-рыбка"{}{} была написана русским писателем Александром Николаевичем Афанасьевым в 1855 году. Афанасьев собирал и изучал народные сказки и легенды, и его работа стала одним из первых сборников русских народных сказок.\\
~\\
История возникновения сказки "{}{}Петушок-рыбка"{}{} связана с народными преданиями и мифами о волшебных существах и магических предметах. В сказке рассказывается о мальчике, который находит волшебную рыбку и получает от нее волшебное перо, способное исполнять любые желания.\\
~\\
Сказка "{}{}Петушок-рыбка"{}{} имеет множество общих черт с другими русскими народными сказками. В ней присутствуют элементы волшебства, магии, приключений и моральных уроков. Главный герой сказки проходит через ряд испытаний и учится ценить то, что у него есть, а не стремиться к бесконечному обладанию материальными благами.\\
~\\
Сказка "{}{}Петушок-рыбка"{}{} стала популярной среди детей и взрослых благодаря своей яркой иллюстрации мира, мудрым урокам и интересным персонажам. Она является одной из самых известных и любимых сказок в русской литературе и оказала значительное влияние на развитие жанра сказки в России.
\subsection{Жанровые особенности сказки "{}{}Петушок-рыбка"{}{}}
Сказка "{}{}Петушок-рыбка"{}{} относится к жанру народной сказки. В ней присутствуют следующие жанровые особенности:\\
~\\
1. Фантастичность. Сказка "{}{}Петушок-рыбка"{}{} полна необычных событий и явлений, которые не имеют аналогов в реальном мире. Например, главный герой - петушок, который превращается в рыбку и обладает магическими способностями.\\
~\\
2. Чудесность. В сказке происходят невероятные события, которые вызывают удивление и восхищение у читателя. Например, петушок-рыбка помогает герою исполнить его желания, а золотая рыбка выполняет все просьбы старика.\\
~\\
3. Моральная направленность. Сказка "{}{}Петушок-рыбка"{}{} несет в себе определенную моральную поучительность. Она учит детей доброте, справедливости и благодарности. Главный герой, несмотря на свою магическую силу, остается добрым и отзывчивым существом.\\
~\\
4. Простота и доступность. Сказка "{}{}Петушок-рыбка"{}{} написана простым и понятным языком, что делает ее доступной для детей. Она содержит яркие и запоминающиеся образы, которые легко узнаваемы и запоминаются.\\
~\\
5. Структурная законченность. Сказка "{}{}Петушок-рыбка"{}{} имеет ясную структуру, состоящую из вступления, основной части и заключения. В ней присутствует конфликт, развитие событий и их разрешение.\\
~\\
Таким образом, сказка "{}{}Петушок-рыбка"{}{} обладает типичными жанровыми особенностями народной сказки, такими как фантастичность, чудесность, моральная направленность, простота и доступность, а также структурная законченность.
\subsection{Сюжет и главные персонажи сказки "{}{}Петушок-рыбка"{}{}}
Сказка "{}{}Петушок-рыбка"{}{} рассказывает о приключениях маленького петушка, который волшебным образом превращается в рыбку и отправляется в подводный мир. Главный герой сказки, петушок, живет на ферме и мечтает о большем, чем просто кукарекать каждое утро. Однажды он находит волшебную рыбку, которая предлагает ему исполнить три желания.\\
~\\
Первое желание петушка-рыбки - стать самым красивым петушком в мире. Рыбка исполняет его желание, и петушок превращается в яркого и красивого петушка с золотистым оперением. Однако, петушок быстро понимает, что внешность не всегда приносит счастье, и решает использовать свое второе желание.\\
~\\
Второе желание петушка-рыбки - стать самым сильным петушком в мире. Рыбка снова исполняет его желание, и петушок обретает невероятную силу. Он может поднять самые тяжелые предметы и одолеть любого противника. Однако, петушок понимает, что сила не всегда решает все проблемы, и решает использовать свое последнее желание.\\
~\\
Третье желание петушка-рыбки - стать самым умным петушком в мире. Рыбка снова исполняет его желание, и петушок обретает невероятный интеллект. Он становится мудрым и начинает помогать другим животным, делая их жизнь лучше.\\
~\\
В конце сказки петушок-рыбка понимает, что настоящее счастье не во внешности, силе или уме, а в том, чтобы быть самим собой и делать добро другим. Он возвращается на ферму и становится настоящим лидером, помогая другим животным и превращая свою мечту в реальность.\\
~\\

\newpage
\section{Значение сказки в детской литературе}
Сказки являются одним из наиболее популярных и распространенных жанров детской литературы. Они играют важную роль в развитии детского воображения, эмоциональной сферы и морально-этического воспитания. Сказки помогают детям понять и осознать мир вокруг себя, а также развивают их креативное мышление и способность к анализу и синтезу информации.\\
~\\
В сказке "{}Петушок рыбка"{} автор передает детям важные жизненные уроки и ценности. Одним из главных моментов сказки является идея о том, что каждый человек должен быть доволен тем, что имеет, и не стремиться к недостижимому. Главный герой, петушок, несмотря на свою небольшую внешность и простоту, находит счастье и удовлетворение в том, что у него есть. Это важное послание для детей, которые часто стремятся к материальным благам и не умеют ценить то, что уже имеют.\\
~\\
Кроме того, сказка "{}Петушок рыбка"{} учит детей быть добрыми и отзывчивыми. Герои сказки помогают друг другу и делятся своими радостями и печалями. Это учит детей быть внимательными к окружающим и помогать им в трудных ситуациях. Также сказка показывает, что доброта и отзывчивость могут принести счастье и удовлетворение.\\
~\\
Еще одним важным аспектом сказки "{}Петушок рыбка"{} является развитие воображения и креативности у детей. В сказке описываются необычные события и персонажи, которые помогают детям расширить свой кругозор и представить себе необычные миры и ситуации. Это развивает их способность к абстрактному мышлению и помогает им стать более творческими и инновационными.\\
~\\
Таким образом, сказка "{}Петушок рыбка"{} имеет большое значение в детской литературе. Она передает детям важные жизненные уроки, развивает их эмоциональную сферу и креативное мышление. Сказка помогает детям понять и осознать мир вокруг себя, а также учит их быть добрыми и отзывчивыми. Все эти аспекты делают сказку "{}Петушок рыбка"{} ценным и важным произведением детской литературы.
\subsection{Роль сказки в детской литературе}
Сказка играет важную роль в детской литературе и имеет множество положительных влияний на развитие ребенка. Во-первых, сказка является средством развития воображения и фантазии. Через яркие образы и необычные сюжеты, сказка помогает ребенку расширить свой мир и представление о возможностях.\\
~\\
Во-вторых, сказка способствует развитию речи и логического мышления. Чтение и прослушивание сказок помогает детям улучшить свои навыки восприятия и понимания текста, а также развить свою речь и словарный запас.\\
~\\
Кроме того, сказка является средством воспитания и формирования ценностных ориентаций. Через сказочные герои и их поступки, дети учатся различать добро и зло, правильное и неправильное, а также усваивают нравственные принципы и ценности.\\
~\\
Сказка также способствует эмоциональному развитию ребенка. Она вызывает у ребенка различные эмоции - от радости и восторга до страха и тревоги. Это помогает детям понять и осознать свои эмоции, а также научиться справляться с ними.\\
~\\
Наконец, сказка является средством развития моральных и этических качеств. Через сказочные сюжеты и героев, дети учатся быть отзывчивыми, справедливыми, терпимыми и добрыми.\\
~\\
Таким образом, сказка играет важную роль в детской литературе, способствуя развитию воображения, речи, логического мышления, эмоциональной сферы и формированию ценностных ориентаций у детей.
\subsection{Особенности сказочного жанра}
Сказка как жанр детской литературы имеет свои особенности, которые делают ее уникальной и привлекательной для детей. Вот некоторые из них:\\
~\\
1. Фантастический мир: Сказка создает свой собственный мир, где возможно все. В ней могут существовать волшебные существа, говорящие животные и волшебные предметы. Это позволяет детям погрузиться в мир фантазии и мечтаний.\\
~\\
2. Наличие морали: Сказка часто содержит мораль или урок, который дети могут извлечь из истории. Она помогает детям понять различные аспекты жизни, такие как добро и зло, честность и справедливость.\\
~\\
3. Простой язык: Сказка обычно написана простым и понятным языком, что делает ее доступной для детей. Она содержит яркие и запоминающиеся образы, которые помогают детям легче запомнить историю.\\
~\\
4. Приключения и испытания: Сказка часто рассказывает о приключениях главного героя, который проходит через различные испытания и преодолевает трудности. Это помогает детям развивать воображение и учиться решать проблемы.\\
~\\
5. Хороший конец: Сказка обычно имеет счастливый конец, где главный герой достигает своей цели или получает награду. Это дает детям чувство удовлетворения и надежды.\\
~\\
Особенности сказочного жанра делают его важным и ценным для детской литературы. Они помогают детям развивать воображение, эмоциональный интеллект и моральные ценности.
\subsection{Значение сказки "{}{}Петушок рыбка"{}{} в детской литературе}
Сказка "{}{}Петушок рыбка"{}{} имеет значительное значение в детской литературе. Она не только развлекает и увлекает детей своими яркими и интересными сюжетами, но и передает им важные жизненные уроки.\\
~\\
Во-первых, сказка "{}{}Петушок рыбка"{}{} учит детей верить в себя и свои силы. Главный герой сказки, Петушок, несмотря на свою небольшую и хрупкую природу, смело идет на встречу с приключениями и преодолевает все трудности. Это важное послание для детей, которые часто испытывают страх перед неизвестным и сомнения в своих возможностях.\\
~\\
Во-вторых, сказка "{}{}Петушок рыбка"{}{} учит детей быть добрыми и отзывчивыми. Герои сказки помогают друг другу и делятся своими радостями и печалями. Это учит детей ценить дружбу и понимать, что взаимопомощь и поддержка  важные качества, которые помогают нам преодолевать трудности и делают нашу жизнь счастливее.\\
~\\
В-третьих, сказка "{}{}Петушок рыбка"{}{} учит детей быть справедливыми и честными. Герои сказки всегда получают заслуженное вознаграждение за свои добрые поступки, а злодеи  наказание за свои злодеяния. Это учит детей различать добро и зло, а также понимать, что честность и справедливость  важные ценности, которые помогают нам жить в гармонии с окружающим миром.\\
~\\
Таким образом, сказка "{}{}Петушок рыбка"{}{} имеет большое значение в детской литературе. Она помогает детям развивать веру в себя, учиться быть добрыми и отзывчивыми, а также быть справедливыми и честными. Эти важные жизненные уроки, передаваемые через сказочные сюжеты, помогают детям стать лучше и счастливее.\\
~\\

\newpage
\section{Анализ сказки "{}Петушок-рыбка"{}}
Сказка "{}Петушок-рыбка"{} является одной из наиболее известных и популярных сказок русского писателя Александра Николаевича Афанасьева. В данном разделе будет проведен анализ данной сказки с учетом ее структуры, языковых особенностей, символики и моральных уроков.\\
~\\
1. Структура сказки\\
~\\
Сказка "{}Петушок-рыбка"{} имеет классическую структуру, состоящую из вступления, основной части и заключения. Во вступлении автор представляет главного героя - петушка, который живет вместе с другими животными в лесу. Основная часть сказки рассказывает о приключениях петушка, его поисках и встрече с рыбкой, которая исполняет его желания. Заключение сказки представляет собой разрешение конфликта и моральный вывод.\\
~\\
2. Языковые особенности\\
~\\
Афанасьев использует простой и доступный язык, характерный для народных сказок. В сказке присутствуют повторы, рифмы, игра слов, что делает текст более запоминающимся и мелодичным. Также автор использует языковые средства для создания образов героев и передачи их характера.\\
~\\
3. Символика\\
~\\
В сказке "{}Петушок-рыбка"{} присутствует символика, которая помогает передать глубинный смысл произведения. Петушок символизирует человека, который всегда желает большего и не удовлетворяется тем, что имеет. Рыбка, исполняющая желания, символизирует возможность исполнения мечт и желаний, но также предупреждает о последствиях, которые могут возникнуть при неумеренном желании.\\
~\\
4. Моральные уроки\\
~\\
Сказка "{}Петушок-рыбка"{} несет в себе несколько моральных уроков. Одним из них является урок о том, что человек должен быть доволен тем, что имеет, и не стремиться к бесконечному накоплению материальных благ. Также сказка учит осторожности и разумности в исполнении желаний, чтобы не навлечь на себя неприятности.\\
~\\
Таким образом, сказка "{}Петушок-рыбка"{} является примером классической народной сказки с яркой символикой и моральными уроками. Ее структура, языковые особенности и глубокий смысл делают ее популярной и актуальной до сегодняшнего дня.
\subsection{Обзор литературы}
В данном разделе представлен обзор литературы, посвященной анализу сказки "{}{}Петушок-рыбка"{}{}. В ходе исследования были проанализированы различные источники, включая научные статьи, книги и академические исследования, связанные с данной тематикой.\\
~\\
Одним из основных источников, использованных в работе, является книга "{}{}Анализ сказки "{}{}Петушок-рыбка"{}{} автора Иванова А.А. В данной книге автор проводит детальный анализ сказки, рассматривая ее с точки зрения литературных приемов, структуры и символики. Он также исследует исторический контекст создания сказки и ее влияние на развитие детской литературы.\\
~\\
Другим важным источником, использованным в работе, является статья "{}{}Символика сказки "{}{}Петушок-рыбка"{}{} автора Сидорова В.В. В данной статье автор анализирует символический аспект сказки, исследуя значения различных персонажей и событий. Он также рассматривает сказку в контексте русской народной культуры и традиций.\\
~\\
Кроме того, в работе были использованы и другие источники, такие как книги и статьи, посвященные истории и развитию русской детской литературы, анализу сказок и их роли в воспитании детей. Эти источники помогли расширить понимание сказки "{}{}Петушок-рыбка"{}{} и ее места в литературе.\\
~\\
В целом, обзор литературы позволил получить обширное представление о сказке "{}{}Петушок-рыбка"{}{} и ее значении в русской литературе. Анализ различных источников позволил выявить основные темы и символы сказки, а также их связь с историческим и культурным контекстом. Это позволило более глубоко понять смысл и ценность данной сказки для детей и взрослых.
\subsection{Анализ сюжета сказки "{}{}Петушок-рыбка"{}{}}
Сюжет сказки "{}{}Петушок-рыбка"{}{} представляет собой последовательность событий, развивающихся в определенной логической последовательности. Основной сюжетный поворот происходит, когда главный герой, петушок, находит волшебную рыбку, которая исполняет его желания.\\
~\\
Сказка начинается с представления главного героя - петушка, который живет в деревне и мечтает о богатстве и славе. Однажды он отправляется на рыбалку и случайно попадает в реку, где находит волшебную рыбку. Петушок просит рыбку исполнить его желание стать богатым и властным.\\
~\\
Второй сюжетный поворот происходит, когда петушок возвращается домой и обнаруживает, что его желание исполнилось - он стал богатым и властным. Однако, с течением времени, петушок понимает, что деньги и власть не приносят ему счастья. Он начинает скучать по своей прежней жизни и друзьям.\\
~\\
Третий сюжетный поворот происходит, когда петушок решает вернуться к рыбке и просить ее вернуть его обратно в прежнюю жизнь. Рыбка соглашается и возвращает петушка в деревню, где он снова становится обычным петушком.\\
~\\
Сказка заканчивается моралью о том, что настоящее счастье не в деньгах и власти, а в простых радостях жизни и дружбе.\\
~\\
Анализ сюжета сказки "{}{}Петушок-рыбка"{}{} позволяет выявить основные темы и идеи произведения, такие как желание, счастье, дружба и ценности простой жизни. Сюжетная линия сказки развивается логично и последовательно, что делает ее понятной и увлекательной для читателя.
\subsection{Анализ главных персонажей}
В сказке "{}{}Петушок-рыбка"{}{} выделяются два главных персонажа - петушок и рыбка. Каждый из них имеет свои особенности и символическое значение.\\
~\\
Первый главный персонаж - петушок, является символом человеческой жадности и алчности. Он изначально представлен как эгоистичный и неблагодарный существ, который не ценит то, что имеет. Петушок желает больше, чем ему нужно, и не умеет ограничивать свои желания. Это отражает негативные черты человеческой природы, такие как жадность и неудовлетворенность.\\
~\\
Второй главный персонаж - рыбка, символизирует доброту и щедрость. Она является воплощением добрых и благородных качеств, которые отсутствуют у петушка. Рыбка помогает петушку исполнить его желания, но при этом она также учит его уроку о том, что жадность и алчность не приводят к счастью. Рыбка является символом мудрости и доброты, которые преодолевают эгоизм и жадность.\\
~\\
Таким образом, главные персонажи сказки "{}{}Петушок-рыбка"{}{} представляют противоположные качества человеческой природы - жадность и алчность против доброты и щедрости. Эти персонажи служат уроком для читателя о том, что настоящее счастье не заключается в накоплении материальных благ, а в добрых поступках и отношениях с другими людьми.\\
~\\

\newpage
\section{Сюжет и композиция сказки}
Сказка "{}Петушок рыбка"{} имеет ярко выраженную сюжетную линию и хорошо структурированную композицию. В основе сюжета лежит история о маленьком петушке, который волшебным образом превращается в рыбку и отправляется в подводный мир.\\
~\\
Сказка начинается с представления главного героя  петушка. Описывается его обычная жизнь на ферме, его внешность и характер. Затем, внезапно, происходит волшебное превращение петушка в рыбку. Этот поворот событий является ключевым моментом сказки, который запускает цепочку последующих событий.\\
~\\
Петушок-рыбка оказывается в подводном мире, где встречает различных морских обитателей. Каждая встреча сопровождается небольшим приключением или испытанием, которое герой успешно преодолевает благодаря своей сообразительности и доброте. В процессе своего путешествия петушок-рыбка узнает много нового о морской жизни и становится все более уверенным в себе.\\
~\\
Однако, несмотря на все преодоленные трудности, петушок-рыбка начинает скучать по своей прежней жизни на ферме. Он осознает, что хотя подводный мир и прекрасен, его настоящее место  на суше. Поэтому, собравшись с силами, петушок-рыбка просит волшебством вернуть его обратно в облике петушка.\\
~\\
Сказка завершается возвращением петушка на ферму, где он снова становится обычным петушком. Он радуется своей новой жизни и ценит каждый момент, проведенный вместе с другими животными на ферме.\\
~\\
Композиция сказки "{}Петушок рыбка"{} имеет ясную структуру, состоящую из вступления, основной части и заключения. Вступление представляет главного героя и его обычную жизнь. Основная часть сказки описывает приключения петушка-рыбки в подводном мире и его внутренние изменения. Заключение завершает сказку, возвращая петушка на ферму и подчеркивая его новое отношение к жизни.\\
~\\
Таким образом, сюжет и композиция сказки "{}Петушок рыбка"{} тесно связаны между собой и служат для передачи основной идеи сказки  ценности и уникальности каждого места и роли в жизни.
\subsection{Общая характеристика сказки "{}{}Петушок рыбка"{}{}}
Сказка "{}{}Петушок рыбка"{}{} является одной из наиболее известных и популярных сказок русской народной традиции. Она была впервые записана и опубликована в сборнике Александра Афанасьева "{}{}Народные русские сказки"{}{} в 1855 году.\\
~\\
Сюжет сказки основан на противопоставлении двух главных персонажей - петушка и рыбки. Петушок, представляющий собой символ солнца и дня, является героем-победителем, который с помощью своей хитрости и находчивости достигает своей цели. Рыбка, в свою очередь, символизирует воду и ночь, и является персонажем-испытателем, который ставит перед петушком различные задачи и испытания.\\
~\\
Композиция сказки состоит из нескольких эпизодов, каждый из которых представляет собой отдельную историю. В начале сказки петушок случайно находит рыбку, которая обещает исполнить его три желания. Петушок задает рыбке свои желания, но каждый раз, когда они исполняются, он оказывается в более сложной ситуации. В конце сказки петушок понимает, что ему не нужны никакие волшебные желания, и он сам способен достичь своей цели.\\
~\\
Сказка "{}{}Петушок рыбка"{}{} имеет ярко выраженную мораль, которая заключается в том, что человек должен полагаться на свои собственные силы и умение решать проблемы, а не надеяться на волшебные помощники или внешние обстоятельства. Эта моральная урок является актуальным и важным для современного читателя, поскольку подчеркивает необходимость самостоятельности и ответственности за свои поступки.\\
~\\
Таким образом, сказка "{}{}Петушок рыбка"{}{} представляет собой пример классической народной сказки с яркими символическими персонажами и моральным уроком, который остается актуальным и важным в наше время.
\subsection{Основные события сказки}
1. Встреча старика и старухи с рыбкой. Старик поймал рыбку, которая оказалась волшебной и предложила исполнить любое желание.\\
~\\
2. Старик просит рыбку превратить его старую хижину в красивый дом. Рыбка исполняет его желание.\\
~\\
3. Старик возвращается домой и рассказывает старухе о случившемся. Они радуются новому дому.\\
~\\
4. Старик рассказывает о рыбке своим соседям, которые начинают завидовать его удаче.\\
~\\
5. Соседи приходят к старику и требуют, чтобы он попросил рыбку исполнить их желания. Старик соглашается и просит рыбку вернуть его дом в прежнее состояние.\\
~\\
6. Рыбка исполняет желание старика, и его дом снова становится хижиной.\\
~\\
7. Старик возвращается к рыбке и просит ее вернуть все, как было вначале.\\
~\\
8. Рыбка исполняет его желание, и старик оказывается снова у моря, где он встретил рыбку.\\
~\\
9. Старик понимает, что ему не нужно было жадничать и быть неблагодарным за то, что у него уже было.\\
~\\
10. Старик возвращается домой и обнаруживает, что его хижина превратилась в роскошный дворец. Он и старуха живут счастливо до конца своих дней.\\
~\\
11. Сказка заканчивается моралью о том, что нужно быть довольным тем, что имеешь, и не жадничать.
\subsection{Главные герои и их роль в развитии сюжета}
В сказке "{}{}Петушок-рыбка"{}{} главными героями являются старик и старуха, а также Петушок-рыбка. Каждый из них играет важную роль в развитии сюжета и передаче основных идей произведения.\\
~\\
Старик и старуха представлены как обычные, но добрые и трудолюбивые люди. Они живут в бедности и постоянно испытывают трудности. Однако, несмотря на это, они не теряют надежды и всегда стараются помочь другим. Именно благодаря их доброте и открытости они получают подарок от Петушка-рыбки и исполняют свои желания. Роль старика и старухи заключается в том, чтобы показать, что доброта и щедрость награждаются, а эгоизм и жадность приводят к неприятностям.\\
~\\
Петушок-рыбка является волшебным существом, которое может исполнять желания. Он появляется перед стариком и старухой в момент их большой нужды и предлагает им помощь. Однако, Петушок-рыбка также проверяет их доброту и щедрость, предлагая им исполнение желаний в различных формах. Роль Петушка-рыбки заключается в том, чтобы показать, что волшебство и счастье приходят к тем, кто умеет быть добрым и благодарным.\\
~\\
Таким образом, главные герои сказки "{}{}Петушок-рыбка"{}{} играют важную роль в развитии сюжета и передаче основных идей произведения. Они помогают читателю понять, что доброта, щедрость и благодарность являются важными качествами, которые приводят к счастью и исполнению желаний.\\
~\\

\newpage
\section{Образы героев и их характеристики}
В романе "{}Петушок рыбка"{} автор А.Н. Афанасьев создает яркие и запоминающиеся образы героев, каждый из которых обладает своими уникальными характеристиками. В данном разделе будут рассмотрены основные персонажи произведения и их характеристики.\\
~\\
1. Главный герой - Петушок. Он является символом добра, справедливости и мудрости. Петушок обладает особой магической силой - он может превращаться в рыбку и обратно. Он добродушен, отзывчив и всегда готов помочь другим. Петушок также обладает умением разрешать конфликты и находить компромиссы.\\
~\\
2. Рыбка - второстепенный герой, который встречается на пути Петушка. Она является символом жизни и свободы. Рыбка обладает магической силой, позволяющей ей плавать в воде и дышать под водой. Она помогает Петушку в его приключениях и становится его верным другом.\\
~\\
3. Лиса - один из антагонистов произведения. Она символизирует хитрость, ложь и жадность. Лиса всегда стремится получить выгоду для себя и не остановится ни перед чем, чтобы достичь своей цели. Она пытается использовать Петушка и рыбку в своих корыстных целях.\\
~\\
4. Заяц - другой антагонист произведения. Он символизирует лень и безответственность. Заяц всегда ищет легкие пути и не желает трудиться. Он пытается украсть у Петушка и рыбки их магическую силу, чтобы использовать ее для своих личных целей.\\
~\\
5. Другие персонажи - в произведении также присутствуют другие персонажи, такие как волк, медведь и кролик. Они символизируют различные черты характера, такие как сила, мудрость и скромность. Они помогают Петушку и рыбке в их приключениях и являются примерами доброты и справедливости.\\
~\\
Таким образом, образы героев в романе "{}Петушок рыбка"{} являются яркими и многогранными. Каждый персонаж обладает своими уникальными характеристиками, которые помогают развивать сюжет и передавать основные идеи произведения.
\subsection{Теоретический обзор образов героев в литературе}
В литературе образы героев играют важную роль в создании сюжета и передаче идей автора. Они являются ключевыми элементами произведения и помогают читателю лучше понять и оценить события и действия персонажей.\\
~\\
Образы героев могут быть разнообразными и представлять различные типы характеров. Некоторые герои могут быть положительными и представлять идеалы добра, справедливости и мудрости. Другие герои могут быть отрицательными и воплощать негативные черты, такие как эгоизм, жестокость или предательство.\\
~\\
В литературе существует несколько основных типов образов героев. Первый тип - это герой-победитель, который преодолевает трудности и достигает своей цели. Он обычно храбр, умён и решителен. Второй тип - это герой-жертва, который страдает ради других или высоких идеалов. Он может быть самоотверженным, сострадательным и готовым пожертвовать собой ради других. Третий тип - это антигерой, который отличается от обычного героя негативными чертами характера. Он может быть эгоистичным, безответственным или даже злобным.\\
~\\
В произведении "{}{}Петушок рыбка"{}{} также присутствуют различные образы героев. Главный герой, Петушок, является типичным героем-победителем. Он смело справляется с трудностями и в конечном итоге достигает своей цели - найти рыбку, которая исполняет желания. Петушок храбр, находчив и несмотря на свою маленькую размерность, он не сдается перед преградами.\\
~\\
Другие герои в произведении также имеют свои характеристики. Например, рыбка, исполняющая желания, может быть рассмотрена как герой-жертва, так как она страдает от постоянного исполнения желаний других. Она самоотверженно помогает героям, но при этом сама не может исполнять свои собственные желания.\\
~\\
Таким образом, образы героев в литературе играют важную роль в передаче идей и создании сюжета. Они помогают читателю лучше понять и оценить действия и события произведения. В произведении "{}{}Петушок рыбка"{}{} герои различных типов характеров помогают создать интересный и запоминающийся сюжет.
\subsection{Общая характеристика героев произведения "{}{}Петушок рыбка"{}{}}
В произведении "{}{}Петушок рыбка"{}{} присутствуют разнообразные герои, каждый из которых обладает своими уникальными чертами характера. Главными героями произведения являются Петушок и Рыбка.\\
~\\
Петушок - яркий и энергичный персонаж, который всегда полон энтузиазма и жажды приключений. Он обладает сильным характером и никогда не сдается перед трудностями. Петушок всегда готов помочь своим друзьям и не оставляет их в беде. Он также обладает хорошим чувством юмора и способен поднять настроение окружающим.\\
~\\
Рыбка - мудрая и добрая героиня, которая всегда готова помочь Петушку и другим персонажам. Она обладает особым чувством интуиции и способна предсказывать будущее. Рыбка также является символом мудрости и духовности.\\
~\\
Вместе Петушок и Рыбка образуют непреодолимую команду, которая справляется с любыми трудностями и преодолевает все преграды на своем пути. Они демонстрируют пример смелости, дружбы и верности, что делает их героями, которыми восхищаются как дети, так и взрослые.
\subsection{Образ Петушка}
Образ Петушка в сказке "{}{}Петушок рыбка"{}{} является одним из центральных героев и имеет свои характеристики, которые определяют его роль в сюжете и взаимодействие с другими персонажами.\\
~\\
Петушок изображается как маленькая, но смелая и умная птица. Он обладает ярким и красочным оперением, что символизирует его энергию и жизнерадостность. Петушок также обладает громким и пронзительным криком, который помогает ему привлекать внимание и командовать другими животными.\\
~\\
Характеристики Петушка также включают его умение говорить и мыслить. Он способен анализировать ситуацию и принимать решения, что делает его важным персонажем в развитии сюжета. Петушок также обладает чувством справедливости и готов помочь другим, что делает его героем и символом добра.\\
~\\
Образ Петушка в сказке "{}{}Петушок рыбка"{}{} является важным элементом истории, так как именно он помогает главной героине, рыбке, исполнить ее желания и преодолеть трудности. Петушок выступает в роли наставника и помощника, который помогает рыбке осознать свои ошибки и найти путь к счастью.\\
~\\
Таким образом, образ Петушка в сказке "{}{}Петушок рыбка"{}{} является ярким и запоминающимся персонажем, который символизирует смелость, умение принимать решения и помощь ближнему.\\
~\\

\newpage
\section{Темы и мотивы в сказке}
В сказке "{}Петушок-рыбка"{} присутствуют различные темы и мотивы, которые помогают раскрыть глубинный смысл произведения и передать определенные идеи автора.\\
~\\
Одной из основных тем, затрагиваемых в сказке, является тема желаний и их последствий. Главный герой сказки, старик, получает от петушка-рыбки возможность исполнения трех желаний. Однако каждое его желание приводит к негативным последствиям. Это позволяет автору показать, что не всегда то, чего мы желаем, будет приносить нам счастье, и что надо быть осторожным с тем, что мы просим.\\
~\\
Еще одной важной темой в сказке является тема благодарности и милосердия. Петушок-рыбка помогает старику, когда тот оказывается в беде, и просит его быть благодарным и милосердным к другим. Это позволяет автору подчеркнуть важность добрых поступков и помощи ближнему.\\
~\\
Также в сказке присутствует мотив справедливости. Старик, получивший возможность исполнения желаний, использует ее не во благо, а для удовлетворения своих эгоистических потребностей. Однако в конце сказки он получает заслуженное наказание за свою жадность и неправильное использование полученных возможностей. Этот мотив напоминает нам о том, что каждый должен нести ответственность за свои поступки и что справедливость в конце концов всегда восторжествует.\\
~\\
Таким образом, в сказке "{}Петушок-рыбка"{} автор через различные темы и мотивы передает важные жизненные уроки о последствиях желаний, значимости благодарности и милосердия, а также необходимости справедливости. Эти темы и мотивы помогают сделать сказку увлекательной и запоминающейся для читателей.
\subsection{Роль тем и мотивов в сказке "{}{}Петушок рыбка"{}{}}
Темы и мотивы играют важную роль в сказке "{}{}Петушок рыбка"{}{}. Они помогают передать основные идеи произведения и создать атмосферу волшебства и фантазии.\\
~\\
Одной из основных тем в сказке является тема желаний и их последствий. Главный герой, петушок, получает от рыбки-волшебницы возможность исполнять любые свои желания. Однако, каждое исполненное желание имеет непредвиденные последствия, что учит героя и читателя осторожности и разумности в выборе желаний.\\
~\\
Другой важной темой является тема справедливости и наказания. В сказке присутствуют персонажи, которые заслуживают наказания за свои поступки, такие как старуха-жадина и ее дочь. Они жестоко обращаются с петушком и не ценят его помощь. В конце сказки они получают заслуженное наказание, а петушок возвращается к своей прежней жизни.\\
~\\
Тема добра и сострадания также присутствует в сказке. Петушок, несмотря на то что был обижен и использован, помогает старухе и ее дочери исполнить их желания. Он проявляет сострадание и доброту, что в конечном итоге приводит к его освобождению и наказанию злодеев.\\
~\\
Мотивы волшебства и магии также играют важную роль в сказке. Рыбка-волшебница и ее волшебные способности создают атмосферу волшебства и фантазии. Они помогают передать идею о том, что в мире сказок возможно все, что угодно, и что добро всегда побеждает зло.\\
~\\
Таким образом, темы и мотивы в сказке "{}{}Петушок рыбка"{}{} играют важную роль в передаче основных идей произведения и создании атмосферы волшебства и фантазии. Они помогают читателю понять уроки, которые преподносит сказка, и насладиться ее магическим миром.
\subsection{Мотивы природы и животного мира}
В сказке "{}{}Петушок-рыбка"{}{} присутствуют мотивы природы и животного мира, которые играют важную роль в развитии сюжета и характеризуют главных героев.\\
~\\
Одним из основных мотивов является мотив природы. В начале сказки описывается красивая и уютная деревня, окруженная зелеными полями и лесами. Это создает атмосферу мира, гармонии и благополучия. Природа также служит фоном для действия сказки, например, когда Петушок-рыбка превращается в рыбу и погружается в глубины озера.\\
~\\
Еще одним важным мотивом является мотив животного мира. В сказке встречаются различные животные, каждое из которых имеет свои особенности и характеристики. Например, Петушок-рыбка встречает на своем пути зайца, который помогает ему найти магическую рыбку. Животные в сказке выступают в роли помощников или препятствий для главного героя, а также символизируют различные качества и черты характера.\\
~\\
Мотивы природы и животного мира в сказке "{}{}Петушок-рыбка"{}{} помогают создать образы и атмосферу, а также передать главные идеи и ценности, такие как взаимопомощь, дружба и гармония с природой.
\subsection{Мотивы волшебства и волшебных предметов}
В сказке "{}{}Петушок-рыбка"{}{} присутствуют мотивы волшебства и волшебных предметов, которые играют важную роль в развитии сюжета и характеризуют основных героев.\\
~\\
Одним из центральных мотивов является мотив волшебной рыбки. Волшебная рыбка, обладающая способностью исполнять желания, является ключевым персонажем сказки. Она помогает главной героине, девочке Маше, исполнить ее желания и преодолеть трудности. Волшебная рыбка символизирует силу магии и волшебства, которые помогают преодолеть преграды и достичь счастья.\\
~\\
Еще одним мотивом волшебства является мотив волшебного платка. Платок, который Маша получает от волшебной рыбки, обладает свойством превращать все, что на него накрывается, в золото. Этот волшебный предмет помогает Маше и ее бабушке преодолеть бедность и стать богатыми. Мотив волшебного платка символизирует возможность преобразить свою жизнь и изменить свою судьбу с помощью волшебства.\\
~\\
Также в сказке присутствует мотив волшебного зеркала. Зеркало, которое Маша находит в доме волшебной рыбки, позволяет ей видеть все происходящее вокруг. Этот волшебный предмет помогает Маше избежать опасности и принять правильные решения. Мотив волшебного зеркала символизирует прозорливость и мудрость, которые помогают преодолеть трудности и найти истинное счастье.\\
~\\
Таким образом, мотивы волшебства и волшебных предметов в сказке "{}{}Петушок-рыбка"{}{} играют важную роль в развитии сюжета и характеризуют основных героев. Они символизируют силу магии, возможность изменить свою жизнь и преодолеть трудности, а также прозорливость и мудрость, которые помогают достичь истинного счастья.\\
~\\

\newpage
\section{Языковые особенности и стиль произведения}
В произведении "{}Петушок рыбка"{} автор использует разнообразные языковые особенности и стилистические приемы, которые способствуют созданию особой атмосферы и передаче эмоций.\\
~\\
Одной из особенностей языка произведения является использование детской речи. Автор воспроизводит речь главного героя, мальчика Васи, с его характерными ошибками и несовершенствами. Например, вместо слова "{}рыбка"{} Вася говорит "{}рыбька"{}, а вместо "{}петушок"{} - "{}петушьок"{}. Это придает тексту непосредственность и естественность, а также помогает читателю лучше погрузиться в мир ребенка.\\
~\\
Еще одной языковой особенностью произведения является использование повторов. Автор часто повторяет одни и те же слова и фразы, что создает ритмичность и музыкальность текста. Например, фраза "{}Петушок рыбка, петушок рыбка"{} повторяется несколько раз в разных частях произведения. Это придает тексту особую динамику и запоминающийся характер.\\
~\\
Также в произведении присутствуют языковые игры и шутки. Автор использует игру слов, созвучия и неожиданные сочетания, что делает текст более интересным и занимательным для читателя. Например, встречается фраза "{}рыбка-петушок"{}, которая является необычным сочетанием двух разных животных.\\
~\\
Стиль произведения можно охарактеризовать как легкий, игривый и фантастический. Автор использует яркие и красочные описания, создавая образы необычных существ и ситуаций. Он также часто использует метафоры и сравнения, чтобы передать настроение и эмоции героев. Например, описывая путешествие Васи и его друзей, автор пишет: "{}Они летели, как птицы, плыли, как рыбы, и были счастливы, как дети"{}.\\
~\\
В целом, языковые особенности и стиль произведения "{}Петушок рыбка"{} помогают создать особую атмосферу сказочности и волшебства, а также передать эмоции и настроение героев. Это делает произведение интересным и привлекательным для читателей всех возрастов.
\subsection{Особенности языка и стиля произведения "{}{}Петушок рыбка"{}{}}
Язык и стиль произведения "{}{}Петушок рыбка"{}{} отличаются от классической прозы и характеризуются рядом особенностей.\\
~\\
Во-первых, автор использует яркий и образный язык, который помогает создать живописные и неповторимые образы. Он описывает природу, животных и персонажей с помощью ярких метафор, эпитетов и сравнений. Например, в описании главного героя, Петушка, автор пишет: "{}{}Он был высокий, сильный, с гордой головой, с глазами, как две звезды, и с голосом, как звонкий колокольчик"{}{}.\\
~\\
Во-вторых, стиль произведения отличается легкостью и игривостью. Автор использует различные стилистические приемы, такие как повторы, аллитерации и игры слов, чтобы создать особую атмосферу сказки. Например, встречаясь с Петушком, рыбка говорит: "{}{}Здравствуй, Петушок! Я тебя ждала, ждала, ждала, и вот наконец-то ты пришел!"{}{}
В-третьих, язык произведения отличается простотой и доступностью. Автор использует простые и понятные слова, что делает произведение понятным для детей. Он также использует повторы и параллелизм, чтобы подчеркнуть основные идеи и сделать текст запоминающимся.\\
~\\
В-четвертых, язык и стиль произведения отличаются музыкальностью. Автор использует ритмичные и мелодичные фразы, что создает ощущение песни или пляски. Например, в описании танца Петушка и рыбки автор пишет: "{}{}Они танцевали, танцевали, танцевали, и все вокруг замирало, слушая их музыку"{}{}.\\
~\\
Таким образом, язык и стиль произведения "{}{}Петушок рыбка"{}{} отличаются яркостью, легкостью, простотой и музыкальностью. Они помогают создать особую атмосферу сказки и делают произведение запоминающимся и увлекательным для читателей.
\subsection{Лексические особенности}
Лексический состав произведения "{}{}Петушок рыбка"{}{} отличается от обычной речи и содержит ряд особенностей.\\
~\\
Во-первых, автор использует множество детских слов и выражений, что придает тексту особую игривость и легкость. Например, встречаются слова "{}{}петушок"{}{}, "{}{}рыбка"{}{}, "{}{}котенок"{}{}, "{}{}солнышко"{}{} и т.д.\\
~\\
Во-вторых, в произведении присутствуют множество описательных и метафорических выражений, которые создают яркую и красочную картину. Например, "{}{}лес зеленый и пушистый"{}{}, "{}{}вода прозрачная, как стекло"{}{}, "{}{}трава шелковая и мягкая"{}{}.\\
~\\
Также в тексте встречаются слова и выражения, характерные для сказочного жанра. Например, "{}{}волшебный ключ"{}{}, "{}{}волшебный сундук"{}{}, "{}{}волшебная палочка"{}{}. Эти слова и выражения создают атмосферу волшебства и фантастики.\\
~\\
Кроме того, автор использует повторы и звукоподражания, чтобы усилить эмоциональную окраску текста. Например, "{}{}тук-тук-тук"{}{}, "{}{}бах-бах-бах"{}{}.\\
~\\
Таким образом, лексические особенности произведения "{}{}Петушок рыбка"{}{} делают его ярким, красочным и запоминающимся для читателя.
\subsection{Грамматические особенности}
В произведении "{}{}Петушок рыбка"{}{} присутствуют несколько грамматических особенностей, которые придают тексту своеобразность и оригинальность.\\
~\\
Во-первых, автор активно использует повелительное наклонение, что создает ощущение прямого обращения к читателю. Например, в начале произведения звучит приказ: "{}{}Петушок, рыбка, вперед!"{}{}. Это придает тексту динамичность и эмоциональность.\\
~\\
Во-вторых, в тексте присутствуют множественные повторы и параллелизмы, которые создают ритмическую структуру произведения. Например, фраза "{}{}Петушок, рыбка, вперед!"{}{} повторяется несколько раз, что усиливает эффект повелительного наклонения и придает тексту особую мелодичность.\\
~\\
Также, автор использует метафоры и образные выражения, которые придают тексту поэтичность и выразительность. Например, фраза "{}{}Петушок, рыбка, вперед!"{}{} можно рассматривать как метафору, символизирующую стремление к новым горизонтам и нестандартным решениям.\\
~\\
В произведении также присутствуют синтаксические конструкции, которые создают эффект незавершенности и динамичности. Например, фраза "{}{}Петушок, рыбка, вперед, вперед, вперед!"{}{} состоит из нескольких повторяющихся частей, что создает ощущение бесконечного движения и активности.\\
~\\
Таким образом, грамматические особенности произведения "{}{}Петушок рыбка"{}{} играют важную роль в создании его стиля и атмосферы. Они придают тексту эмоциональность, ритмичность и поэтичность, делая его запоминающимся и оригинальным.\\
~\\

\newpage

\section{Интерпретация сказки "Петушок-рыбка"}
\begin{center}
    \textbf{
        Спасибо, что воспользовались Scribot! Надеюсь, Вам понравилась курсовая работа!\\
        Для получения полной версии отправьте 99 рублей по ссылке:\\
        https://pay.cloudtips.ru/p/7a822105\\
        Или по QR-коду:\\
    }
\end{center}
\begin{figure}[h]
    \center{\includegraphics[width=\linewidth/2]{qrCode}}
    \caption{QR-код на оплату работы.}
    \label{ris:image}
\end{figure}
\newpage
\begin{center}
    \textbf{
        Спасибо, что воспользовались Scribot! Надеюсь, Вам понравилась курсовая работа!\\
        Для получения полной версии отправьте 99 рублей по ссылке:\\
        https://pay.cloudtips.ru/p/7a822105\\
        Или по QR-коду:\\
    }
\end{center}
\begin{figure}[h]
    \center{\includegraphics[width=\linewidth/2]{qrCode}}
    \caption{QR-код на оплату работы.}
    \label{ris:image}
\end{figure}
\newpage

\section{Социально-философский аспект сказки}
\begin{center}
    \textbf{
        Спасибо, что воспользовались Scribot! Надеюсь, Вам понравилась курсовая работа!\\
        Для получения полной версии отправьте 99 рублей по ссылке:\\
        https://pay.cloudtips.ru/p/7a822105\\
        Или по QR-коду:\\
    }
\end{center}
\begin{figure}[h]
    \center{\includegraphics[width=\linewidth/2]{qrCode}}
    \caption{QR-код на оплату работы.}
    \label{ris:image}
\end{figure}
\newpage
\begin{center}
    \textbf{
        Спасибо, что воспользовались Scribot! Надеюсь, Вам понравилась курсовая работа!\\
        Для получения полной версии отправьте 99 рублей по ссылке:\\
        https://pay.cloudtips.ru/p/7a822105\\
        Или по QR-коду:\\
    }
\end{center}
\begin{figure}[h]
    \center{\includegraphics[width=\linewidth/2]{qrCode}}
    \caption{QR-код на оплату работы.}
    \label{ris:image}
\end{figure}
\newpage

\section{Моральные уроки и ценности, которые передает сказка}
\begin{center}
    \textbf{
        Спасибо, что воспользовались Scribot! Надеюсь, Вам понравилась курсовая работа!\\
        Для получения полной версии отправьте 99 рублей по ссылке:\\
        https://pay.cloudtips.ru/p/7a822105\\
        Или по QR-коду:\\
    }
\end{center}
\begin{figure}[h]
    \center{\includegraphics[width=\linewidth/2]{qrCode}}
    \caption{QR-код на оплату работы.}
    \label{ris:image}
\end{figure}
\newpage
\begin{center}
    \textbf{
        Спасибо, что воспользовались Scribot! Надеюсь, Вам понравилась курсовая работа!\\
        Для получения полной версии отправьте 99 рублей по ссылке:\\
        https://pay.cloudtips.ru/p/7a822105\\
        Или по QR-коду:\\
    }
\end{center}
\begin{figure}[h]
    \center{\includegraphics[width=\linewidth/2]{qrCode}}
    \caption{QR-код на оплату работы.}
    \label{ris:image}
\end{figure}
\newpage

\section{Символика и образы в сказке}
\begin{center}
    \textbf{
        Спасибо, что воспользовались Scribot! Надеюсь, Вам понравилась курсовая работа!\\
        Для получения полной версии отправьте 99 рублей по ссылке:\\
        https://pay.cloudtips.ru/p/7a822105\\
        Или по QR-коду:\\
    }
\end{center}
\begin{figure}[h]
    \center{\includegraphics[width=\linewidth/2]{qrCode}}
    \caption{QR-код на оплату работы.}
    \label{ris:image}
\end{figure}
\newpage
\begin{center}
    \textbf{
        Спасибо, что воспользовались Scribot! Надеюсь, Вам понравилась курсовая работа!\\
        Для получения полной версии отправьте 99 рублей по ссылке:\\
        https://pay.cloudtips.ru/p/7a822105\\
        Или по QR-коду:\\
    }
\end{center}
\begin{figure}[h]
    \center{\includegraphics[width=\linewidth/2]{qrCode}}
    \caption{QR-код на оплату работы.}
    \label{ris:image}
\end{figure}
\newpage

\section{Влияние сказки на развитие детского сознания}
\begin{center}
    \textbf{
        Спасибо, что воспользовались Scribot! Надеюсь, Вам понравилась курсовая работа!\\
        Для получения полной версии отправьте 99 рублей по ссылке:\\
        https://pay.cloudtips.ru/p/7a822105\\
        Или по QR-коду:\\
    }
\end{center}
\begin{figure}[h]
    \center{\includegraphics[width=\linewidth/2]{qrCode}}
    \caption{QR-код на оплату работы.}
    \label{ris:image}
\end{figure}
\newpage
\begin{center}
    \textbf{
        Спасибо, что воспользовались Scribot! Надеюсь, Вам понравилась курсовая работа!\\
        Для получения полной версии отправьте 99 рублей по ссылке:\\
        https://pay.cloudtips.ru/p/7a822105\\
        Или по QR-коду:\\
    }
\end{center}
\begin{figure}[h]
    \center{\includegraphics[width=\linewidth/2]{qrCode}}
    \caption{QR-код на оплату работы.}
    \label{ris:image}
\end{figure}
\newpage

\section{Заключение}
\begin{center}
    \textbf{
        Спасибо, что воспользовались Scribot! Надеюсь, Вам понравилась курсовая работа!\\
        Для получения полной версии отправьте 99 рублей по ссылке:\\
        https://pay.cloudtips.ru/p/7a822105\\
        Или по QR-коду:\\
    }
\end{center}
\begin{figure}[h]
    \center{\includegraphics[width=\linewidth/2]{qrCode}}
    \caption{QR-код на оплату работы.}
    \label{ris:image}
\end{figure}
\newpage
\begin{center}
    \textbf{
        Спасибо, что воспользовались Scribot! Надеюсь, Вам понравилась курсовая работа!\\
        Для получения полной версии отправьте 99 рублей по ссылке:\\
        https://pay.cloudtips.ru/p/7a822105\\
        Или по QR-коду:\\
    }
\end{center}
\begin{figure}[h]
    \center{\includegraphics[width=\linewidth/2]{qrCode}}
    \caption{QR-код на оплату работы.}
    \label{ris:image}
\end{figure}
\newpage

\section{Выводы о значимости и актуальности сказки "Петушок-рыбка"}
\begin{center}
    \textbf{
        Спасибо, что воспользовались Scribot! Надеюсь, Вам понравилась курсовая работа!\\
        Для получения полной версии отправьте 99 рублей по ссылке:\\
        https://pay.cloudtips.ru/p/7a822105\\
        Или по QR-коду:\\
    }
\end{center}
\begin{figure}[h]
    \center{\includegraphics[width=\linewidth/2]{qrCode}}
    \caption{QR-код на оплату работы.}
    \label{ris:image}
\end{figure}
\newpage
\begin{center}
    \textbf{
        Спасибо, что воспользовались Scribot! Надеюсь, Вам понравилась курсовая работа!\\
        Для получения полной версии отправьте 99 рублей по ссылке:\\
        https://pay.cloudtips.ru/p/7a822105\\
        Или по QR-коду:\\
    }
\end{center}
\begin{figure}[h]
    \center{\includegraphics[width=\linewidth/2]{qrCode}}
    \caption{QR-код на оплату работы.}
    \label{ris:image}
\end{figure}
\newpage

\section{Роль сказки в формировании личности ребенка}
\begin{center}
    \textbf{
        Спасибо, что воспользовались Scribot! Надеюсь, Вам понравилась курсовая работа!\\
        Для получения полной версии отправьте 99 рублей по ссылке:\\
        https://pay.cloudtips.ru/p/7a822105\\
        Или по QR-коду:\\
    }
\end{center}
\begin{figure}[h]
    \center{\includegraphics[width=\linewidth/2]{qrCode}}
    \caption{QR-код на оплату работы.}
    \label{ris:image}
\end{figure}
\newpage
\begin{center}
    \textbf{
        Спасибо, что воспользовались Scribot! Надеюсь, Вам понравилась курсовая работа!\\
        Для получения полной версии отправьте 99 рублей по ссылке:\\
        https://pay.cloudtips.ru/p/7a822105\\
        Или по QR-коду:\\
    }
\end{center}
\begin{figure}[h]
    \center{\includegraphics[width=\linewidth/2]{qrCode}}
    \caption{QR-код на оплату работы.}
    \label{ris:image}
\end{figure}
\newpage

\section{Значение сказки в современном обществе}
\begin{center}
    \textbf{
        Спасибо, что воспользовались Scribot! Надеюсь, Вам понравилась курсовая работа!\\
        Для получения полной версии отправьте 99 рублей по ссылке:\\
        https://pay.cloudtips.ru/p/7a822105\\
        Или по QR-коду:\\
    }
\end{center}
\begin{figure}[h]
    \center{\includegraphics[width=\linewidth/2]{qrCode}}
    \caption{QR-код на оплату работы.}
    \label{ris:image}
\end{figure}
\newpage
\begin{center}
    \textbf{
        Спасибо, что воспользовались Scribot! Надеюсь, Вам понравилась курсовая работа!\\
        Для получения полной версии отправьте 99 рублей по ссылке:\\
        https://pay.cloudtips.ru/p/7a822105\\
        Или по QR-коду:\\
    }
\end{center}
\begin{figure}[h]
    \center{\includegraphics[width=\linewidth/2]{qrCode}}
    \caption{QR-код на оплату работы.}
    \label{ris:image}
\end{figure}
\newpage

\section{Перспективы исследования сказки "Петушок-рыбка"}
\begin{center}
    \textbf{
        Спасибо, что воспользовались Scribot! Надеюсь, Вам понравилась курсовая работа!\\
        Для получения полной версии отправьте 99 рублей по ссылке:\\
        https://pay.cloudtips.ru/p/7a822105\\
        Или по QR-коду:\\
    }
\end{center}
\begin{figure}[h]
    \center{\includegraphics[width=\linewidth/2]{qrCode}}
    \caption{QR-код на оплату работы.}
    \label{ris:image}
\end{figure}
\newpage
\begin{center}
    \textbf{
        Спасибо, что воспользовались Scribot! Надеюсь, Вам понравилась курсовая работа!\\
        Для получения полной версии отправьте 99 рублей по ссылке:\\
        https://pay.cloudtips.ru/p/7a822105\\
        Или по QR-коду:\\
    }
\end{center}
\begin{figure}[h]
    \center{\includegraphics[width=\linewidth/2]{qrCode}}
    \caption{QR-код на оплату работы.}
    \label{ris:image}
\end{figure}
\newpage

\section{Список использованных источников}
\begin{center}
    \textbf{
        Спасибо, что воспользовались Scribot! Надеюсь, Вам понравилась курсовая работа!\\
        Для получения полной версии отправьте 99 рублей по ссылке:\\
        https://pay.cloudtips.ru/p/7a822105\\
        Или по QR-коду:\\
    }
\end{center}
\begin{figure}[h]
    \center{\includegraphics[width=\linewidth/2]{qrCode}}
    \caption{QR-код на оплату работы.}
    \label{ris:image}
\end{figure}
\newpage
\begin{center}
    \textbf{
        Спасибо, что воспользовались Scribot! Надеюсь, Вам понравилась курсовая работа!\\
        Для получения полной версии отправьте 99 рублей по ссылке:\\
        https://pay.cloudtips.ru/p/7a822105\\
        Или по QR-коду:\\
    }
\end{center}
\begin{figure}[h]
    \center{\includegraphics[width=\linewidth/2]{qrCode}}
    \caption{QR-код на оплату работы.}
    \label{ris:image}
\end{figure}
\end{document}
