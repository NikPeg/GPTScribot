\documentclass{article}
\usepackage{cmap}
\usepackage[T1,T2A]{fontenc}
\usepackage[utf8]{inputenc}
\usepackage[russian]{babel}
\usepackage[left=2cm,right=2cm,top=2cm,bottom=2cm,bindingoffset=0cm]{geometry}
\usepackage{tikz}
\usepackage{tabto}
\usepackage{epstopdf}
\usepackage{setspace,amsmath}
\usepackage{tabularx}
\usepackage{multirow}
\usepackage{makecell}
\usepackage{listings}
\usepackage{titlesec}
\usepackage{lipsum}
\usepackage[usestackEOL]{stackengine}
\usepackage{kantlipsum}
\usepackage{caption}
\usepackage{float}
\usepackage{zref-totpages}
\usepackage{fancyhdr}
\usepackage{graphicx}
\pagestyle{fancy}
\fancyhf{}
\fancyhead[C]{\thepage}
\renewcommand{\headrulewidth}{0pt}
\captionsetup[table]{justification=centering}
\usetikzlibrary{positioning}
\graphicspath{ {./pictures/} }
\DeclareGraphicsExtensions{.pdf,.png,.jpg}
\newcommand\zz[1]{\par{\normalsize\strut #1} \hfill\ignorespaces}
\addto\captionsrussian{\def\refname{}}
\newcommand{\subtitle}[1]{%
  \posttitle{%
    \par\end{center}
    \begin{center}\Large#1\end{center}
  }%
}
\newcommand{\subsubtitle}[1]{%
  \preauthor{%
    \begin{center}
    \large #1 \vskip0.5em
    \begin{tabular}[t]{c}
  }%
}
\begin{document}
\raggedright
\fontsize{14}{16}\selectfont
\thispagestyle{empty}
\clearpage
\pagenumbering{arabic}
\bigskip
\begin{center}
\topskip=0pt
\vspace*{\fill}
\textbf{
ЧИСЛА~\\
~\\
~\\
Исследовательская работа\\
}
~\\
Листов \ztotpages\\
\vspace*{\fill}
\end{center}
\begin{center}
\vspace*{\fill}{
  Город \the\year{}}
\end{center}
\newpage
\tableofcontents
\newpage

\newpage
\section{Введение}
Числа - одно из самых фундаментальных понятий в математике. Они используются для измерения, подсчета, классификации и многих других целей. Числа присутствуют во всех областях науки и техники, и без них невозможно представить себе современный мир.\\
~\\
Цель данной исследовательской работы - рассмотреть различные аспекты чисел и их свойств. Мы изучим различные типы чисел, операции над ними, их свойства и взаимосвязи. Также мы рассмотрим историю развития чисел и их роль в современной математике.\\
~\\
Изучение чисел имеет огромное значение не только для математиков, но и для всех, кто хочет лучше понять мир вокруг себя. Надеемся, что данная работа поможет читателям расширить свои знания и увлечение в области чисел и математики.
\subsection{Общая характеристика чисел}
Числа - это абстрактные математические объекты, которые используются для измерения количества, размера, расстояния и других характеристик объектов и явлений. Числа могут быть натуральными, целыми, рациональными, иррациональными, вещественными, комплексными и другими. Они играют важную роль во многих областях науки, техники, экономики и других сферах человеческой деятельности. В данной работе мы рассмотрим различные свойства и характеристики чисел, их взаимосвязи и применение в различных задачах.
\subsection{История изучения чисел}
Изучение чисел имеет древнюю историю, начиная с древних цивилизаций, таких как древний Египет и Вавилон. В древности числа использовались для счета, измерения времени и пространства, а также для проведения различных расчетов.\\
~\\
Одним из первых математиков, который внес значительный вклад в изучение чисел, был Пифагор. Он разработал теорему Пифагора и изучал свойства простых чисел. Позднее арабские математики, такие как Аль-Хорезми и Аль-Кваризми, внесли свой вклад в развитие алгебры и теории чисел.\\
~\\
С развитием науки и технологий изучение чисел стало более глубоким и разнообразным. Современные математики исследуют различные типы чисел, такие как комплексные числа, рациональные числа и иррациональные числа, а также разрабатывают новые методы и теории для работы с числами.
\subsection{Значение чисел в различных областях науки}
Числа играют важную роль во многих областях науки, таких как математика, физика, химия, биология и т.д. В математике числа используются для описания количества, измерения величин, решения уравнений и многих других задач. В физике числа используются для описания физических законов, вычисления физических величин и прогнозирования результатов экспериментов. В химии числа используются для описания химических реакций, вычисления молекулярных масс и пропорций веществ. В биологии числа используются для описания генетических кодов, вычисления вероятностей наследования признаков и анализа биологических данных. Таким образом, числа играют важную роль в понимании и описании мира в различных областях науки.
\subsection{Цели и задачи исследования чисел}
Целью данного исследования является изучение основных свойств чисел и их взаимосвязей, а также исследование различных математических операций, связанных с числами. Для достижения данной цели были поставлены следующие задачи:
\begin{itemize}
\item Изучить основные свойства натуральных, целых, рациональных и вещественных чисел;
\item Исследовать операции сложения, вычитания, умножения и деления чисел;
\item Изучить различные методы работы с числами, такие как разложение на множители, нахождение НОД и НОК;
\item Провести анализ простых и составных чисел, исследовать их свойства и взаимосвязи;
\item Изучить основные свойства простых чисел и провести анализ их распределения в натуральном ряду.
\end{itemize}

\newpage
\section{История чисел}
Числа играют важную роль в жизни человека с древних времен. Они используются для измерения, подсчета, описания и многих других целей. История чисел насчитывает тысячелетия и включает в себя множество интересных фактов и событий.\\
~\\
Одним из самых древних систем числения является римская система, которая использовалась в Древнем Риме. В этой системе числа обозначались римскими цифрами, такими как I, V, X, L, C, D, M. С помощью этих символов можно было записать любое число, но система была неудобной для математических операций.\\
~\\
В средние века в Европе начали использовать арабские цифры, которые мы используем и сегодня. Эта система числения была разработана в Индии и быстро распространилась по всему миру благодаря арабским ученым. Арабские цифры состоят из десяти символов: 0, 1, 2, 3, 4, 5, 6, 7, 8, 9. Они удобны для математических операций и записи больших чисел.\\
~\\
С развитием науки и технологий появились новые системы числения, такие как двоичная, восьмеричная и шестнадцатеричная. Эти системы используются в компьютерах и других устройствах для хранения и обработки информации.\\
~\\
История чисел богата различными открытиями и достижениями математиков и ученых. Они изучали свойства чисел, разрабатывали новые методы вычислений и решения задач. Сегодня числа играют важную роль во всех областях науки и техники и продолжают быть объектом изучения исследователей.
\subsection{История развития чисел}
История чисел насчитывает тысячелетия развития и эволюции. Одним из первых видов чисел, которые использовали древние цивилизации, были натуральные числа. Они позволяли считать предметы и измерять количество.\\
~\\
С развитием математики были введены целые числа, которые включали в себя как натуральные числа, так и их отрицательные значения. Это позволило решать более сложные задачи и операции.\\
~\\
Дальнейшим шагом в развитии чисел стали рациональные числа, которые включают в себя дроби. Они позволяют представлять числа в виде отношения двух целых чисел и решать еще более сложные задачи.\\
~\\
С появлением иррациональных чисел, таких как корень из двух, математики столкнулись с новыми вызовами и задачами. Эти числа не могут быть представлены в виде дроби и имеют бесконечную десятичную дробь без периода.\\
~\\
Современная математика включает в себя комплексные числа, которые включают в себя вещественную и мнимую части. Они играют важную роль в различных областях науки и техники.\\
~\\
Таким образом, история развития чисел отражает постоянное стремление человечества к расширению знаний и возможностей в области математики.
\subsection{Роль чисел в различных цивилизациях}
Числа играли важную роль в различных цивилизациях на протяжении истории человечества. В древних цивилизациях, таких как древний Египет, числа использовались для измерения времени, расчетов и построения архитектурных сооружений. В древнем Китае числа имели символическое значение и использовались в философии и астрологии.\\
~\\
В средневековой Европе числа были связаны с религией и мистицизмом, а также использовались для торговли и финансовых расчетов. В Индии числа играли важную роль в математике и астрономии, а также были связаны с религиозными практиками.\\
~\\
В современном мире числа используются повсеместно в науке, технике, экономике и других областях. Они являются основой математики и статистики, а также используются для кодирования информации и обработки данных. В целом, числа играют важную роль в различных аспектах жизни человека и продолжают оставаться неотъемлемой частью нашей культуры и общества.
\subsection{Математические открытия и открытия в области чисел}
В истории чисел было совершено множество математических открытий, которые изменили наше представление о числах и их свойствах. Одним из самых значимых открытий было введение нуля как числа и его использование в позиционной системе счисления. Это открытие позволило упростить арифметические операции и сделало возможным работу с большими числами.\\
~\\
Другим важным математическим открытием было открытие бесконечности чисел. Древние математики столкнулись с понятием бесконечности при изучении простых чисел и различных математических последовательностей. Это открытие привело к развитию теории множеств и теории чисел.\\
~\\
Важным открытием в области чисел было также открытие иррациональных чисел. Древние греки обнаружили, что не все числа можно представить в виде дробей, и ввели понятие иррациональных чисел. Это открытие имело огромное значение для развития математики и привело к созданию теории чисел.\\
~\\
Таким образом, математические открытия и открытия в области чисел играли ключевую роль в развитии математики и помогли нам лучше понять природу чисел и их свойства.
\subsection{Влияние чисел на развитие науки и технологий}
Числа играют ключевую роль в развитии науки и технологий. Они используются для измерения, описания и анализа различных явлений в природе и обществе. Благодаря числам ученые могут проводить точные измерения, строить математические модели и прогнозировать результаты исследований.\\
~\\
Важное значение чисел в науке проявляется во всех ее областях, от физики и химии до биологии и экономики. Например, в физике числа используются для описания физических законов и явлений, в химии - для расчета химических реакций и свойств веществ, а в биологии - для анализа генетических данных и популяционных процессов.\\
~\\
Технологии также тесно связаны с числами. В разработке новых технологий используются математические модели, алгоритмы и вычисления, которые невозможны без чисел. Благодаря числам были созданы компьютеры, мобильные устройства, интернет и множество других технологий, которые сегодня широко используются во всех сферах жизни.\\
~\\
Таким образом, числа играют важную роль в развитии науки и технологий, обеспечивая точность, надежность и эффективность исследований и разработок.\\
~\\

\newpage

\section{Применение чисел в различных областях}
\begin{center}
    \textbf{
        Спасибо, что воспользовались Scribot! Надеюсь, Вам понравилась курсовая работа!\\
        Для получения полной версии отправьте 99 рублей по ссылке:\\
        link
        Или по QR-коду:\\
    }
\end{center}
\begin{figure}[h]
    \center{\includegraphics[width=\linewidth/2]{qrCode}}
    \caption{QR-код на оплату работы.}
    \label{ris:image}
\end{figure}
\newpage
\begin{center}
    \textbf{
        Спасибо, что воспользовались Scribot! Надеюсь, Вам понравилась курсовая работа!\\
        Для получения полной версии отправьте 99 рублей по ссылке:\\
        link
        Или по QR-коду:\\
    }
\end{center}
\begin{figure}[h]
    \center{\includegraphics[width=\linewidth/2]{qrCode}}
    \caption{QR-код на оплату работы.}
    \label{ris:image}
\end{figure}
\newpage

\section{Список использованных источников}
\begin{center}
    \textbf{
        Спасибо, что воспользовались Scribot! Надеюсь, Вам понравилась курсовая работа!\\
        Для получения полной версии отправьте 99 рублей по ссылке:\\
        link
        Или по QR-коду:\\
    }
\end{center}
\begin{figure}[h]
    \center{\includegraphics[width=\linewidth/2]{qrCode}}
    \caption{QR-код на оплату работы.}
    \label{ris:image}
\end{figure}
\newpage
\begin{center}
    \textbf{
        Спасибо, что воспользовались Scribot! Надеюсь, Вам понравилась курсовая работа!\\
        Для получения полной версии отправьте 99 рублей по ссылке:\\
        link
        Или по QR-коду:\\
    }
\end{center}
\begin{figure}[h]
    \center{\includegraphics[width=\linewidth/2]{qrCode}}
    \caption{QR-код на оплату работы.}
    \label{ris:image}
\end{figure}
\end{document}