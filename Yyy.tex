\documentclass{article}
\usepackage{cmap}
\usepackage[T1,T2A]{fontenc}
\usepackage[utf8]{inputenc}
\usepackage[russian]{babel}
\usepackage[left=2cm,right=2cm,top=2cm,bottom=2cm,bindingoffset=0cm]{geometry}
\usepackage{tikz}
\usepackage{tabto}
\usepackage{epstopdf}
\usepackage{setspace,amsmath}
\usepackage{tabularx}
\usepackage{multirow}
\usepackage{makecell}
\usepackage{listings}
\usepackage{titlesec}
\usepackage{lipsum}
\usepackage[usestackEOL]{stackengine}
\usepackage{kantlipsum}
\usepackage{caption}
\usepackage{float}
\usepackage{zref-totpages}
\usepackage{fancyhdr}
\usepackage{graphicx}
\pagestyle{fancy}
\fancyhf{}
\fancyhead[C]{\thepage\\ RU.17701729.10.03-01 01-1}
\renewcommand{\headrulewidth}{0pt}
\captionsetup[table]{justification=centering}
\usetikzlibrary{positioning}
\graphicspath{ {./pictures/} }
\DeclareGraphicsExtensions{.pdf,.png,.jpg}
\newcommand\zz[1]{\par{\normalsize\strut #1} \hfill\ignorespaces}
\addto\captionsrussian{\def\refname{}}
\newcommand{\subtitle}[1]{%
  \posttitle{%
    \par\end{center}
    \begin{center}\Large#1\end{center}
  }%
}
\newcommand{\subsubtitle}[1]{%
  \preauthor{%
    \begin{center}
    \large #1 \vskip0.5em
    \begin{tabular}[t]{c}
  }%
}
\begin{document}
\fontsize{14}{16}\selectfont
\thispagestyle{empty}
\clearpage
\pagenumbering{arabic}
\bigskip
\begin{center}
\topskip=0pt
\vspace*{\fill}
\textbf{ЫЫЫ~\\
~\\
~\\
CourseWorkType.COURSE\_WORK\\
~\\
RU.17701729.10.03-01 01-1-ЛУ}\\
~\\
Листов \ztotpages\\
\vspace*{\fill}
\end{center}
\begin{center}
\vspace*{\fill}{
  Город \the\year{}}
\end{center}
\newpage
\tableofcontents
\newpage
\newpage
\section{Введение}
В настоящее время информационные технологии играют важную роль в различных сферах деятельности человека. Они позволяют автоматизировать процессы, упростить работу и повысить эффективность работы. Одной из таких сфер является образование.\\
~\\
С развитием интернета и доступности компьютеров, образовательные учреждения все чаще используют информационные технологии в своей работе. Однако, несмотря на это, в некоторых случаях преподаватели и студенты сталкиваются с определенными трудностями при использовании информационных технологий в образовательном процессе.\\
~\\
Целью данной курсовой работы является исследование проблем, с которыми сталкиваются преподаватели и студенты при использовании информационных технологий в образовательном процессе, а также разработка рекомендаций по их решению.\\
~\\
Для достижения поставленной цели были сформулированы следующие задачи:
\begin{itemize}
\item проанализировать существующие проблемы при использовании информационных технологий в образовательном процессе;
\item выявить основные причины возникновения этих проблем;
\item разработать рекомендации по решению выявленных проблем.
\end{itemize}
В работе был использован аналитический метод исследования, а также метод сравнения и обобщения научной литературы и практического опыта.\\
~\\
Первый раздел работы посвящен обзору существующих проблем при использовании информационных технологий в образовательном процессе. Второй раздел содержит анализ причин возникновения этих проблем. В третьем разделе представлены разработанные рекомендации по решению проблем.\\
~\\
В заключении работы приведены основные выводы, сделанные в ходе исследования, а также возможные направления дальнейших исследований в данной области.\\
~\\
Таким образом, данная курсовая работа имеет практическую значимость и может быть использована преподавателями и студентами для повышения эффективности использования информационных технологий в образовательном процессе.
\subsection{Общая характеристика темы работы}
Тема данной курсовой работы "{}{}Ыыы"{}{} посвящена исследованию и анализу [описание темы работы]. В работе рассматриваются основные аспекты [описание основных аспектов], а также проводится [описание проводимых исследований или анализов].\\
~\\
Целью работы является [описание цели работы]. Для достижения данной цели были поставлены следующие задачи:
\begin{enumerate}
\item [описание задачи 1];
\item [описание задачи 2];
\item [описание задачи 3].
\end{enumerate}
В работе использованы следующие методы исследования [описание методов исследования]. Для анализа [описание анализируемых данных] были применены [описание используемых методов анализа].\\
~\\
Результаты работы могут быть использованы для [описание возможных практических применений результатов работы].\\
~\\
Структура работы следующая. Во введении обосновывается актуальность темы, формулируются цель и задачи работы. В первом разделе проводится обзор литературы по теме исследования. Во втором разделе представлены материалы и методы исследования. В третьем разделе приведены результаты исследования. В заключении подводятся итоги работы, делаются выводы и предлагаются рекомендации для дальнейших исследований.\\
~\\
В работе использованы следующие источники информации: [перечисление использованных источников].
\subsection{Актуальность выбранной темы}
Выбранная тема исследования является актуальной в современном обществе. В настоящее время информационные технологии проникают во все сферы жизни людей, включая образование, бизнес, медицину, государственное управление и другие. Однако, вместе с преимуществами, которые они предоставляют, существуют и риски, связанные с использованием информационных технологий.\\
~\\
Одной из таких угроз является кибербуллинг  форма электронного насилия, при которой люди используют информационные и коммуникационные технологии для унижения, запугивания и преследования других людей. Кибербуллинг может иметь серьезные последствия для жертвы, включая психологические проблемы, снижение самооценки, депрессию и даже самоубийство.\\
~\\
В связи с этим, важно исследовать проблему кибербуллинга и разработать эффективные методы предотвращения и борьбы с ним. Это позволит создать безопасную и здоровую среду в онлайн-пространстве, где люди могут свободно общаться и развиваться, не подвергаясь угрозам и насилию.\\
~\\
Целью данной курсовой работы является изучение проблемы кибербуллинга, анализ существующих методов предотвращения и борьбы с ним, а также разработка рекомендаций по созданию безопасной онлайн-среды. Результаты исследования могут быть полезными для педагогов, родителей, специалистов в области информационных технологий и всех, кто заинтересован в решении проблемы кибербуллинга.
\subsection{Цель и задачи работы}
Целью данной курсовой работы является исследование и анализ основных принципов и методов, используемых в области Ыыы. Для достижения данной цели были поставлены следующие задачи:
\begin{enumerate}
\item Изучить основные понятия и теоретические аспекты, связанные с Ыыы.
\item Проанализировать существующие подходы и методы, применяемые в области Ыыы.
\item Разработать собственную методику/алгоритм/модель для решения задачи Ыыы.
\item Провести экспериментальное исследование разработанной методики/алгоритма/модели на реальных данных.
\item Сравнить полученные результаты с результатами, полученными с использованием других подходов и методов.
\item Сделать выводы о применимости и эффективности разработанной методики/алгоритма/модели в области Ыыы.
\end{enumerate}
Таким образом, выполнение поставленных задач позволит достичь цели работы и получить новые знания и практические навыки в области Ыыы.\\
~\\

\newpage
\section{Обзор литературы}
В начале раздела обзора литературы представлены основные понятия и теоретические основы, связанные с темой работы. Затем приводятся результаты предыдущих исследований, которые были проведены в данной области. Важно указать на проблемы и недостатки этих исследований, которые могут быть релевантными для данной работы.\\
~\\
Далее следует анализ и сравнение различных подходов и методов, использованных в предыдущих исследованиях. Важно обратить внимание на их преимущества и недостатки, а также на их применимость к данной работе.\\
~\\
После этого представляются результаты эмпирических исследований, проведенных в данной области. Важно описать методы, использованные в этих исследованиях, а также их основные результаты. При необходимости можно провести сравнение собственных результатов с результатами предыдущих исследований.\\
~\\
В заключении раздела обзора литературы можно указать на пробелы в существующих исследованиях и обосновать актуальность данной работы. Также можно предложить возможные направления для дальнейших исследований в данной области.\\
~\\
В целом, раздел "{}Обзор литературы"{} представляет собой критическую оценку предыдущих исследований, а также обоснование актуальности и значимости данной работы.
\subsection{Введение в обзор литературы}
В данном разделе представлен обзор литературы, посвященный теме исследования. В работе рассматриваются различные источники, включая научные статьи, книги, диссертации и другие публикации, которые имеют отношение к данной теме.\\
~\\
Цель обзора литературы заключается в том, чтобы представить читателю существующие исследования и теоретические подходы, связанные с темой работы. Обзор литературы помогает определить актуальность исследования, выявить проблемы, которые уже были рассмотрены другими авторами, и определить пробелы в существующих знаниях.\\
~\\
В данном разделе представлены основные теоретические концепции и модели, которые используются в работе. Также рассматриваются результаты предыдущих исследований, связанные с темой работы, и их основные выводы. Обзор литературы также включает критическую оценку предыдущих исследований и указание на их ограничения.\\
~\\
Обзор литературы является важной частью курсовой работы, так как он позволяет установить связь между текущим исследованием и предыдущими работами в данной области. Это помогает читателю лучше понять контекст исследования и его значимость.\\
~\\
В заключение обзора литературы указывается, какие аспекты темы работы будут рассмотрены в дальнейшем и какие вопросы будут решены в рамках данного исследования.
\subsection{Исторический обзор исследований}
Исследования в области \textit{[тема курсовой работы]} имеют долгую историю и привлекают внимание ученых уже на протяжении многих лет. В данном разделе представлен обзор исторического развития исследований в данной области.\\
~\\
Первые исследования в области \textit{[тема курсовой работы]} были проведены в \textit{[годы]}, когда \textit{[описание первых исследований]}. Эти исследования положили основу для дальнейших исследований и стали отправной точкой для развития данной области.\\
~\\
В последующие годы, исследователи продолжали изучать \textit{[тема курсовой работы]} и расширять границы знаний в этой области. Важным вехом в развитии исследований стало открытие \textit{[открытие]}, которое позволило \textit{[описание вклада открытия в развитие области]}.\\
~\\
В \textit{[годы]}, с развитием компьютерных технологий, исследования в области \textit{[тема курсовой работы]} получили новый импульс. Благодаря возможностям вычислительной техники, исследователи смогли проводить более сложные и точные эксперименты, а также анализировать большие объемы данных.\\
~\\
В настоящее время, исследования в области \textit{[тема курсовой работы]} активно продолжаются. Ученые постоянно работают над разработкой новых методов и подходов, а также применением современных технологий для решения актуальных проблем в данной области.\\
~\\
Таким образом, исторический обзор исследований в области \textit{[тема курсовой работы]} позволяет увидеть эволюцию и развитие данной области на протяжении времени. Понимание предшествующих исследований является важным шагом для понимания текущего состояния исследований и определения направлений для будущих исследований.
\subsection{Теоретические основы и концепции}
В данном разделе представлен обзор литературы, посвященной теоретическим основам и концепциям, связанным с темой исследования. Рассмотрены основные теоретические аспекты, которые являются основой для дальнейшего анализа и исследования.\\
~\\
В работе были рассмотрены следующие теоретические основы и концепции:
\begin{enumerate}
\item Теория X и Y МакГрегора. Данная теория представляет два различных подхода к управлению персоналом. Теория X предполагает, что люди по своей природе ленивы и не любят работать, поэтому им необходимо навязывать контроль и наказания. В то время как теория Y предполагает, что люди могут быть самоорганизованными и мотивированными, и им необходимо предоставить свободу и возможность для саморазвития.
\item Теория геройского лидерства. Данная теория утверждает, что лидер должен быть героем, который способен преодолевать трудности и препятствия, чтобы достичь поставленных целей. Геройский лидер обладает особыми качествами, такими как харизма, вдохновение и способность вести людей за собой.
\item Теория ситуационного лидерства. Данная теория предполагает, что эффективный лидер должен уметь адаптироваться к различным ситуациям и применять различные стили лидерства в зависимости от обстоятельств. В рамках этой теории выделяются четыре стиля лидерства: директивный, коучинговый, поддерживающий и делегирующий.
\item Теория трансформационного лидерства. Данная теория утверждает, что эффективный лидер должен быть способен вдохновлять и мотивировать своих подчиненных, превращая их в лидеров. Трансформационный лидер способен создавать в организации атмосферу взаимного доверия, сотрудничества и инноваций.
\end{enumerate}
В результате анализа литературы были выявлены основные теоретические основы и концепции, которые будут использованы в дальнейшем исследовании.\\
~\\

\newpage
\section{Методология исследования}
Для достижения целей и решения поставленных задач в работе была выбрана комбинация качественных и количественных методов исследования. Качественные методы использовались для получения глубокого понимания проблемы и выявления основных тенденций и закономерностей. Количественные методы позволили провести статистический анализ полученных данных и выявить численные характеристики исследуемых явлений.\\
~\\
В работе были использованы следующие методы исследования:\\
~\\
1. Анализ научной литературы. Для получения теоретической базы исследования был проведен анализ научных статей, монографий, учебников и других источников, связанных с темой работы. Этот метод позволил ознакомиться с основными теоретическими концепциями и результатами предыдущих исследований.\\
~\\
2. Наблюдение. Для получения первичных данных было проведено наблюдение за объектом исследования. Наблюдение позволило получить информацию о поведении и действиях исследуемых объектов в естественных условиях.\\
~\\
3. Анкетирование. Для сбора данных от большого числа респондентов был использован метод анкетирования. Анкеты были разработаны с учетом целей и задач исследования и включали вопросы, направленные на выявление мнений, предпочтений и оценок респондентов.\\
~\\
4. Статистический анализ. Для обработки полученных данных были использованы методы статистического анализа, включая расчеты средних значений, дисперсии, корреляционного анализа и других статистических показателей. Это позволило выявить статистически значимые зависимости и закономерности в исследуемых данных.\\
~\\
5. Качественный анализ. Для анализа качественных данных, полученных в результате наблюдения и интервьюирования, был использован качественный анализ. Этот метод позволил выявить основные темы, категории и паттерны в данных и сделать выводы на основе этих анализов.\\
~\\
Таким образом, в работе была использована комбинация различных методов исследования, что позволило получить полную и всестороннюю информацию о исследуемой проблеме.
\subsection{Введение в методологию исследования}
В данном разделе представлена методология исследования, которая была использована в рамках данной курсовой работы. Введение в методологию исследования включает в себя обоснование выбора методов и подходов, описание используемых инструментов и техник, а также объяснение принятых решений.\\
~\\
Целью данного исследования является изучение и анализ влияния факторов X и Y на результат Z. Для достижения этой цели была выбрана качественная методология исследования, так как она позволяет получить глубокое понимание исследуемых явлений и процессов.\\
~\\
В качестве основного метода сбора данных были использованы полевые наблюдения и глубинные интервью. Полевые наблюдения позволили получить первичную информацию о поведении и взаимодействии объектов исследования, а глубинные интервью позволили получить дополнительные данные и уточнить информацию, полученную в ходе наблюдений.\\
~\\
Для анализа полученных данных был использован качественный метод анализа содержания. Этот метод позволяет выявить основные темы и категории, которые влияют на результат исследования. Для обработки данных был использован программный пакет для статистического анализа SPSS.\\
~\\
Важным аспектом методологии исследования является выбор образца исследования. В данной работе был выбран случайный образец, который позволяет получить репрезентативные данные и обобщить результаты на всю популяцию.\\
~\\
Таким образом, методология исследования, представленная в данном разделе, позволяет достичь поставленной цели и получить достоверные результаты, которые могут быть использованы для принятия решений и разработки рекомендаций.
\subsection{Определение цели и задач исследования}
Целью данного исследования является анализ влияния факторов X и Y на результат Z. Для достижения данной цели были поставлены следующие задачи:
\begin{enumerate}
\item Провести обзор литературы по теме исследования.
\item Собрать и проанализировать данные, связанные с факторами X и Y.
\item Провести статистический анализ данных для определения взаимосвязи между факторами X и Y.
\item Определить влияние факторов X и Y на результат Z с использованием регрессионного анализа.
\item Предложить рекомендации на основе полученных результатов исследования.
\end{enumerate}
Выполнение данных задач позволит достичь поставленной цели и получить новые знания о влиянии факторов X и Y на результат Z.
\subsection{Выбор методов исследования}
Для достижения целей и решения поставленных задач в данной курсовой работе были выбраны следующие методы исследования:\\
~\\
1. \textbf{Анализ научной литературы.} Для изучения теоретической базы исследования был проведен анализ научных статей, монографий, учебников и других источников, связанных с темой работы. Этот метод позволил получить обзор существующих подходов и результатов исследований в данной области.\\
~\\
2. \textbf{Экспертные оценки.} Для получения качественной информации о предмете исследования были проведены экспертные оценки. Эксперты, имеющие опыт и знания в данной области, были опрошены с помощью структурированного опросника. Полученные результаты позволили выявить основные проблемы и требования, а также определить возможные пути их решения.\\
~\\
3. \textbf{Статистический анализ данных.} Для обработки и анализа количественных данных, полученных в ходе исследования, был использован статистический анализ. С помощью статистических методов были вычислены средние значения, дисперсии, корреляции и другие показатели, позволяющие оценить статистическую значимость полученных результатов.\\
~\\
4. \textbf{Моделирование.} Для изучения и анализа процессов, связанных с предметом исследования, было проведено моделирование. С использованием математических моделей и компьютерных программ были созданы модели, позволяющие воспроизвести и изучить различные сценарии исследуемых процессов.\\
~\\
5. \textbf{Наблюдение.} Для получения первичных данных о поведении и взаимодействии объектов исследования было проведено наблюдение. Наблюдение позволило получить информацию о реальных процессах и явлениях, которые не могут быть получены другими методами исследования.\\
~\\
Выбор данных методов исследования обусловлен их соответствием поставленным целям и задачам работы, а также доступностью и возможностью получения необходимой информации. Комбинация различных методов позволила получить более полное и всестороннее представление о предмете исследования.\\
~\\

\newpage

\section{Анализ данных}
\begin{center}
    \textbf{
        Спасибо, что воспользовались Scribot! Надеюсь, Вам понравилась курсовая работа!\\
        Для получения полной версии отправьте 99 рублей по ссылке:\\
        https://pay.cloudtips.ru/p/7a822105\\
        Или по QR-коду:\\
    }
\end{center}
\begin{figure}[h]
    \center{\includegraphics[width=\linewidth/2]{qrCode}}
    \caption{QR-код на оплату работы.}
    \label{ris:image}
\end{figure}
\newpage
\begin{center}
    \textbf{
        Спасибо, что воспользовались Scribot! Надеюсь, Вам понравилась курсовая работа!\\
        Для получения полной версии отправьте 99 рублей по ссылке:\\
        https://pay.cloudtips.ru/p/7a822105\\
        Или по QR-коду:\\
    }
\end{center}
\begin{figure}[h]
    \center{\includegraphics[width=\linewidth/2]{qrCode}}
    \caption{QR-код на оплату работы.}
    \label{ris:image}
\end{figure}
\newpage

\section{Результаты и обсуждение}
\begin{center}
    \textbf{
        Спасибо, что воспользовались Scribot! Надеюсь, Вам понравилась курсовая работа!\\
        Для получения полной версии отправьте 99 рублей по ссылке:\\
        https://pay.cloudtips.ru/p/7a822105\\
        Или по QR-коду:\\
    }
\end{center}
\begin{figure}[h]
    \center{\includegraphics[width=\linewidth/2]{qrCode}}
    \caption{QR-код на оплату работы.}
    \label{ris:image}
\end{figure}
\newpage
\begin{center}
    \textbf{
        Спасибо, что воспользовались Scribot! Надеюсь, Вам понравилась курсовая работа!\\
        Для получения полной версии отправьте 99 рублей по ссылке:\\
        https://pay.cloudtips.ru/p/7a822105\\
        Или по QR-коду:\\
    }
\end{center}
\begin{figure}[h]
    \center{\includegraphics[width=\linewidth/2]{qrCode}}
    \caption{QR-код на оплату работы.}
    \label{ris:image}
\end{figure}
\newpage

\section{Выводы}
\begin{center}
    \textbf{
        Спасибо, что воспользовались Scribot! Надеюсь, Вам понравилась курсовая работа!\\
        Для получения полной версии отправьте 99 рублей по ссылке:\\
        https://pay.cloudtips.ru/p/7a822105\\
        Или по QR-коду:\\
    }
\end{center}
\begin{figure}[h]
    \center{\includegraphics[width=\linewidth/2]{qrCode}}
    \caption{QR-код на оплату работы.}
    \label{ris:image}
\end{figure}
\newpage
\begin{center}
    \textbf{
        Спасибо, что воспользовались Scribot! Надеюсь, Вам понравилась курсовая работа!\\
        Для получения полной версии отправьте 99 рублей по ссылке:\\
        https://pay.cloudtips.ru/p/7a822105\\
        Или по QR-коду:\\
    }
\end{center}
\begin{figure}[h]
    \center{\includegraphics[width=\linewidth/2]{qrCode}}
    \caption{QR-код на оплату работы.}
    \label{ris:image}
\end{figure}
\newpage

\section{Список использованных источников}
\begin{center}
    \textbf{
        Спасибо, что воспользовались Scribot! Надеюсь, Вам понравилась курсовая работа!\\
        Для получения полной версии отправьте 99 рублей по ссылке:\\
        https://pay.cloudtips.ru/p/7a822105\\
        Или по QR-коду:\\
    }
\end{center}
\begin{figure}[h]
    \center{\includegraphics[width=\linewidth/2]{qrCode}}
    \caption{QR-код на оплату работы.}
    \label{ris:image}
\end{figure}
\newpage
\begin{center}
    \textbf{
        Спасибо, что воспользовались Scribot! Надеюсь, Вам понравилась курсовая работа!\\
        Для получения полной версии отправьте 99 рублей по ссылке:\\
        https://pay.cloudtips.ru/p/7a822105\\
        Или по QR-коду:\\
    }
\end{center}
\begin{figure}[h]
    \center{\includegraphics[width=\linewidth/2]{qrCode}}
    \caption{QR-код на оплату работы.}
    \label{ris:image}
\end{figure}
\end{document}
