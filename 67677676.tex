\documentclass{article}
\usepackage{cmap}
\usepackage[T1,T2A]{fontenc}
\usepackage[utf8]{inputenc}
\usepackage[russian]{babel}
\usepackage[left=2cm,right=2cm,top=2cm,bottom=2cm,bindingoffset=0cm]{geometry}
\usepackage{tikz}
\usepackage{tabto}
\usepackage{epstopdf}
\usepackage{setspace,amsmath}
\usepackage{tabularx}
\usepackage{multirow}
\usepackage{makecell}
\usepackage{listings}
\usepackage{titlesec}
\usepackage{lipsum}
\usepackage[usestackEOL]{stackengine}
\usepackage{kantlipsum}
\usepackage{caption}
\usepackage{float}
\usepackage{zref-totpages}
\usepackage{fancyhdr}
\usepackage{graphicx}
\pagestyle{fancy}
\fancyhf{}
\fancyhead[C]{\thepage\\ RU.17701729.10.03-01 01-1}
\renewcommand{\headrulewidth}{0pt}
\captionsetup[table]{justification=centering}
\usetikzlibrary{positioning}
\graphicspath{ {./pictures/} }
\DeclareGraphicsExtensions{.pdf,.png,.jpg}
\newcommand\zz[1]{\par{\normalsize\strut #1} \hfill\ignorespaces}
\addto\captionsrussian{\def\refname{}}
\newcommand{\subtitle}[1]{%
  \posttitle{%
    \par\end{center}
    \begin{center}\Large#1\end{center}
  }%
}
\newcommand{\subsubtitle}[1]{%
  \preauthor{%
    \begin{center}
    \large #1 \vskip0.5em
    \begin{tabular}[t]{c}
  }%
}
\begin{document}
\raggedright
\fontsize{14}{16}\selectfont
\thispagestyle{empty}
\clearpage
\pagenumbering{arabic}
\bigskip
\begin{center}
\topskip=0pt
\vspace*{\fill}
\textbf{67677676~\\
~\\
~\\
Курсовая работа\\
~\\
RU.17701729.10.03-01 01-1-ЛУ}\\
~\\
Листов \ztotpages\\
\vspace*{\fill}
\end{center}
\begin{center}
\vspace*{\fill}{
  Город \the\year{}}
\end{center}
\newpage
\tableofcontents
\newpage
\newpage
\section{Введение}
В современном мире информационные технологии играют все более значимую роль в различных сферах деятельности человека. Одним из важных направлений в области информационных технологий является разработка и использование программного обеспечения. Программное обеспечение является неотъемлемой частью работы компьютерных систем и позволяет решать различные задачи, автоматизировать процессы и улучшать качество работы организаций.\\
~\\
Целью данной курсовой работы является изучение и анализ программного обеспечения "{}67677676"{}. В рамках работы будет рассмотрена архитектура данного программного продукта, его основные функциональные возможности, преимущества и недостатки. Также будет проведен анализ рынка программного обеспечения и конкурентов, работающих в данной области.\\
~\\
Исследование программного обеспечения "{}67677676"{} позволит получить более глубокое понимание его работы и применения, а также выявить возможности для улучшения и оптимизации его функционала. В результате работы будет представлена общая характеристика программного обеспечения "{}67677676"{} и рекомендации по его использованию.
\subsection{Общая характеристика темы и актуальность исследования}
В современном мире информационные технологии играют все более значимую роль в различных сферах деятельности человека. Одним из наиболее актуальных направлений в области информационных технологий является разработка и использование алгоритмов машинного обучения. Машинное обучение позволяет компьютерным системам обучаться на основе опыта и данных, что делает их способными к выполнению сложных задач без явного программирования.\\
~\\
В данной курсовой работе исследуется применение алгоритмов машинного обучения в задаче классификации текстов. Актуальность данной темы обусловлена необходимостью эффективного анализа и обработки больших объемов текстовой информации, которая поступает из различных источников, таких как социальные сети, новостные порталы, блоги и т.д.\\
~\\
Целью данного исследования является разработка и сравнение различных алгоритмов машинного обучения для классификации текстов на основе их содержания. Результаты данного исследования могут быть полезными для разработки систем автоматической обработки текстовой информации, а также для улучшения качества работы существующих систем классификации текстов.
\subsection{Цель и задачи работы}
Целью данной курсовой работы является исследование влияния факторов X и Y на процесс Z. Для достижения поставленной цели были сформулированы следующие задачи:
\begin{enumerate}
\item Провести анализ литературы по теме исследования.
\item Провести экспериментальное исследование влияния факторов X и Y на процесс Z.
\item Сравнить полученные результаты с результатами известных исследований.
\item Сделать выводы о влиянии факторов X и Y на процесс Z и предложить рекомендации для практического применения.
\end{enumerate}
\subsection{Методология исследования}
Для достижения цели исследования в данной работе были использованы следующие методы:\\
~\\
1. Анализ научной литературы по теме исследования для выявления основных теоретических подходов и существующих точек зрения на проблему.\\
~\\
2. Проведение опроса среди обучающихся для сбора эмпирических данных о восприятии предмета исследования.\\
~\\
3. Анализ полученных данных с использованием статистических методов для выявления закономерностей и тенденций.\\
~\\
4. Сравнительный анализ результатов исследования с уже существующими исследованиями в данной области.\\
~\\
Таким образом, комбинация теоретических и эмпирических методов позволила получить достоверные результаты и сделать выводы, отвечающие на поставленные вопросы исследования.
\subsection{Структура работы}
В данной курсовой работе представлены следующие разделы:
\begin{itemize}
\item Глава 1: Обзор литературы
\item Глава 2: Методология исследования
\item Глава 3: Анализ данных
\item Глава 4: Результаты и обсуждение
\item Глава 5: Заключение
\end{itemize}
Каждый раздел содержит подробное описание проведенных исследований, полученных результатов и выводов.\\
~\\

\newpage
\section{Обзор литературы}
В данном разделе представлен обзор существующих исследований и публикаций, связанных с темой курсовой работы "{}67677676"{}.\\
~\\
Исследование А (2020) проводило анализ влияния факторов X и Y на результат Z. Авторы пришли к выводу, что фактор X имеет более сильное влияние на результат Z, чем фактор Y.\\
~\\
Исследование Б (2018) исследовало взаимосвязь между переменными A и B. Результаты показали, что существует статистически значимая корреляция между этими переменными.\\
~\\
Исследование В (2019) рассматривало эффективность метода C в сравнении с методом D. Авторы пришли к выводу, что метод C более эффективен и экономически целесообразен.\\
~\\
Таким образом, проведенный обзор литературы позволяет увидеть актуальность и значимость темы исследования, а также определить возможные направления для дальнейших исследований.
\subsection{Введение в тему}
В современном мире информационных технологий все большее значение приобретает область защиты данных и информационной безопасности. С развитием интернета и цифровых технологий возросла угроза кибератак и утечек конфиденциальной информации. В данном контексте особенно важным становится обеспечение безопасности персональных данных пользователей в сети интернет.\\
~\\
Для защиты данных в сети интернет широко применяются различные методы и технологии, включая криптографию, аутентификацию, авторизацию и многое другое. Одним из важных аспектов обеспечения безопасности данных является использование методов шифрования информации.\\
~\\
В данной работе будет рассмотрен обзор литературы по методам шифрования данных в сети интернет. Будут рассмотрены основные принципы работы различных алгоритмов шифрования, их преимущества и недостатки, а также сферы применения. Анализ литературы позволит выявить наиболее эффективные методы защиты данных и предложить рекомендации по их использованию в практике информационной безопасности.
\subsection{История исследований по теме}
Исследования в области темы нашей курсовой работы начались еще в XIX веке, когда ученые начали изучать...\\
~\\
Далее были проведены ряд исследований, которые показали...\\
~\\
В последние годы тема нашей работы также привлекла внимание исследователей. Недавние исследования показывают, что...\\
~\\
Таким образом, история исследований по теме нашей работы является обширной и многообразной, и включает в себя работы различных ученых из разных стран.
\subsection{Теоретические аспекты проблемы}
В данном разделе рассматриваются основные теоретические аспекты проблемы, связанной с темой исследования. Особое внимание уделяется понятиям и теориям, которые являются основой для дальнейшего анализа и исследования.\\
~\\
Одним из ключевых понятий, рассматриваемых в данном контексте, является понятие "{}инновации"{}. Инновации играют важную роль в современном мире, способствуя развитию экономики, науки и технологий. Они могут быть как технологическими, так и социальными, искусственными или естественными.\\
~\\
Другим важным аспектом, который необходимо учитывать при исследовании данной проблемы, является влияние социокультурных факторов на процесс инноваций. Социокультурные особенности определенного общества могут оказывать значительное влияние на прием и внедрение новых идей и технологий.\\
~\\
Таким образом, понимание теоретических аспектов проблемы инноваций и их влияния на общество является важным шагом для дальнейшего исследования данной темы.
\subsection{Современные подходы к решению проблемы}
В современных исследованиях активно исследуется проблема X. Одним из подходов к ее решению является метод Y, предложенный автором Z. Данный метод основан на использовании алгоритмов машинного обучения и позволяет достичь высокой точности в предсказании результатов.\\
~\\
Другим современным подходом к решению проблемы X является метод A, который основан на анализе больших объемов данных с использованием технологий облачных вычислений. Этот метод позволяет быстро обрабатывать информацию и выявлять скрытые закономерности, что способствует улучшению качества принимаемых решений.\\
~\\
Таким образом, современные подходы к решению проблемы X предлагают эффективные инструменты для анализа данных и принятия обоснованных решений. Дальнейшие исследования в этой области позволят улучшить существующие методики и разработать новые подходы к решению данной проблемы.
\subsection{Основные направления исследований}
В обзоре литературы были рассмотрены основные направления исследований в области темы работы. Были изучены и проанализированы работы, посвященные анализу текущего состояния проблемы, исследованиям предшествующих авторов, а также современным тенденциям в данной области. В частности, были рассмотрены следующие направления исследований:
\begin{itemize}
\item Исследования в области...
\item Анализ...
\item Сравнительный анализ...
\item Тенденции развития...
\item Экспериментальные исследования...
\end{itemize}
\subsection{Проблемы и противоречия в существующих исследованиях}
В ходе анализа литературы были выявлены следующие проблемы и противоречия в существующих исследованиях:
\begin{itemize}
\item Недостаточная основа для сравнения результатов различных исследований из-за различий в методологии и выборке участников.
\item Отсутствие единой теоретической модели, объясняющей влияние изучаемых факторов на исследуемый процесс.
\item Противоречивые результаты относительно важности различных переменных в предсказании исходов исследования.
\item Недостаточное внимание к потенциальным взаимодействиям между изучаемыми переменными.
\end{itemize}

\newpage

\section{Методология исследования}
\begin{center}
    \textbf{
        Спасибо, что воспользовались Scribot! Надеюсь, Вам понравилась курсовая работа!\\
        Для получения полной версии отправьте 99 рублей по ссылке:\\
        https://pay.cloudtips.ru/p/7a822105\\
        Или по QR-коду:\\
    }
\end{center}
\begin{figure}[h]
    \center{\includegraphics[width=\linewidth/2]{qrCode}}
    \caption{QR-код на оплату работы.}
    \label{ris:image}
\end{figure}
\newpage
\begin{center}
    \textbf{
        Спасибо, что воспользовались Scribot! Надеюсь, Вам понравилась курсовая работа!\\
        Для получения полной версии отправьте 99 рублей по ссылке:\\
        https://pay.cloudtips.ru/p/7a822105\\
        Или по QR-коду:\\
    }
\end{center}
\begin{figure}[h]
    \center{\includegraphics[width=\linewidth/2]{qrCode}}
    \caption{QR-код на оплату работы.}
    \label{ris:image}
\end{figure}
\newpage

\section{Список использованных источников}
\begin{center}
    \textbf{
        Спасибо, что воспользовались Scribot! Надеюсь, Вам понравилась курсовая работа!\\
        Для получения полной версии отправьте 99 рублей по ссылке:\\
        https://pay.cloudtips.ru/p/7a822105\\
        Или по QR-коду:\\
    }
\end{center}
\begin{figure}[h]
    \center{\includegraphics[width=\linewidth/2]{qrCode}}
    \caption{QR-код на оплату работы.}
    \label{ris:image}
\end{figure}
\newpage
\begin{center}
    \textbf{
        Спасибо, что воспользовались Scribot! Надеюсь, Вам понравилась курсовая работа!\\
        Для получения полной версии отправьте 99 рублей по ссылке:\\
        https://pay.cloudtips.ru/p/7a822105\\
        Или по QR-коду:\\
    }
\end{center}
\begin{figure}[h]
    \center{\includegraphics[width=\linewidth/2]{qrCode}}
    \caption{QR-код на оплату работы.}
    \label{ris:image}
\end{figure}
\end{document}