\documentclass{article}
\usepackage{cmap}
\usepackage[T1,T2A]{fontenc}
\usepackage[utf8]{inputenc}
\usepackage[russian]{babel}
\usepackage[left=2cm,right=2cm,top=2cm,bottom=2cm,bindingoffset=0cm]{geometry}
\usepackage{tikz}
\usepackage{tabto}
\usepackage{epstopdf}
\usepackage{setspace,amsmath}
\usepackage{tabularx}
\usepackage{multirow}
\usepackage{makecell}
\usepackage{listings}
\usepackage{titlesec}
\usepackage{lipsum}
\usepackage[usestackEOL]{stackengine}
\usepackage{kantlipsum}
\usepackage{caption}
\usepackage{float}
\usepackage{zref-totpages}
\usepackage{fancyhdr}
\usepackage{graphicx}
\pagestyle{fancy}
\fancyhf{}
\fancyhead[C]{\thepage\\ RU.17701729.10.03-01 01-1}
\renewcommand{\headrulewidth}{0pt}
\captionsetup[table]{justification=centering}
\usetikzlibrary{positioning}
\graphicspath{ {./pictures/} }
\DeclareGraphicsExtensions{.pdf,.png,.jpg}
\newcommand\zz[1]{\par{\normalsize\strut #1} \hfill\ignorespaces}
\addto\captionsrussian{\def\refname{}}
\newcommand{\subtitle}[1]{%
  \posttitle{%
    \par\end{center}
    \begin{center}\Large#1\end{center}
  }%
}
\newcommand{\subsubtitle}[1]{%
  \preauthor{%
    \begin{center}
    \large #1 \vskip0.5em
    \begin{tabular}[t]{c}
  }%
}
\begin{document}
\fontsize{14}{16}\selectfont
\thispagestyle{empty}
\clearpage
\pagenumbering{arabic}
\bigskip
\begin{center}
\topskip=0pt
\vspace*{\fill}
\textbf{МОНТАЖ СЕТИ ОСВЕЩЕНИЯ\\
ЖИЛОГО ПОМЕЩЕНИЯ, ПРОВЕРКА, ИСПЫТАНИЯ\\
~\\
~\\
~\\
Курсовая работа\\
~\\
RU.17701729.10.03-01 01-1-ЛУ}\\
~\\
Листов \ztotpages\\
\vspace*{\fill}
\end{center}
\begin{center}
\vspace*{\fill}{
  Город \the\year{}}
\end{center}
\newpage
\tableofcontents
\newpage
\newpage
\section{Введение}
В современном мире освещение является неотъемлемой частью нашей жизни. Оно играет важную роль в создании комфортной и безопасной обстановки в жилых помещениях. Правильно спроектированная и установленная система освещения позволяет создать оптимальные условия для работы, отдыха и развлечений.\\
~\\
Целью данной курсовой работы является изучение процесса монтажа сети освещения в жилом помещении, а также проведение проверки и испытаний данной системы. В работе будут рассмотрены основные этапы монтажа, необходимые материалы и инструменты, а также требования к безопасности при проведении работ.\\
~\\
В первом разделе работы будет рассмотрена теоретическая основа монтажа сети освещения. Будут рассмотрены основные принципы работы электрической сети, виды и характеристики осветительных приборов, а также требования к электробезопасности при монтаже.\\
~\\
Во втором разделе будет представлен практический аспект монтажа сети освещения. Будут описаны этапы монтажа, начиная от подготовки помещения и выбора необходимых материалов, до установки и подключения осветительных приборов.\\
~\\
В третьем разделе будет проведена проверка и испытание установленной системы освещения. Будут описаны методы и приборы, используемые для проверки работоспособности и безопасности системы. Также будут представлены результаты проведенных испытаний.\\
~\\
В заключении будут подведены итоги работы, сделаны выводы о выполненных работах, а также предложены рекомендации по улучшению системы освещения.\\
~\\
В ходе выполнения данной курсовой работы были использованы различные источники информации, включая научные статьи, нормативные документы и руководства по монтажу. Все использованные источники будут указаны в списке литературы.
\subsection{Общая информация о курсовой работе}
В данной курсовой работе рассматривается вопрос монтажа сети освещения жилого помещения, а также проведение проверок и испытаний данной сети. Освещение является одним из важных аспектов комфорта и безопасности в жилых помещениях, поэтому правильный монтаж и испытания сети освещения имеют большое значение.\\
~\\
Целью работы является изучение основных принципов монтажа сети освещения, а также разработка методики проведения проверок и испытаний данной сети. Для достижения этой цели были поставлены следующие задачи:
\begin{enumerate}
\item Изучить основные требования и нормативы, регулирующие монтаж сети освещения в жилых помещениях.
\item Разработать методику проведения проверок и испытаний сети освещения.
\item Провести практические испытания сети освещения в жилом помещении и проанализировать полученные результаты.
\item Сделать выводы о качестве монтажа и исправности сети освещения в жилом помещении.
\end{enumerate}
В работе использовались следующие методы исследования:
\begin{itemize}
\item Анализ литературных источников по теме работы.
\item Изучение нормативных документов, регулирующих монтаж и испытания сети освещения.
\item Проведение практических испытаний сети освещения в жилом помещении.
\item Статистический анализ полученных результатов.
\end{itemize}
Работа состоит из введения, трех глав, заключения и списка использованных источников. В первой главе рассматриваются основные требования и нормативы, регулирующие монтаж сети освещения. Во второй главе представлена разработанная методика проведения проверок и испытаний сети освещения. В третьей главе проводятся практические испытания сети освещения в жилом помещении и анализируются полученные результаты.\\
~\\
В заключении подводятся итоги работы, делаются выводы о качестве монтажа и исправности сети освещения в жилом помещении, а также предлагаются рекомендации по улучшению данного процесса.
\subsection{Цель и задачи работы}
Целью данной курсовой работы является разработка и монтаж сети освещения в жилом помещении, а также проведение проверки и испытаний данной сети.\\
~\\
Для достижения поставленной цели были поставлены следующие задачи:
\begin{enumerate}
\item Изучить основные принципы и требования к монтажу сети освещения в жилых помещениях.
\item Разработать проект сети освещения, учитывая особенности конкретного жилого помещения.
\item Подобрать необходимое оборудование для монтажа сети освещения.
\item Произвести монтаж сети освещения в соответствии с разработанным проектом.
\item Провести проверку и испытания сети освещения на соответствие требованиям и нормам безопасности.
\item Оценить эффективность и качество работы сети освещения в жилом помещении.
\end{enumerate}
Выполнение данных задач позволит достичь поставленной цели работы и обеспечить безопасность и комфортность освещения в жилом помещении.
\subsection{Актуальность темы}
В современном мире освещение является неотъемлемой частью жизни людей. Качество освещения в жилых помещениях напрямую влияет на комфорт и безопасность проживания. Правильно спроектированная и установленная система освещения обеспечивает не только достаточную яркость, но и равномерность освещения, что способствует улучшению зрительного комфорта и предотвращению возникновения зрительной усталости.\\
~\\
Однако, несмотря на важность освещения, монтаж и проверка сети освещения в жилых помещениях часто остаются недооцененными. Неправильно выполненный монтаж может привести к неравномерности освещения, появлению теней и бликов, а также к возникновению электрических аварий и пожаров. Поэтому актуальность данной темы заключается в необходимости изучения и понимания процесса монтажа сети освещения, а также проведения проверки и испытаний для обеспечения безопасности и качества освещения в жилых помещениях.\\
~\\
Целью данной курсовой работы является изучение основных этапов монтажа сети освещения в жилом помещении, а также разработка методики проверки и испытаний данной сети. Результаты работы могут быть использованы специалистами в области электротехники и освещения при проектировании и монтаже систем освещения в жилых помещениях.\\
~\\

\newpage
\section{Обзор существующих методов монтажа сети освещения}
Одним из наиболее распространенных методов монтажа сети освещения является метод скрытой проводки. При этом методе провода и кабели укладываются внутри стен, потолков или полов, что позволяет сохранить эстетический вид помещения. Для этого используются специальные каналы, гофры или трубы, в которых прокладываются провода. Однако данный метод требует дополнительных работ по прокладке каналов и может быть затруднен в случае ремонта или модернизации сети.\\
~\\
Другим распространенным методом монтажа является метод открытой проводки. При этом методе провода и кабели укладываются на поверхности стен, потолков или полов с использованием специальных крепежных элементов. Открытая проводка является более простым и быстрым способом монтажа, однако она несколько ухудшает эстетический вид помещения и может быть непрактичной в случае высоких потолков или сложной конфигурации помещения.\\
~\\
Также существует метод монтажа с использованием подвесных систем. При этом методе осветительные приборы крепятся к специальным подвесным конструкциям, которые могут быть выполнены из металла или пластика. Подвесные системы позволяют легко регулировать высоту и положение осветительных приборов, а также обеспечивают возможность скрыть провода и кабели. Однако данный метод требует дополнительных затрат на приобретение и установку подвесных систем.\\
~\\
Кроме того, существуют методы монтажа с использованием специальных систем управления освещением, таких как "{}умный дом"{} или "{}умное освещение"{}. При этом методе осветительные приборы подключаются к центральной системе управления, которая позволяет автоматически регулировать яркость освещения, создавать различные сценарии освещения и управлять освещением с помощью мобильного приложения или голосовых команд. Однако данный метод требует дополнительных затрат на приобретение и установку системы управления, а также на обучение пользователей.\\
~\\
Таким образом, существует несколько методов монтажа сети освещения в жилых помещениях, каждый из которых имеет свои преимущества и недостатки. Выбор оптимального метода зависит от конкретных условий и требований, таких как эстетический вид помещения, сложность монтажа, бюджет и функциональные возможности системы освещения.
\subsection{Основные понятия и определения}
В данном разделе приведены основные понятия и определения, необходимые для понимания обзора существующих методов монтажа сети освещения.\\
~\\
\textbf{Сеть освещения} - система, предназначенная для обеспечения освещения помещений или территорий.\\
~\\
\textbf{Монтаж} - комплекс работ, включающий установку, подключение и настройку оборудования сети освещения.\\
~\\
\textbf{Жилое помещение} - помещение, предназначенное для проживания людей.\\
~\\
\textbf{Проверка} - процесс, в ходе которого осуществляется контроль соответствия установленного оборудования и проведенных работ требованиям нормативных документов.\\
~\\
\textbf{Испытания} - комплекс мероприятий, направленных на проверку работоспособности и соответствия установленного оборудования требованиям технических условий и нормативных документов.\\
~\\
\textbf{Латекс} - система компьютерной вёрстки, позволяющая создавать профессионально оформленные документы, включающие математические формулы и специальные символы.
\subsection{Обзор существующих методов монтажа сети освещения}
В данном разделе рассматриваются различные методы монтажа сети освещения, используемые при проведении работ в жилых помещениях. Описываются основные принципы и технологии, применяемые при монтаже, а также их преимущества и недостатки.\\
~\\
Первым методом, рассматриваемым в данном обзоре, является метод скрытого монтажа. Он предполагает укладку проводов внутри стен, потолков или полов, что позволяет сохранить эстетический вид помещения. Для этого используются специальные каналы, гофры или трубы, в которых прокладываются провода. Преимуществами данного метода являются отсутствие видимых проводов и возможность установки светильников в любом месте помещения. Однако, данный метод требует дополнительных затрат на материалы и трудоемких работ.\\
~\\
Вторым методом является метод открытого монтажа. Он предполагает укладку проводов на поверхности стен, потолков или полов с использованием специальных крепежных элементов. Этот метод является более простым и экономичным, поскольку не требует дополнительных материалов и трудоемких работ. Однако, провода остаются видимыми, что может негативно сказываться на эстетическом восприятии помещения.\\
~\\
Третьим методом является метод комбинированного монтажа. Он предполагает использование как скрытого, так и открытого монтажа в зависимости от конкретных условий и требований. Например, в некоторых местах помещения провода могут быть уложены внутри стен, а в других  на поверхности. Этот метод позволяет достичь оптимального сочетания эстетики и экономичности, однако требует более тщательного планирования и организации работ.\\
~\\
В обзоре также рассматриваются различные технологии монтажа, такие как использование светодиодных светильников, диммеров, датчиков движения и т.д. Описываются их особенности и преимущества, а также возможные способы подключения и установки.\\
~\\
Таким образом, обзор существующих методов монтажа сети освещения позволяет ознакомиться с различными подходами к проведению работ в жилых помещениях. Выбор конкретного метода зависит от требований заказчика, особенностей помещения и доступных ресурсов.
\subsection{Монтаж проводки и электрооборудования}
Монтаж проводки и электрооборудования является одним из важных этапов при установке сети освещения в жилом помещении. В данном разделе рассмотрены основные методы монтажа проводки и электрооборудования, используемые при установке сети освещения.
\subsubsection{Монтаж проводки}
Монтаж проводки включает в себя укладку и подключение проводов, а также установку распределительных коробок и розеток. При монтаже проводки необходимо соблюдать определенные правила и нормы безопасности, чтобы исключить возможность короткого замыкания или пожара.\\
~\\
Одним из методов монтажа проводки является скрытый монтаж. При скрытом монтаже провода прокладываются внутри стен, потолков или полов. Для этого необходимо выполнить прорези или каналы в материалах конструкции помещения. После прокладки проводов прорези или каналы закрываются штукатуркой, гипсокартоном или другим отделочным материалом.\\
~\\
Другим методом монтажа проводки является открытый монтаж. При открытом монтаже провода прокладываются по поверхности стен, потолков или полов. Для этого используются специальные кабель-каналы, гофрированные трубы или крепежные элементы. Открытый монтаж проводки является более простым и быстрым способом установки, однако он менее эстетичен, поэтому чаще используется в технических помещениях.
\subsubsection{Монтаж электрооборудования}
Монтаж электрооборудования включает в себя установку и подключение выключателей, розеток, светильников и других электрических устройств. При монтаже электрооборудования необходимо соблюдать правила безопасности и правильно подключить провода к соответствующим контактам.\\
~\\
Одним из методов монтажа электрооборудования является монтаж наружной электроустановки. При этом методе электрооборудование устанавливается на поверхности стен, потолков или полов с использованием специальных крепежных элементов. Монтаж наружной электроустановки применяется в технических помещениях или в случаях, когда скрытый монтаж невозможен или нецелесообразен.\\
~\\
Другим методом монтажа электрооборудования является монтаж внутренней электроустановки. При этом методе электрооборудование устанавливается внутри стен, потолков или полов. Для этого используются специальные распределительные коробки, в которых производится подключение проводов и установка выключателей, розеток и светильников. Монтаж внутренней электроустановки является более эстетичным и предпочтительным методом, однако требует более трудоемкой установки.\\
~\\
В результате монтажа проводки и электрооборудования достигается правильное подключение сети освещения в жилом помещении. После монтажа проводки и электрооборудования необходимо провести проверку и испытания сети освещения для убедиться в ее правильной работе и соответствии нормам безопасности.\\
~\\

\newpage
\section{Планирование монтажа сети освещения}
1. Выбор оборудования:\\
~\\
Первым этапом планирования монтажа сети освещения является выбор подходящего оборудования. Для этого необходимо определить требования к освещению в помещении, учитывая его площадь, функциональное назначение и эстетические предпочтения. Важно также учесть энергоэффективность и долговечность выбранного оборудования.\\
~\\
2. Расчет освещенности:\\
~\\
Для обеспечения комфортного и безопасного освещения необходимо провести расчет освещенности помещения. Для этого используются специальные программы или формулы, учитывающие параметры помещения, тип и мощность выбранного оборудования. Результаты расчета позволяют определить оптимальное количество и расположение светильников.\\
~\\
3. Разработка схемы освещения:\\
~\\
На основе расчета освещенности разрабатывается схема освещения, которая определяет расположение светильников, их тип и мощность. Схема освещения должна учитывать функциональные зоны помещения, а также создавать гармоничную и эстетически привлекательную атмосферу.\\
~\\
4. Подготовка к монтажу:\\
~\\
После разработки схемы освещения необходимо подготовить помещение к монтажу. Это включает в себя проведение электромонтажных работ, установку электрощитка, прокладку кабелей и подготовку точек подключения светильников. Важно также учесть требования безопасности при проведении электромонтажных работ.\\
~\\
Таким образом, планирование монтажа сети освещения включает выбор оборудования, расчет освещенности, разработку схемы освещения и подготовку к монтажу. Этот процесс позволяет обеспечить эффективное и комфортное освещение в жилом помещении.
\subsection{Анализ требований к освещению жилого помещения}
Для обеспечения комфортного проживания и безопасности в жилом помещении необходимо учитывать требования к освещению. Основные факторы, которые следует учесть при планировании монтажа сети освещения, включают:
\begin{enumerate}
\item \textbf{Интенсивность освещения:} В соответствии с нормативными требованиями, интенсивность освещения в жилых помещениях должна быть достаточной для выполнения различных видов деятельности, таких как чтение, письмо, приготовление пищи и т.д. Обычно используется мера освещенности в люксах (лк) для определения требуемого уровня освещения.
\item \textbf{Равномерность освещения:} Освещение должно быть равномерным по всему помещению, чтобы избежать создания теней и засветок. Для достижения равномерности освещения необходимо правильно распределить источники света и выбрать подходящие светильники.
\item \textbf{Цветовая температура:} Выбор цветовой температуры освещения зависит от функционального назначения помещения. Например, для спальни рекомендуется использовать теплый свет с низкой цветовой температурой (около 2700K), а для рабочей зоны - более холодный свет с высокой цветовой температурой (около 5000K).
\item \textbf{Энергоэффективность:} Важным аспектом при выборе осветительных приборов является их энергоэффективность. Использование энергоэффективных источников света, таких как светодиодные лампы, позволяет снизить энергопотребление и экономить электроэнергию.
\item \textbf{Безопасность:} При монтаже сети освещения необходимо соблюдать требования безопасности, чтобы предотвратить возможные аварийные ситуации. Это включает правильное подключение проводов, использование защитных элементов (распределительных коробок, предохранителей и т.д.) и соблюдение нормативных требований по защите от поражения электрическим током.
\end{enumerate}
Анализ требований к освещению жилого помещения позволяет определить оптимальное решение для монтажа сети освещения, учитывая потребности и предпочтения владельца помещения, а также соответствие нормативным требованиям.
\subsection{Разработка схемы освещения}
Для обеспечения эффективного и комфортного освещения жилого помещения необходимо разработать соответствующую схему освещения. При разработке схемы освещения учитываются следующие факторы:
\begin{enumerate}
\item Расположение и функциональное назначение каждого помещения.
\item Расчет освещенности в каждом помещении в соответствии с требованиями нормативных документов.
\item Выбор типов и мощностей светильников.
\item Расположение светильников на потолке или стенах помещения.
\item Расчет и размещение выключателей и розеток.
\end{enumerate}
Схема освещения должна обеспечивать равномерное освещение всего помещения, отсутствие теней и бликов, а также возможность регулировки яркости света в зависимости от потребностей пользователей.\\
~\\
Для разработки схемы освещения используются специальные программы, позволяющие моделировать освещение помещения и оптимизировать его параметры. При этом учитываются такие факторы, как цвет стен и потолка, наличие окон и дверей, а также расположение мебели и других предметов интерьера.\\
~\\
После разработки схемы освещения необходимо составить спецификацию светильников, выключателей и розеток, а также провести расчет электрической нагрузки и выбрать соответствующее оборудование.\\
~\\
В результате разработки схемы освещения получается документ, содержащий план помещения с указанием расположения светильников, выключателей и розеток, а также спецификацию оборудования. Этот документ является основой для проведения монтажных работ по установке сети освещения в жилом помещении.
\subsection{Расчет освещенности помещения}
Для обеспечения комфортного освещения жилого помещения необходимо правильно рассчитать освещенность. Освещенность - это величина, характеризующая количество светового потока, приходящего на единицу площади поверхности.\\
~\\
Расчет освещенности помещения проводится с учетом следующих параметров:
\begin{enumerate}
\item Площадь помещения (\\$S\\$) - измеряется в квадратных метрах.
\item Коэффициент отражения поверхностей (\\$K\\$) - характеризует способность поверхностей отражать свет. Обычно принимается в диапазоне от 0 до 1.
\item Необходимая освещенность (\\$E_{\text{н}}\\$) - определяется в соответствии с функциональным назначением помещения и регламентируется нормативными документами. Измеряется в люксах.
\end{enumerate}
Расчет освещенности помещения производится по формуле:
\begin{equation}
E = \frac{F \cdot K}{S},
\end{equation}
где \\$E\\$ - освещенность помещения, \\$F\\$ - световой поток, создаваемый осветительными приборами.\\
~\\
Световой поток (\\$F\\$) определяется суммой световых потоков всех осветительных приборов, установленных в помещении. Для каждого осветительного прибора известен его световой поток (\\$F_{\text{пр}}\\$) и коэффициент использования (\\$\eta\\$). Световой поток осветительного прибора рассчитывается по формуле:
\begin{equation}
F = F_{\text{пр}} \cdot \eta.
\end{equation}
Таким образом, для расчета освещенности помещения необходимо знать световой поток каждого осветительного прибора, его коэффициент использования, площадь помещения и коэффициент отражения поверхностей.\\
~\\
После расчета освещенности помещения необходимо проверить соответствие полученного значения необходимой освещенности (\\$E_{\text{н}}\\$). Если расчетная освещенность меньше необходимой, необходимо увеличить количество осветительных приборов или их световой поток. Если расчетная освещенность больше необходимой, можно уменьшить количество осветительных приборов или их световой поток.\\
~\\
Также важно учесть равномерность освещения помещения. Для этого необходимо правильно расположить осветительные приборы и рассчитать расстояние между ними.\\
~\\
В результате проведенного расчета освещенности помещения можно определить оптимальное количество и тип осветительных приборов, а также их расположение для обеспечения комфортного и эффективного освещения.\\
~\\

\newpage
\section{Подготовка к монтажу сети освещения}
Подготовка к монтажу сети освещения является важным этапом работы, который включает в себя ряд подготовительных мероприятий. В данном разделе будет рассмотрена последовательность действий, необходимых для успешного монтажа сети освещения в жилом помещении.\\
~\\
1. Планирование и проектирование:\\
~\\
Перед началом монтажа сети освещения необходимо провести планирование и проектирование. Этот этап включает в себя определение требований к освещению, выбор типов и мощностей светильников, расчет необходимого количества светильников и их расположение. Также на этом этапе определяются места установки выключателей и розеток.\\
~\\
2. Приобретение необходимых материалов и инструментов:\\
~\\
После завершения проектирования необходимо приобрести все необходимые материалы и инструменты для монтажа сети освещения. К ним могут относиться светильники, провода, выключатели, розетки, клеммники, кабель-каналы и другие компоненты.\\
~\\
3. Подготовка рабочей зоны:\\
~\\
Перед началом монтажа необходимо подготовить рабочую зону. Это включает в себя очистку помещения от лишних предметов и мусора, а также защиту мебели и других поверхностей от возможных повреждений во время работ.\\
~\\
4. Подготовка электрической разводки:\\
~\\
Перед монтажом сети освещения необходимо подготовить электрическую разводку. Это включает в себя проверку состояния существующей электропроводки, замену устаревших или поврежденных проводов, установку распределительной коробки и прокладку новых проводов до мест установки светильников, выключателей и розеток.\\
~\\
5. Монтаж светильников, выключателей и розеток:\\
~\\
После подготовки электрической разводки можно приступить к монтажу светильников, выключателей и розеток. Для этого необходимо правильно подключить провода к соответствующим контактам, установить светильники на потолок или стены, а также закрепить выключатели и розетки в соответствующих местах.\\
~\\
6. Проверка и испытания:\\
~\\
После завершения монтажа необходимо провести проверку и испытания сети освещения. Это включает в себя проверку правильности подключения проводов, проверку работоспособности светильников, выключателей и розеток, а также проверку соответствия освещения требованиям и нормам безопасности.\\
~\\
Таким образом, подготовка к монтажу сети освещения включает в себя планирование и проектирование, приобретение необходимых материалов и инструментов, подготовку рабочей зоны, подготовку электрической разводки, монтаж светильников, выключателей и розеток, а также проверку и испытания сети освещения.
\subsection{Планирование монтажа сети освещения}
В данном разделе будет рассмотрено планирование монтажа сети освещения жилого помещения. Планирование монтажа является важным этапом перед началом работ, так как позволяет определить последовательность и объем работ, а также необходимые материалы и инструменты.\\
~\\
В первую очередь необходимо провести анализ помещения и определить его особенности, такие как размеры, форма, высота потолков и расположение окон. Это позволит определить количество и типы светильников, а также расположение электропроводки.\\
~\\
Далее следует разработать схему освещения, которая будет учитывать функциональные и эстетические требования. Схема освещения должна обеспечивать равномерное освещение всего помещения, а также учитывать особенности каждой зоны (например, рабочая зона, зона отдыха и т.д.).\\
~\\
После разработки схемы освещения необходимо определить требуемую мощность освещения и выбрать подходящие светильники. При выборе светильников следует учитывать их энергоэффективность, долговечность, цветовую температуру и яркость.\\
~\\
Затем следует разработать план размещения светильников и проводки. План размещения должен учитывать оптимальное расположение светильников для достижения требуемого освещения, а также удобство монтажа и обслуживания.\\
~\\
После разработки плана размещения необходимо определить необходимое количество и типы электропроводки, а также выбрать необходимые материалы и инструменты для монтажа.\\
~\\
В заключение данного раздела следует составить график работ, определить последовательность и сроки выполнения каждого этапа монтажа сети освещения. Это позволит организовать работу более эффективно и своевременно завершить монтаж сети освещения.
\subsection{Подбор и расчет осветительных приборов}
Для обеспечения оптимального освещения жилого помещения необходимо правильно подобрать и распределить осветительные приборы. При выборе осветительных приборов следует учитывать следующие факторы:
\begin{enumerate}
\item Площадь помещения. Расчет освещенности основывается на площади помещения. Для жилых помещений рекомендуется обеспечивать освещенность не менее 300 лк.
\item Высота потолков. Высота потолков влияет на выбор типа осветительных приборов. Для помещений с высокими потолками рекомендуется использовать светильники с направленным светом, чтобы обеспечить равномерное освещение на уровне пола.
\item Цвет стен и потолков. Цвет стен и потолков также влияет на освещенность помещения. Темные поверхности поглощают свет, поэтому для таких помещений рекомендуется использовать более мощные осветительные приборы.
\item Функциональное назначение помещения. Различные помещения требуют разного уровня освещенности. Например, для кухни или рабочего кабинета рекомендуется использовать более яркий свет, а для спальни или гостиной - более мягкий и рассеянный свет.
\end{enumerate}
После определения требуемого уровня освещенности и учета вышеперечисленных факторов можно приступить к выбору конкретных осветительных приборов. Для этого необходимо учитывать следующие параметры:
\begin{enumerate}
\item Мощность осветительного прибора. Мощность осветительного прибора должна быть достаточной для обеспечения требуемого уровня освещенности. Расчет мощности осветительных приборов производится по формуле:\\
~\\
\[ P = S \times E \times K, \]
где \( P \) - мощность осветительного прибора (в Вт), \( S \) - площадь помещения (в \( \text{м}^2 \)), \( E \) - требуемая освещенность (в лк), \( K \) - коэффициент запаса (обычно принимается равным 1.2).
\item Тип осветительного прибора. В зависимости от требований и функционального назначения помещения можно выбрать различные типы осветительных приборов, такие как люстры, светильники, настольные лампы и т.д.
\item Цветовая температура. Цветовая температура определяет оттенок света, который излучает осветительный прибор. Для жилых помещений рекомендуется выбирать приборы с цветовой температурой около 2700-3000 К, что соответствует теплому белому свету.
\item КПД осветительного прибора. КПД (коэффициент полезного действия) осветительного прибора показывает, какая часть энергии превращается в свет. Чем выше КПД, тем более эффективен осветительный прибор.
\end{enumerate}
После выбора осветительных приборов необходимо распределить их по помещению с учетом требуемого уровня освещенности. Расчет расположения осветительных приборов производится с использованием специальных программных средств или с помощью специалиста в области освещения.\\
~\\
Таким образом, правильный подбор и расчет осветительных приборов является важным этапом монтажа сети освещения жилого помещения. Он позволяет обеспечить комфортное и эффективное освещение, соответствующее требованиям и функциональному назначению помещения.
\subsection{Разработка схемы электрической проводки}
Для обеспечения правильной и безопасной работы сети освещения необходимо разработать схему электрической проводки. Схема должна учитывать требования электробезопасности, а также удобство эксплуатации и эстетические аспекты.\\
~\\
Схема электрической проводки должна включать в себя следующие элементы:
\begin{enumerate}
\item Расположение и тип осветительных приборов. Необходимо определить точное расположение каждого осветительного прибора в помещении, а также выбрать подходящий тип (например, люстра, бра, настольная лампа и т.д.).
\item Расположение и тип выключателей. Выключатели должны быть удобно расположены и легко доступны для пользователей. Также необходимо выбрать подходящий тип выключателей (например, одноклавишный, двухклавишный, с диммером и т.д.).
\item Расположение и тип розеток. Розетки должны быть установлены в удобных местах для подключения электроприборов. Также необходимо выбрать подходящий тип розеток (например, одинарная, двойная, с заземлением и т.д.).
\item Расположение и тип проводки. Необходимо определить маршрут проводки от электрощитка до каждого осветительного прибора, выключателя и розетки. Также необходимо выбрать подходящий тип проводки (например, медный или алюминиевый провод).
\item Расположение и тип распределительной коробки. Распределительная коробка должна быть установлена в удобном месте для подключения проводки от электрощитка и распределения проводки к осветительным приборам, выключателям и розеткам.
\end{enumerate}
При разработке схемы электрической проводки необходимо учитывать требования электробезопасности, такие как правильное заземление, защиту от короткого замыкания и перегрузки, а также соблюдение норм и правил проведения электромонтажных работ.\\
~\\
Схема электрической проводки должна быть подробно описана и визуально представлена на плане помещения. Также необходимо указать все используемые материалы и оборудование, а также привести расчеты мощности и тока для выбора подходящих проводов и защитных устройств.\\
~\\
В результате разработки схемы электрической проводки будет получен план монтажа сети освещения, который будет использоваться при проведении монтажных работ.\\
~\\

\newpage
\section{Установка и подключение осветительных приборов}
В данном разделе будет рассмотрена процедура установки и подключения осветительных приборов в жилом помещении. Для обеспечения безопасности и эффективности работы осветительной сети необходимо правильно выполнить монтаж и подключение каждого прибора.\\
~\\
1. Выбор места установки осветительных приборов\\
~\\
Перед установкой осветительных приборов необходимо определить их расположение в помещении. Расстановка приборов должна обеспечивать равномерное освещение всей площади помещения. Также следует учитывать эргономику и функциональность осветительных приборов в соответствии с требованиями заказчика.\\
~\\
2. Подготовка к установке\\
~\\
Перед установкой осветительных приборов необходимо провести следующие работы:\\
~\\
- Отключить электропитание в помещении, где будет производиться установка.\\
~\\
- Подготовить инструменты и материалы, необходимые для монтажа (отвертки, кусачки, провода, клеммы и т.д.).\\
~\\
- Проверить соответствие осветительных приборов требованиям электробезопасности и их готовность к установке.\\
~\\
3. Монтаж осветительных приборов\\
~\\
Процесс монтажа осветительных приборов включает следующие шаги:\\
~\\
- Установка крепежных элементов. В соответствии с выбранным местом установки осветительных приборов необходимо установить крепежные элементы (крюки, шурупы, подвесы и т.д.).\\
~\\
- Подключение проводов. Следует подключить провода осветительных приборов к электрической сети. Для этого необходимо правильно соединить провода приборов с проводами электрической сети, используя клеммы или другие соединительные элементы.\\
~\\
- Закрепление осветительных приборов. После подключения проводов осветительные приборы должны быть надежно закреплены на крепежных элементах.\\
~\\
4. Проверка и испытания\\
~\\
После установки и подключения осветительных приборов необходимо провести проверку и испытания работы осветительной сети. В ходе проверки следует убедиться в правильности подключения проводов, отсутствии коротких замыканий и неполадок в работе приборов. Также рекомендуется измерить напряжение и силу тока в сети для проверки соответствия требованиям безопасности.\\
~\\
В случае обнаружения неполадок или несоответствия требованиям безопасности необходимо провести дополнительные исправительные работы.\\
~\\
Таким образом, правильная установка и подключение осветительных приборов является важным этапом монтажа осветительной сети в жилом помещении. Это позволяет обеспечить безопасность и эффективность работы осветительной системы.
\subsection{Подготовка к установке осветительных приборов}
Перед установкой осветительных приборов необходимо выполнить ряд подготовительных работ. В данном разделе будет рассмотрено, какие шаги необходимо предпринять перед началом установки осветительных приборов.
\subsubsection{Определение места установки}
Первым шагом является определение места установки осветительных приборов. Для этого необходимо учесть функциональные требования освещения, а также эстетические предпочтения. Место установки должно обеспечивать равномерное освещение всего помещения и удобство использования осветительных приборов.
\subsubsection{Выбор типа осветительных приборов}
После определения места установки необходимо выбрать подходящий тип осветительных приборов. Это может быть потолочный светильник, настенный светильник, подвесной светильник и т.д. При выборе типа осветительных приборов необходимо учесть требования безопасности, энергоэффективность, дизайн и стоимость.
\subsubsection{Расчет освещенности}
Для обеспечения комфортного освещения необходимо рассчитать требуемую освещенность помещения. Освещенность измеряется в люксах и зависит от функционального назначения помещения. Расчет освещенности позволяет определить необходимую мощность осветительных приборов и их количество.
\subsubsection{Подготовка электрической сети}
Перед установкой осветительных приборов необходимо подготовить электрическую сеть. Это включает в себя проверку состояния проводки, замену старых или поврежденных проводов, установку автоматических выключателей и дифференциальных автоматов. Также необходимо убедиться в наличии свободных мест в электрической щитовой для подключения новых осветительных приборов.
\subsubsection{Приобретение необходимых материалов и инструментов}
Перед началом установки осветительных приборов необходимо приобрести все необходимые материалы и инструменты. К ним могут относиться осветительные приборы, провода, розетки, выключатели, клеммники, кабель-каналы, крепежные элементы и т.д. Также потребуется инструментарий, включающий отвертки, кусачки, пассатижи, отвертки с изолированными ручками, набор ключей и т.д.
\subsubsection{Проведение маркировки и прокладка проводов}
После подготовки электрической сети необходимо провести маркировку и прокладку проводов. Маркировка проводов позволяет идентифицировать каждый провод и упрощает последующую установку и подключение осветительных приборов. Прокладка проводов должна выполняться с соблюдением требований безопасности и эстетических норм.
\subsubsection{Подготовка осветительных приборов}
Перед установкой осветительных приборов необходимо их подготовить. Это может включать в себя сборку, подключение кабелей и проводов, установку ламп и т.д. При подготовке осветительных приборов необходимо следовать инструкциям производителя и соблюдать требования безопасности.
\subsubsection{Проверка и испытания}
После установки осветительных приборов необходимо провести проверку и испытания. Проверка включает в себя проверку правильности подключения проводов, отсутствие короткого замыкания и обрыва проводов, а также проверку работоспособности осветительных приборов. Испытания включают в себя измерение освещенности помещения, проверку работы выключателей и розеток, а также проверку электрической безопасности.
\subsubsection{Документирование}
По завершении установки осветительных приборов необходимо составить документацию. В документации должны быть указаны типы и мощности установленных осветительных приборов, схема подключения, результаты проверки и испытаний, а также рекомендации по эксплуатации и обслуживанию. Документация является важным инструментом для последующего обслуживания и ремонта осветительной сети.
\subsection{Выбор и приобретение осветительных приборов}
При выборе осветительных приборов для установки в жилом помещении необходимо учитывать ряд факторов, таких как тип помещения, его площадь, функциональное назначение, требования к освещенности и дизайну.\\
~\\
Перед приобретением осветительных приборов необходимо провести расчет освещенности помещения, определить необходимую мощность осветительных приборов и выбрать подходящую систему освещения (общее, местное, комбинированное).\\
~\\
При выборе осветительных приборов следует обратить внимание на следующие характеристики:
\begin{itemize}
\item Тип источника света (лампа): галогенная, энергосберегающая, светодиодная и т.д. Каждый тип имеет свои преимущества и недостатки, такие как энергоэффективность, цветовая температура, срок службы и стоимость.
\item Мощность осветительного прибора, которая определяется требуемой освещенностью помещения и его площадью. Рекомендуется выбирать осветительные приборы с мощностью, близкой к рассчитанной.
\item Светораспределение осветительного прибора, которое определяет, как равномерно будет распределен свет в помещении. Важно выбирать приборы с оптимальным светораспределением для конкретного помещения.
\item Цветовая температура света, которая влияет на визуальный комфорт и атмосферу помещения. Рекомендуется выбирать приборы с цветовой температурой, соответствующей требованиям и предпочтениям владельца помещения.
\item Дизайн и стиль осветительного прибора, который должен гармонично вписываться в интерьер помещения.
\item Качество и надежность осветительного прибора, которые влияют на его срок службы и безопасность использования.
\item Стоимость осветительного прибора, которая должна соответствовать бюджету владельца помещения.
\end{itemize}
Приобретение осветительных приборов можно осуществить в специализированных магазинах, интернет-магазинах или у производителей. При выборе поставщика следует обратить внимание на его репутацию, гарантийные обязательства и качество обслуживания.\\
~\\
Перед приобретением осветительных приборов рекомендуется ознакомиться с отзывами и рекомендациями других покупателей, а также проконсультироваться с профессионалами в области освещения для получения дополнительной информации и советов.
\subsection{Подключение осветительных приборов к электрической сети}
Подключение осветительных приборов к электрической сети является одним из важных этапов монтажа сети освещения жилого помещения. Для обеспечения безопасности и надежности работы осветительных приборов необходимо правильно выполнить их подключение.\\
~\\
Перед подключением осветительных приборов необходимо убедиться в отсутствии напряжения на проводах. Для этого следует использовать приборы для проверки напряжения, такие как фазовращатель или тестер напряжения. При обнаружении напряжения на проводах подключение осветительных приборов следует отложить до полного отключения электрической сети.\\
~\\
Подключение осветительных приборов производится с использованием электрических соединений. Для этого необходимо обнажить концы проводов, снять изоляцию и соединить провода с помощью клеммных колодок или пайки. При подключении необходимо учитывать правильность соединения проводов по цветам: фазовый провод (обычно красного цвета) подключается к фазовому контакту осветительного прибора, нулевой провод (обычно синего цвета) подключается к нулевому контакту, а защитный провод (обычно зеленого или желтого цвета) подключается к контакту заземления.\\
~\\
После подключения осветительных приборов необходимо проверить правильность и надежность подключения. Для этого следует включить электрическую сеть и проверить работу осветительных приборов. При обнаружении неисправностей или неправильной работы осветительных приборов необходимо отключить электрическую сеть и проверить подключение проводов.\\
~\\
Также необходимо провести испытания осветительных приборов. Для этого следует использовать специальные приборы, такие как ламповый тестер или тестер сопротивления изоляции. Испытания позволяют выявить возможные неисправности и дефекты осветительных приборов, а также проверить их соответствие требованиям безопасности.\\
~\\
Таким образом, подключение осветительных приборов к электрической сети является важным этапом монтажа сети освещения жилого помещения. Правильное подключение осветительных приборов обеспечивает безопасность и надежность работы системы освещения.\\
~\\

\newpage
\section{Проверка и испытания сети освещения}
Проверка и испытания сети освещения являются важной частью процесса монтажа освещения в жилом помещении. Они позволяют убедиться в правильности подключения и функционирования осветительных приборов, а также в безопасности эксплуатации всей системы освещения.\\
~\\
В данном разделе будут рассмотрены основные этапы проверки и испытаний сети освещения, а также приведены соответствующие методы и инструменты, необходимые для проведения этих работ.\\
~\\
1. Проверка правильности подключения осветительных приборов:\\
~\\
- Проверка соответствия подключения фазы, нулевого провода и заземления осветительных приборов согласно электрической схеме.\\
~\\
- Проверка правильности подключения выключателей и розеток, контрольных и защитных устройств.\\
~\\
2. Проверка функционирования осветительных приборов:\\
~\\
- Проверка работы выключателей и розеток, включение и выключение осветительных приборов.\\
~\\
- Проверка работы диммеров и регуляторов яркости, при необходимости настройка их работы.\\
~\\
3. Проверка безопасности эксплуатации системы освещения:\\
~\\
- Проверка отсутствия коротких замыканий и перегрузок в сети освещения.\\
~\\
- Проверка отсутствия утечки тока и замыкания на землю.\\
~\\
- Проверка соответствия уровня освещенности требованиям нормативных документов.\\
~\\
Для проведения проверки и испытаний сети освещения необходимы следующие инструменты и приборы:\\
~\\
- Вольтметр для измерения напряжения в сети освещения.\\
~\\
- Амперметр для измерения силы тока в сети освещения.\\
~\\
- Измеритель освещенности для проверки соответствия уровня освещенности требованиям.\\
~\\
- Испытательный штепсель для проверки заземления и отсутствия утечки тока.\\
~\\
- Мультиметр для измерения различных параметров сети освещения.\\
~\\
После проведения проверки и испытаний сети освещения необходимо составить протокол, в котором будут указаны результаты проверки, выявленные недостатки (если таковые имеются) и рекомендации по их устранению.\\
~\\
Таким образом, проверка и испытания сети освещения являются неотъемлемой частью процесса монтажа освещения в жилом помещении. Они позволяют убедиться в правильности подключения и функционирования осветительных приборов, а также в безопасности эксплуатации всей системы освещения.
\subsection{Подготовка к проверке и испытаниям сети освещения}
Перед проведением проверки и испытаний сети освещения необходимо выполнить ряд подготовительных мероприятий.\\
~\\
В первую очередь следует убедиться в правильности монтажа осветительных приборов и соединений. Проверка должна включать в себя осмотр всех элементов сети освещения, а также проверку качества и надежности их крепления.\\
~\\
Далее необходимо проверить правильность подключения проводов и соединений. Проверка должна включать в себя измерение сопротивления изоляции проводов и соединений, а также проверку наличия замыканий и обрывов.\\
~\\
Также перед проведением проверки и испытаний следует убедиться в правильности подключения и настройки управляющих и защитных устройств, таких как выключатели, автоматические выключатели, предохранители и др.\\
~\\
При подготовке к проверке и испытаниям сети освещения необходимо также убедиться в наличии необходимого оборудования и инструментов для проведения испытаний, таких как мультиметр, измеритель сопротивления изоляции, токовые клещи и др.\\
~\\
Важным этапом подготовки является разработка плана проверки и испытаний, в котором должны быть указаны последовательность и методы проведения испытаний, а также критерии приемки.\\
~\\
Таким образом, подготовка к проверке и испытаниям сети освещения включает в себя осмотр и проверку монтажа осветительных приборов, проверку подключения проводов и соединений, проверку управляющих и защитных устройств, наличие необходимого оборудования и разработку плана проверки и испытаний.
\subsection{Проверка правильности монтажа осветительных приборов}
Проверка правильности монтажа осветительных приборов является важным этапом проверки и испытаний сети освещения. В данном подразделе будут рассмотрены основные этапы и методы проверки правильности монтажа осветительных приборов.
\subsubsection{Визуальный осмотр}
Перед началом проверки необходимо провести визуальный осмотр осветительных приборов. В ходе осмотра следует обратить внимание на следующие аспекты:
\begin{itemize}
\item Наличие видимых повреждений корпуса осветительного прибора, трещин, сколов и других дефектов.
\item Правильность установки осветительных приборов в соответствии с проектной документацией.
\item Наличие и правильность установки защитных элементов, таких как решетки, рассеиватели и т.д.
\item Наличие и правильность установки крепежных элементов, таких как кронштейны, клеммы и т.д.
\end{itemize}
\subsubsection{Проверка электрических параметров}
После визуального осмотра необходимо провести проверку электрических параметров осветительных приборов. Для этого используются следующие методы:
\begin{itemize}
\item Измерение напряжения на клеммах осветительного прибора с помощью мультиметра. Напряжение должно соответствовать требованиям, указанным в проектной документации.
\item Измерение силы тока, потребляемого осветительным прибором, с помощью амперметра. Ток должен быть в пределах номинального значения, указанного на осветительном приборе.
\item Проверка наличия замыкания или обрыва в цепи осветительного прибора с помощью тестера или омметра. Значения сопротивления должны быть в пределах нормы.
\end{itemize}
\subsubsection{Проверка освещенности}
Для проверки правильности монтажа осветительных приборов также необходимо провести измерение освещенности в различных точках помещения. Для этого используются специальные приборы - осветительные люксы или осветительные метры. Измерения проводятся в соответствии с требованиями нормативной документации.
\subsubsection{Проверка работы выключателей и регуляторов}
Окончательным этапом проверки правильности монтажа осветительных приборов является проверка работы выключателей и регуляторов освещения. Для этого необходимо последовательно проверить работу каждого выключателя и регулятора, убедившись в их исправности и соответствии требованиям проектной документации.
\subsubsection{Оформление результатов проверки}
По результатам проверки правильности монтажа осветительных приборов необходимо составить протокол, в котором указываются следующие данные:
\begin{itemize}
\item Дата и время проведения проверки.
\item Результаты визуального осмотра осветительных приборов.
\item Результаты измерения электрических параметров осветительных приборов.
\item Результаты измерения освещенности в различных точках помещения.
\item Результаты проверки работы выключателей и регуляторов освещения.
\item Выводы о правильности монтажа осветительных приборов.
\end{itemize}
Протокол должен быть подписан ответственным лицом и приложен к документации по монтажу сети освещения.
\subsection{Проверка правильности подключения осветительных приборов к сети}
Для обеспечения безопасной и эффективной работы сети освещения необходимо проверить правильность подключения осветительных приборов к сети. Данная проверка включает в себя следующие этапы:
\begin{enumerate}
\item Визуальный осмотр осветительных приборов. Необходимо проверить, что все приборы находятся в исправном состоянии, без видимых повреждений, трещин или других дефектов. Также следует убедиться, что все элементы приборов (лампы, рефлекторы, кабели и т.д.) находятся на своих местах и надежно закреплены.
\item Проверка правильности подключения проводов. Следует убедиться, что провода, подключенные к осветительным приборам, соответствуют их цветовой маркировке. Например, фазный провод (L) должен быть подключен к контакту фазы, нулевой провод (N) - к контакту нуля, а защитный провод (PE) - к контакту заземления. Также необходимо проверить надежность и качество подключения проводов к контактам приборов.
\item Испытание изоляции. Для проверки правильности подключения осветительных приборов к сети необходимо провести испытание изоляции. Для этого используется мегаомметр, который позволяет измерить сопротивление изоляции между проводами и корпусом прибора. Значение сопротивления должно быть выше заданного предела, указанного в технической документации на прибор.
\item Проверка работы осветительных приборов. После подключения осветительных приборов к сети необходимо проверить их работоспособность. Для этого следует включить освещение и убедиться, что все приборы включаются и работают корректно. При необходимости можно провести дополнительные проверки, такие как проверка яркости света, равномерности освещения и т.д.
\end{enumerate}
Проверка правильности подключения осветительных приборов к сети является важным этапом испытаний сети освещения, поскольку неправильное подключение может привести к неисправности приборов, повреждению проводов или даже возникновению пожара. Поэтому данная проверка должна проводиться внимательно и тщательно.\\
~\\

\newpage

\section{Оценка эффективности и безопасности сети освещения}
\begin{center}
    \textbf{
        Спасибо, что воспользовались Scribot! Надеюсь, Вам понравилась курсовая работа!\\
        Для получения полной версии отправьте 99 рублей по ссылке:\\
        https://pay.cloudtips.ru/p/7a822105\\
        Или по QR-коду:\\
    }
\end{center}
\begin{figure}[h]
    \center{\includegraphics[width=\linewidth/2]{qrCode}}
    \caption{QR-код на оплату работы.}
    \label{ris:image}
\end{figure}
\newpage
\begin{center}
    \textbf{
        Спасибо, что воспользовались Scribot! Надеюсь, Вам понравилась курсовая работа!\\
        Для получения полной версии отправьте 99 рублей по ссылке:\\
        https://pay.cloudtips.ru/p/7a822105\\
        Или по QR-коду:\\
    }
\end{center}
\begin{figure}[h]
    \center{\includegraphics[width=\linewidth/2]{qrCode}}
    \caption{QR-код на оплату работы.}
    \label{ris:image}
\end{figure}
\newpage

\section{Монтаж сети освещения жилого помещения}
\begin{center}
    \textbf{
        Спасибо, что воспользовались Scribot! Надеюсь, Вам понравилась курсовая работа!\\
        Для получения полной версии отправьте 99 рублей по ссылке:\\
        https://pay.cloudtips.ru/p/7a822105\\
        Или по QR-коду:\\
    }
\end{center}
\begin{figure}[h]
    \center{\includegraphics[width=\linewidth/2]{qrCode}}
    \caption{QR-код на оплату работы.}
    \label{ris:image}
\end{figure}
\newpage
\begin{center}
    \textbf{
        Спасибо, что воспользовались Scribot! Надеюсь, Вам понравилась курсовая работа!\\
        Для получения полной версии отправьте 99 рублей по ссылке:\\
        https://pay.cloudtips.ru/p/7a822105\\
        Или по QR-коду:\\
    }
\end{center}
\begin{figure}[h]
    \center{\includegraphics[width=\linewidth/2]{qrCode}}
    \caption{QR-код на оплату работы.}
    \label{ris:image}
\end{figure}
\newpage

\section{Проверка}
\begin{center}
    \textbf{
        Спасибо, что воспользовались Scribot! Надеюсь, Вам понравилась курсовая работа!\\
        Для получения полной версии отправьте 99 рублей по ссылке:\\
        https://pay.cloudtips.ru/p/7a822105\\
        Или по QR-коду:\\
    }
\end{center}
\begin{figure}[h]
    \center{\includegraphics[width=\linewidth/2]{qrCode}}
    \caption{QR-код на оплату работы.}
    \label{ris:image}
\end{figure}
\newpage
\begin{center}
    \textbf{
        Спасибо, что воспользовались Scribot! Надеюсь, Вам понравилась курсовая работа!\\
        Для получения полной версии отправьте 99 рублей по ссылке:\\
        https://pay.cloudtips.ru/p/7a822105\\
        Или по QR-коду:\\
    }
\end{center}
\begin{figure}[h]
    \center{\includegraphics[width=\linewidth/2]{qrCode}}
    \caption{QR-код на оплату работы.}
    \label{ris:image}
\end{figure}
\newpage

\section{Испытания}
\begin{center}
    \textbf{
        Спасибо, что воспользовались Scribot! Надеюсь, Вам понравилась курсовая работа!\\
        Для получения полной версии отправьте 99 рублей по ссылке:\\
        https://pay.cloudtips.ru/p/7a822105\\
        Или по QR-коду:\\
    }
\end{center}
\begin{figure}[h]
    \center{\includegraphics[width=\linewidth/2]{qrCode}}
    \caption{QR-код на оплату работы.}
    \label{ris:image}
\end{figure}
\newpage
\begin{center}
    \textbf{
        Спасибо, что воспользовались Scribot! Надеюсь, Вам понравилась курсовая работа!\\
        Для получения полной версии отправьте 99 рублей по ссылке:\\
        https://pay.cloudtips.ru/p/7a822105\\
        Или по QR-коду:\\
    }
\end{center}
\begin{figure}[h]
    \center{\includegraphics[width=\linewidth/2]{qrCode}}
    \caption{QR-код на оплату работы.}
    \label{ris:image}
\end{figure}
\newpage

\section{Заключение}
\begin{center}
    \textbf{
        Спасибо, что воспользовались Scribot! Надеюсь, Вам понравилась курсовая работа!\\
        Для получения полной версии отправьте 99 рублей по ссылке:\\
        https://pay.cloudtips.ru/p/7a822105\\
        Или по QR-коду:\\
    }
\end{center}
\begin{figure}[h]
    \center{\includegraphics[width=\linewidth/2]{qrCode}}
    \caption{QR-код на оплату работы.}
    \label{ris:image}
\end{figure}
\newpage
\begin{center}
    \textbf{
        Спасибо, что воспользовались Scribot! Надеюсь, Вам понравилась курсовая работа!\\
        Для получения полной версии отправьте 99 рублей по ссылке:\\
        https://pay.cloudtips.ru/p/7a822105\\
        Или по QR-коду:\\
    }
\end{center}
\begin{figure}[h]
    \center{\includegraphics[width=\linewidth/2]{qrCode}}
    \caption{QR-код на оплату работы.}
    \label{ris:image}
\end{figure}
\newpage

\section{Список использованных источников}
\begin{center}
    \textbf{
        Спасибо, что воспользовались Scribot! Надеюсь, Вам понравилась курсовая работа!\\
        Для получения полной версии отправьте 99 рублей по ссылке:\\
        https://pay.cloudtips.ru/p/7a822105\\
        Или по QR-коду:\\
    }
\end{center}
\begin{figure}[h]
    \center{\includegraphics[width=\linewidth/2]{qrCode}}
    \caption{QR-код на оплату работы.}
    \label{ris:image}
\end{figure}
\newpage
\begin{center}
    \textbf{
        Спасибо, что воспользовались Scribot! Надеюсь, Вам понравилась курсовая работа!\\
        Для получения полной версии отправьте 99 рублей по ссылке:\\
        https://pay.cloudtips.ru/p/7a822105\\
        Или по QR-коду:\\
    }
\end{center}
\begin{figure}[h]
    \center{\includegraphics[width=\linewidth/2]{qrCode}}
    \caption{QR-код на оплату работы.}
    \label{ris:image}
\end{figure}
\end{document}
