\documentclass{article}
\usepackage{cmap}
\usepackage[T1,T2A]{fontenc}
\usepackage[utf8]{inputenc}
\usepackage[russian]{babel}
\usepackage[left=2cm,right=2cm,top=2cm,bottom=2cm,bindingoffset=0cm]{geometry}
\usepackage{tikz}
\usepackage{tabto}
\usepackage{epstopdf}
\usepackage{setspace,amsmath}
\usepackage{tabularx}
\usepackage{multirow}
\usepackage{makecell}
\usepackage{listings}
\usepackage{titlesec}
\usepackage{lipsum}
\usepackage[usestackEOL]{stackengine}
\usepackage{kantlipsum}
\usepackage{caption}
\usepackage{float}
\usepackage{zref-totpages}
\usepackage{fancyhdr}
\usepackage{graphicx}
\pagestyle{fancy}
\fancyhf{}
\fancyhead[C]{\thepage\\ RU.17701729.10.03-01 01-1}
\renewcommand{\headrulewidth}{0pt}
\captionsetup[table]{justification=centering}
\usetikzlibrary{positioning}
\graphicspath{ {./pictures/} }
\DeclareGraphicsExtensions{.pdf,.png,.jpg}
\newcommand\zz[1]{\par{\normalsize\strut #1} \hfill\ignorespaces}
\addto\captionsrussian{\def\refname{}}
\newcommand{\subtitle}[1]{%
  \posttitle{%
    \par\end{center}
    \begin{center}\Large#1\end{center}
  }%
}
\newcommand{\subsubtitle}[1]{%
  \preauthor{%
    \begin{center}
    \large #1 \vskip0.5em
    \begin{tabular}[t]{c}
  }%
}
\begin{document}
\fontsize{14}{16}\selectfont
\thispagestyle{empty}
\clearpage
\pagenumbering{arabic}
\bigskip
\begin{center}
\topskip=0pt
\vspace*{\fill}
\textbf{СВЕТ КАК ЭЛЕКТРОМАГНИТНАЯ ВОЛНА (ЭМВ).\\
ШКАЛА ЭМВ И ЧЕЛОВЕЧЕСКАЯ СПОСОБНОСТЬ\\
ВОСПРИЯТИЯ СВЕТА ОБЪЕМ 7 СТРАНИЦ\\
~\\
~\\
~\\
Курсовая работа\\
~\\
RU.17701729.10.03-01 01-1-ЛУ}\\
~\\
Листов \ztotpages\\
\vspace*{\fill}
\end{center}
\begin{center}
\vspace*{\fill}{
  Город \the\year{}}
\end{center}
\newpage
\tableofcontents
\newpage
\newpage
\section{Введение}
В настоящее время свет является одним из наиболее изучаемых и важных физических явлений. Он играет огромную роль в нашей повседневной жизни, обеспечивая нам возможность видеть окружающий мир. Свет является электромагнитной волной, которая распространяется в пространстве со скоростью света.\\
~\\
Целью данной курсовой работы является изучение света как электромагнитной волны и его восприятия человеком. В работе будет рассмотрена шкала электромагнитных волн и их основные характеристики. Также будет исследована способность человека воспринимать свет различных длин волн и его влияние на наше зрение.\\
~\\
В первом разделе работы будет рассмотрена природа света как электромагнитной волны. Будут рассмотрены основные свойства электромагнитных волн, их взаимосвязь с электрическими и магнитными полями, а также способы их генерации и детектирования.\\
~\\
Во втором разделе будет представлена шкала электромагнитных волн. Будут рассмотрены различные диапазоны электромагнитного спектра, их особенности и применение в различных областях науки и техники.\\
~\\
В третьем разделе будет исследована способность человека воспринимать свет различных длин волн. Будут рассмотрены особенности работы глаза и механизмы восприятия света. Также будет рассмотрено влияние света на наше зрение и возможные последствия его неправильного использования.\\
~\\
В заключении будут подведены итоги работы, сделаны выводы и предложены возможные направления для дальнейших исследований в данной области.\\
~\\
Исследование света как электромагнитной волны и его восприятия человеком имеет большое практическое значение. Понимание основных принципов работы света позволяет разрабатывать новые технологии в области оптики, фотоники, медицины и других отраслях. Также изучение влияния света на человека помогает создавать комфортные условия освещения и предотвращать возможные проблемы со зрением.
\subsection{Определение понятия "{}{}свет как электромагнитная волна"{}{}}
\section{Введение}
В данной курсовой работе рассматривается понятие "{}{}свет как электромагнитная волна"{}{} и его связь с человеческой способностью восприятия света. Свет является одной из основных форм энергии, которая возникает в результате электромагнитных колебаний. Он играет важную роль в нашей жизни, обеспечивая нам возможность видеть окружающий мир.\\
~\\
Определение понятия "{}{}свет как электромагнитная волна"{}{} основывается на физической теории электромагнетизма. Согласно этой теории, свет представляет собой электромагнитные волны определенного диапазона частот. Эти волны состоят из электрического и магнитного поля, которые перпендикулярны друг другу и распространяются в пространстве со скоростью света.\\
~\\
Основные характеристики света как электромагнитной волны включают его частоту, длину волны и интенсивность. Частота световых волн определяет цвет, который мы воспринимаем, а длина волны связана с энергией, переносимой светом. Интенсивность света определяет его яркость.\\
~\\
Человеческое восприятие света основано на способности глаза и мозга обрабатывать электромагнитные волны определенного диапазона частот. Человеческий глаз способен воспринимать световые волны в диапазоне от красного до фиолетового цвета, который называется видимым спектром. Однако существуют и другие формы электромагнитных волн, которые не воспринимаются человеческим глазом, такие как инфракрасное и ультрафиолетовое излучение.\\
~\\
Целью данной курсовой работы является изучение света как электромагнитной волны и его влияния на человеческую способность восприятия света. В работе будут рассмотрены основные свойства света, его взаимодействие с веществом, а также влияние различных факторов на восприятие света человеком.
\subsection{Исторический обзор развития представлений о свете}
Изучение света и его свойств является одной из важнейших задач физики. Уже в древние времена люди обращали внимание на свет и его влияние на окружающий мир. Однако, до настоящего времени представления о свете и его природе претерпели значительные изменения.\\
~\\
Одной из первых теорий, объясняющих природу света, была теория эмиссии, предложенная античными философами. Согласно этой теории, свет испускается источником и распространяется в виде потоков частиц, называемых корпускулами света. Такую модель света поддерживали такие ученые, как Демокрит, Эпикур и Лукреций.\\
~\\
Однако, в 17 веке Гюйгенс предложил альтернативную теорию, известную как волновая теория света. Согласно этой теории, свет распространяется в виде волн, а не частиц. Волновая теория света объясняла такие явления, как интерференция и дифракция, которые не могли быть объяснены теорией эмиссии.\\
~\\
В 19 веке Максвелл разработал электромагнитную теорию света, которая объединила в себе волновую теорию света и электромагнитную теорию. Согласно этой теории, свет представляет собой электромагнитную волну, распространяющуюся в пространстве. Эта теория была подтверждена экспериментально и стала основой для современной физики света.\\
~\\
В 20 веке была разработана квантовая теория света, которая объясняет такие явления, как фотоэффект и комбинационное рассеяние света. Согласно квантовой теории, свет представляет собой поток квантов энергии, называемых фотонами.\\
~\\
Современные представления о свете основаны на электромагнитной и квантовой теориях. Они позволяют объяснить множество явлений, связанных со светом, и находят широкое применение в различных областях науки и техники.
\subsection{Физические основы света как электромагнитной волны}
\section{Введение}
Физические основы света как электромагнитной волны\\
~\\
Свет является одной из форм электромагнитного излучения, которое распространяется в виде волн. Электромагнитные волны состоят из электрического и магнитного поля, которые перпендикулярны друг другу и распространяются в пространстве. Световые волны имеют различные длины и частоты, что определяет их цветовые характеристики.\\
~\\
Основные физические основы света как электромагнитной волны связаны с волновыми свойствами электромагнитного излучения. Волны света могут быть описаны с помощью таких параметров, как длина волны, частота, амплитуда и фаза. Длина волны определяет цвет света, а частота связана с энергией фотонов, составляющих световую волну.\\
~\\
Световые волны могут распространяться в различных средах, таких как воздух, вода или стекло. При переходе из одной среды в другую, свет может претерпевать явления отражения, преломления и дисперсии. Отражение света от поверхности определяет явление отражения, а преломление света при переходе из одной среды в другую объясняется законами преломления. Дисперсия света связана с его разложением на составляющие цвета при прохождении через прозрачные среды.\\
~\\
Основные характеристики света как электромагнитной волны включают его интенсивность, направление распространения и поляризацию. Интенсивность света определяет его яркость и зависит от амплитуды волны. Направление распространения света определяется вектором распространения, который перпендикулярен волновому фронту. Поляризация света связана с ориентацией электрического поля в плоскости перпендикулярной вектору распространения.\\
~\\
В данной курсовой работе будет рассмотрена шкала электромагнитных волн и их влияние на способность человека воспринимать свет. Будут рассмотрены основные физические основы света как электромагнитной волны, а также его взаимодействие с окружающей средой и человеческим организмом.\\
~\\

\newpage
\section{Основные понятия и определения}
В данном разделе представлены основные понятия и определения, необходимые для понимания темы работы.
\subsection{Свет как электромагнитная волна}
Свет  это электромагнитная волна определенного диапазона частот, видимая для человеческого глаза. Он обладает двумя основными свойствами: волновым и корпускулярным. Волновое свойство света проявляется в его способности распространяться в пространстве в виде электромагнитных колебаний. Корпускулярное свойство света проявляется в его способности взаимодействовать с веществом как поток частиц  фотонов.
\subsection{Электромагнитные волны}
Электромагнитная волна  это колебание электрического и магнитного полей, распространяющееся в пространстве со скоростью света. Она образуется в результате взаимодействия электрического и магнитного полей, которые возникают при движении электрических зарядов.
\subsection{Шкала электромагнитных волн}
Шкала электромагнитных волн представляет собой упорядоченный по возрастанию ряд различных видов электромагнитных волн, от самых коротких до самых длинных. Включает в себя следующие области: радиоволны, микроволны, инфракрасное излучение, видимый свет, ультрафиолетовое излучение, рентгеновское излучение и гамма-излучение.
\subsection{Человеческая способность восприятия света}
Человеческое зрение  это способность глаза воспринимать электромагнитные волны определенного диапазона частот, которые мы называем светом. Человеческое зрение ограничено определенным диапазоном длин волн, который называется видимым спектром. Видимый спектр включает в себя длины волн от 400 до 700 нм. Восприятие света человеком осуществляется благодаря специальным клеткам  светочувствительным рецепторам  расположенным на сетчатке глаза.
\subsection{Определение света как электромагнитной волны}
Свет - это электромагнитная волна определенного диапазона частот, видимая для человеческого глаза. Он представляет собой электромагнитные колебания, распространяющиеся в пространстве со скоростью света. Световые волны обладают двумя основными характеристиками - частотой и длиной волны.\\
~\\
Частота световых волн определяет количество колебаний, происходящих в единицу времени, и измеряется в герцах (Гц). Диапазон частот света, видимого для человеческого глаза, составляет примерно от \\$4 \times 10^{14}\\$ Гц (фиолетовый) до \\$7.5 \times 10^{14}\\$ Гц (красный).\\
~\\
Длина волны света определяет расстояние между двумя соседними точками на волне и измеряется в нанометрах (нм). Диапазон длин волн света, видимого для человеческого глаза, составляет примерно от 400 нм (фиолетовый) до 700 нм (красный).\\
~\\
Световые волны могут быть описаны с помощью электромагнитной теории, которая объясняет их поведение и взаимодействие с другими объектами. Согласно этой теории, свет состоит из электрического и магнитного поля, перпендикулярных друг другу и перпендикулярных направлению распространения волны.\\
~\\
Основные свойства света как электромагнитной волны включают интерференцию, дифракцию, поляризацию и отражение. Интерференция - это явление, при котором две или более волны сливаются вместе, создавая усиление или ослабление. Дифракция - это явление, при котором световая волна изгибается вокруг препятствия или проходит через узкое отверстие. Поляризация - это явление, при котором световая волна колеблется только в одной плоскости. Отражение - это явление, при котором световая волна отражается от поверхности и меняет направление распространения.\\
~\\
Определение света как электромагнитной волны является основой для понимания его природы и взаимодействия с окружающей средой. Это позволяет нам объяснить множество явлений, связанных с светом, и применять его в различных областях, таких как оптика, фотоника и коммуникации.
\subsection{Длина волны}
Длина волны света (\\$\lambda\\$) определяется как расстояние между двумя соседними точками на волне, которые находятся в фазе. Длина волны измеряется в метрах (м) и обычно находится в диапазоне от нанометров (нм) до микрометров (мкм). Различные длины волн света соответствуют различным цветам.
\subsection{Частота}
Частота световой волны (\\$f\\$) определяется как количество колебаний, происходящих за единицу времени. Частота измеряется в герцах (Гц) и обратно пропорциональна длине волны света. Частота света связана с его энергией: чем выше частота, тем больше энергии несет световая волна.
\subsection{Интенсивность}
Интенсивность света (\\$I\\$) определяется как количество энергии, переносимой световой волной через единицу площади в единицу времени. Интенсивность измеряется в ваттах на квадратный метр (Вт/м\\$^2\\$) и зависит от энергии световой волны и площади, на которую она падает.
\subsection{Поляризация}
Поляризация света относится к ориентации электрического поля световой волны. Свет может быть линейно поляризованным, когда электрическое поле колеблется только в одной плоскости, или кругово или эллиптически поляризованным, когда электрическое поле колеблется вокруг оси, перпендикулярной направлению распространения света.
\subsection{Скорость распространения}
Свет распространяется в вакууме со скоростью \\$c\\$, которая является постоянной и равной приблизительно \\$3 \times 10^8\\\$ м/с. Однако скорость света может изменяться при прохождении через различные среды, такие как вода или стекло.
\subsection{Дисперсия}
Дисперсия света относится к зависимости его скорости и длины волны от среды, через которую он проходит. Различные материалы могут вызывать различную дисперсию света, что приводит к явлению разложения света на составляющие его цвета при прохождении через призму или дифракционную решетку.
\subsection{Интерференция и дифракция}
Интерференция и дифракция света являются явлениями, связанными с его волновыми свойствами. Интерференция возникает при наложении двух или более световых волн, что приводит к усилению или ослаблению света в зависимости от фазы волн. Дифракция света происходит, когда свет проходит через узкое отверстие или препятствие, вызывая его изгибание и образование интерференционных полос или дифракционных картин.
\subsection{Фотоэффект}
Фотоэффект относится к явлению высвобождения электронов из поверхности материала при освещении светом. Фотоэффект объясняется тем, что световые фотоны передают свою энергию электронам, преодолевая работу выхода электронов из материала. Фотоэффект имеет важное значение в фотоэлектрических устройствах, таких как солнечные батареи и фотоэлементы.
\subsection{Видимый спектр электромагнитных волн}
Видимый спектр электромагнитных волн представляет собой узкий диапазон частот электромагнитных волн, которые способны вызывать восприятие света у человека. Видимый спектр охватывает диапазон частот примерно от 430 до 770 терагерц (ТГц), что соответствует длинам волн от примерно 400 до 700 нанометров (нм). В этом диапазоне частот находятся различные цвета, которые мы видим: красный, оранжевый, желтый, зеленый, голубой, синий и фиолетовый.\\
~\\
Видимый спектр электромагнитных волн является лишь частью более широкого электромагнитного спектра, который включает в себя и другие диапазоны частот, такие как радиоволны, микроволны, инфракрасное излучение, ультрафиолетовое излучение, рентгеновское излучение и гамма-излучение. Каждый диапазон частот имеет свои особенности и применения в науке, технологии и медицине.\\
~\\
Видимый спектр электромагнитных волн играет важную роль в нашей жизни. Он позволяет нам видеть окружающий мир, различать цвета и формы объектов. Кроме того, видимый спектр используется в различных технологиях, таких как освещение, дисплеи, фотография, видеозапись и многое другое.\\
~\\
Человеческая способность воспринимать свет ограничена видимым спектром электромагнитных волн. Мы не можем видеть электромагнитные волны с частотами ниже и выше видимого спектра. Например, мы не видим инфракрасное излучение, которое имеет более низкую частоту, и ультрафиолетовое излучение, которое имеет более высокую частоту. Это связано с особенностями строения глаза и способности его фоторецепторов (колбочек и палочек) воспринимать определенные диапазоны частот.\\
~\\
Видимый спектр электромагнитных волн является одним из основных объектов изучения в области оптики и физики света. Изучение видимого спектра позволяет лучше понять природу света и его взаимодействие с материей, а также разрабатывать новые технологии и приборы на его основе.\\
~\\

\newpage
\section{Структура электромагнитной волны}
Электромагнитная волна (ЭМВ) представляет собой распространяющееся в пространстве колебание электрического и магнитного поля. Она обладает определенной структурой, которая определяет ее свойства и характеристики.\\
~\\
Структура электромагнитной волны включает в себя следующие элементы:\\
~\\
1. Электрическое поле (Е): Это одно из основных полей, которое возникает в результате колебаний зарядов. Оно характеризуется направлением и интенсивностью. Волна создается в результате изменения направления и интенсивности электрического поля.\\
~\\
2. Магнитное поле (Н): Второе основное поле, возникающее в результате колебаний зарядов. Оно также характеризуется направлением и интенсивностью. Магнитное поле перпендикулярно электрическому полю и изменяется синхронно с ним.\\
~\\
3. Вектор Пойнтинга (S): Это векторная величина, которая определяет направление и интенсивность энергетического потока волны. Он перпендикулярен электрическому и магнитному полям и направлен по вектору распространения волны.\\
~\\
4. Частота (f): Это количество колебаний волны, происходящих за единицу времени. Измеряется в герцах (Гц). Частота волны определяет ее цветовую характеристику.\\
~\\
5. Длина волны (): Это расстояние между двумя соседними точками, находящимися в фазе колебаний. Измеряется в метрах (м). Длина волны обратно пропорциональна ее частоте.\\
~\\
6. Амплитуда (А): Это максимальное значение электрического или магнитного поля волны. Она характеризует интенсивность волны и определяет ее яркость или громкость.\\
~\\
7. Фаза (): Это смещение колебаний волны относительно некоторой начальной точки. Фаза определяет положение волны в пространстве и времени.\\
~\\
Структура электромагнитной волны может быть представлена в виде синусоидальной функции, где электрическое и магнитное поля меняются по синусоидальному закону. Волна распространяется в пространстве со скоростью света, которая составляет около 299 792 458 м/с.\\
~\\
Знание структуры электромагнитной волны позволяет понять ее свойства и взаимодействие с окружающей средой. Это основа для изучения явлений, связанных с электромагнитным излучением, включая световые волны.
\subsection{Определение электромагнитной волны}
Электромагнитная волна - это распространяющееся в пространстве возмущение электромагнитного поля, которое переносит энергию и имеет свойства волны. Она состоит из взаимно перпендикулярных колебаний электрического и магнитного полей, которые изменяются во времени и пространстве.\\
~\\
Электромагнитные волны могут иметь различные длины и частоты, образуя электромагнитный спектр. Видимый свет является частью этого спектра и имеет длину волны от приблизительно 400 до 700 нанометров.\\
~\\
Основные характеристики электромагнитной волны включают амплитуду, частоту, длину волны и скорость распространения. Амплитуда определяет максимальное значение электрического или магнитного поля волны. Частота указывает на количество колебаний волны за единицу времени, а длина волны - расстояние между двумя соседними точками с одинаковой фазой колебаний. Скорость распространения электромагнитной волны в вакууме равна скорости света и составляет приблизительно \\$3 \times 10^8\\$ метров в секунду.\\
~\\
Электромагнитные волны играют важную роль во многих аспектах нашей жизни, включая свет, радиоволны, телевидение, радиосвязь, микроволны, инфракрасное и ультрафиолетовое излучение. Понимание структуры и свойств электромагнитных волн является основой для разработки и применения различных технологий и устройств.
\subsection{Структура электромагнитной волны}
Электромагнитная волна представляет собой распространяющееся в пространстве изменение электрического и магнитного полей. Она обладает определенной структурой, которая включает в себя следующие элементы:\\
~\\
1. \textbf{Электрическое поле} - это физическое поле, создаваемое заряженными частицами, которое оказывает воздействие на другие заряженные частицы. В электромагнитной волне электрическое поле изменяется со временем и пространством, создавая колебания.\\
~\\
2. \textbf{Магнитное поле} - это физическое поле, создаваемое движущимися зарядами, которое оказывает воздействие на другие заряды и магнитные моменты. В электромагнитной волне магнитное поле также изменяется со временем и пространством, синхронно с изменениями электрического поля.\\
~\\
3. \textbf{Период} - это временной интервал, за который происходит одно полное колебание электромагнитной волны. Он обозначается символом T и измеряется в секундах.\\
~\\
4. \textbf{Частота} - это количество полных колебаний электромагнитной волны, происходящих за единицу времени. Она обозначается символом f и измеряется в герцах (Гц).\\
~\\
5. \textbf{Длина волны} - это расстояние между двумя соседними точками на волне, которые находятся в одной фазе колебания. Она обозначается символом  (лямбда) и измеряется в метрах.\\
~\\
6. \textbf{Амплитуда} - это максимальное значение изменения электрического или магнитного поля во время колебаний электромагнитной волны. Она обозначается символом A и измеряется в вольтах или амперах.\\
~\\
7. \textbf{Скорость распространения} - это скорость, с которой электромагнитная волна передвигается в пространстве. В вакууме эта скорость равна скорости света и составляет примерно 299 792 458 метров в секунду.\\
~\\
Структура электромагнитной волны позволяет ей распространяться в пространстве и взаимодействовать с другими объектами и средами. Это явление имеет большое значение в различных областях науки и техники, включая оптику, радиофизику, телекоммуникации и многие другие.
\subsection{Электромагнитное поле}
лектромагнитное поле является одним из основных понятий в физике и играет важную роль в описании света как электромагнитной волны. Оно представляет собой физическое поле, которое возникает вокруг электрических зарядов и токов, и взаимодействует с другими зарядами и токами.\\
~\\
Электромагнитное поле описывается с помощью электромагнитных полей, которые включают электрическое поле и магнитное поле. Электрическое поле создается электрическими зарядами и описывается с помощью электрического поляризационного вектора. Магнитное поле создается движущимися электрическими зарядами и описывается с помощью магнитного поляризационного вектора.\\
~\\
В электромагнитном поле электрическое и магнитное поля взаимосвязаны и взаимодействуют друг с другом. Изменение электрического поля порождает магнитное поле, а изменение магнитного поля порождает электрическое поле. Это взаимодействие создает электромагнитные волны, включая свет.\\
~\\
Свет является электромагнитной волной определенного диапазона частот, который воспринимается человеческим глазом. Он состоит из электрического и магнитного поля, которые колеблются перпендикулярно друг другу и перпендикулярно направлению распространения волны.\\
~\\
Электромагнитное поле света имеет волновую природу и может быть описано с помощью различных параметров, таких как амплитуда, частота, длина волны и фаза. Амплитуда определяет интенсивность света, частота определяет цвет света, а длина волны определяет его спектральные характеристики.\\
~\\
Человеческая способность восприятия света основана на взаимодействии электромагнитного поля света с фоторецепторами в глазу. Фоторецепторы преобразуют энергию света в электрические сигналы, которые затем передаются в мозг для обработки и интерпретации.\\
~\\
В заключение, электромагнитное поле играет ключевую роль в описании света как электромагнитной волны. Оно включает электрическое и магнитное поля, которые взаимодействуют друг с другом и создают электромагнитные волны, включая свет. Человеческая способность восприятия света основана на взаимодействии электромагнитного поля света с фоторецепторами в глазу.\\
~\\

\newpage
\section{Свойства света как электромагнитной волны}
Свет является электромагнитной волной, что означает, что он обладает рядом характеристик, свойственных электромагнитным волнам. В этом разделе мы рассмотрим основные свойства света как электромагнитной волны.\\
~\\
1. Волновая природа света\\
~\\
Свет обладает волновыми свойствами, такими как дифракция, интерференция и поляризация. Дифракция - это явление изгибания света вокруг препятствий или отверстий. Интерференция - это явление наложения двух или более световых волн, что приводит к усилению или ослаблению света в зависимости от фазы волн. Поляризация - это явление, при котором свет распространяется в определенной плоскости.\\
~\\
2. Скорость света\\
~\\
Свет распространяется со скоростью, равной приблизительно 299 792 458 метров в секунду в вакууме. Эта скорость является максимальной скоростью, достижимой в природе, и она не зависит от частоты или длины волны света.\\
~\\
3. Длина волны и частота\\
~\\
Свет имеет определенную длину волны и частоту. Длина волны света определяется расстоянием между двумя соседними точками на волне, которые находятся в фазе. Частота света определяется количеством колебаний, выполняемых световой волной за единицу времени. Длина волны и частота связаны между собой соотношением: скорость света = длина волны  частота.\\
~\\
4. Интенсивность света\\
~\\
Интенсивность света определяет количество энергии, переносимой световой волной за единицу времени через единичную площадку. Интенсивность света зависит от амплитуды световой волны, то есть от максимального значения колебаний электрического и магнитного поля.\\
~\\
5. Поляризация света\\
~\\
Свет может быть поляризованным, то есть распространяться в определенной плоскости. Поляризация света может быть линейной, когда вектор электрического поля колеблется в одной плоскости, или круговой, когда вектор электрического поля вращается вокруг направления распространения света.\\
~\\
6. Закон преломления света\\
~\\
Закон преломления света описывает изменение направления распространения света при переходе из одной среды в другую. Закон преломления устанавливает, что угол падения света равен углу преломления и что отношение синуса угла падения к синусу угла преломления является постоянной величиной, называемой показателем преломления.\\
~\\
В данном разделе мы рассмотрели основные свойства света как электромагнитной волны. Понимание этих свойств позволяет нам лучше понять природу света и его взаимодействие с окружающей средой.
\subsection{Определение света как электромагнитной волны}
Свет  это электромагнитное излучение, которое воспринимается человеческим глазом. Он представляет собой электромагнитные волны определенного диапазона частот, называемого спектром света. Световые волны обладают двумя основными характеристиками  длиной волны и частотой.\\
~\\
Согласно электромагнитной теории света, световые волны состоят из электрического и магнитного поля, которые перпендикулярны друг другу и распространяются в пространстве. Волны света могут быть описаны с помощью электромагнитных уравнений Максвелла.\\
~\\
Длина волны света определяет цвет, который мы видим. Видимый спектр света включает в себя различные цвета, начиная от красного с наибольшей длиной волны, до фиолетового с наименьшей длиной волны. Частота световых волн обратно пропорциональна их длине волны.\\
~\\
Световые волны могут распространяться в вакууме со скоростью, называемой скоростью света. В вакууме скорость света составляет приблизительно 299 792 458 метров в секунду.\\
~\\
Основные свойства света как электромагнитной волны включают интерференцию, дифракцию, поляризацию и отражение. Интерференция света  это явление, при котором две или более световых волн перекрываются и образуют интерференционные полосы. Дифракция света  это явление, при котором световая волна проходит через отверстие или препятствие и изменяет свое направление распространения. Поляризация света  это явление, при котором световая волна колеблется только в одной плоскости. Отражение света  это явление, при котором световая волна отражается от поверхности и меняет свое направление распространения.\\
~\\
Определение света как электромагнитной волны является основой для понимания его свойств и взаимодействия с окружающей средой. Это позволяет разрабатывать различные технологии и приборы, основанные на использовании света, такие как оптические волокна, лазеры, фотодетекторы и другие.
\subsection{Основные свойства света}
Свет является электромагнитной волной, обладающей рядом основных свойств:\\
~\\
1. \textbf{Интерференция и дифракция.} Свет может проявлять интерференцию и дифракцию, что связано с его волновой природой. Интерференция - это явление, при котором две или более волн света перекрываются и образуют интерференционные полосы. Дифракция - это явление, при котором свет изгибается при прохождении через узкое отверстие или препятствие.\\
~\\
2. \textbf{Отражение и преломление.} Свет может отражаться от поверхностей и преломляться при переходе из одной среды в другую. Законы отражения и преломления света описывают, как свет меняет направление при взаимодействии с поверхностями разных сред.\\
~\\
3. \textbf{Поляризация.} Свет может быть поляризованным, то есть иметь определенную ориентацию колебаний электрического и магнитного полей. Поляризация света может быть линейной, круговой или эллиптической.\\
~\\
4. \textbf{Интенсивность.} Свет имеет определенную интенсивность, которая определяется энергией, переносимой световой волной за единицу времени через единичную площадку.\\
~\\
5. \textbf{Скорость распространения.} Свет распространяется со скоростью, равной приблизительно \\$3 \times 10^8\\$ м/с в вакууме. Скорость света зависит от оптических свойств среды, в которой он распространяется.\\
~\\
6. \textbf{Дисперсия.} Свет различных цветов имеет различные длины волн и, следовательно, различные скорости распространения в разных средах. Это явление называется дисперсией света.\\
~\\
7. \textbf{Излучение и поглощение.} Свет может излучаться и поглощаться различными веществами. Излучение света происходит, когда электроны в атомах или молекулах переходят на более высокие энергетические уровни и затем возвращаются на более низкие уровни, испуская фотоны света.\\
~\\
Эти основные свойства света играют важную роль в его восприятии человеком и во многих приложениях, таких как оптика, фотоника, лазерная техника и другие.
\subsection{Интерференция}
Одним из ярких проявлений волновой природы света является интерференция. Интерференция света возникает при наложении двух или более волн, которые совпадают в пространстве и времени. В результате интерференции могут наблюдаться яркие и темные полосы, амплитуда света может усиливаться или ослабевать.
\subsection{Дифракция}
Дифракция света - это явление, при котором свет распространяется вокруг препятствия или через щель, изменяя свое направление и форму. Дифракция света объясняется его волновой природой и проявляется в виде распространения света волнами во все стороны от препятствия или щели.
\subsection{Поляризация}
Поляризация света - это явление, при котором световая волна колеблется только в определенной плоскости. Поляризация света объясняется волновой природой света и проявляется в виде фильтрации света, отражения от поверхностей под определенным углом и других явлений.
\subsection{Интерференция и дифракция в оптических приборах}
Интерференция и дифракция света широко используются в оптических приборах, таких как интерферометры, дифракционные решетки, голограммы и другие. Эти явления позволяют улучшить разрешающую способность оптических систем и получить дополнительную информацию о свете и объектах, с которыми он взаимодействует.
\subsection{Квантовая природа света}
Волновая природа света была дополнена квантовой теорией, которая объясняет такие явления, как фотоэффект, комбинационное рассеяние и другие. Квантовая природа света заключается в том, что свет представляет собой поток квантов энергии, называемых фотонами. Фотоны обладают дискретными значениями энергии и имеют волновые свойства.
\subsection{Применение волновой природы света}
Волновая природа света имеет широкий спектр применений в различных областях науки и техники. Она используется в оптике, лазерных технологиях, спектроскопии, фотографии, медицине и других областях. Понимание волновой природы света позволяет разрабатывать новые методы и приборы для исследования и использования света.\\
~\\

\newpage
\section{Частотный диапазон электромагнитных волн}
Электромагнитные волны представляют собой колебания электрического и магнитного поля, которые распространяются в пространстве со скоростью света. Они обладают различными частотами и длинами волн, что определяет их свойства и способность взаимодействовать с окружающей средой.\\
~\\
Частотный диапазон электромагнитных волн охватывает широкий спектр значений, начиная от очень низких частот до крайне высоких. Всего можно выделить несколько основных диапазонов, каждый из которых имеет свои особенности и применения.
\subsection{Радиоволны}
Самыми низкими по частоте являются радиоволны. Они имеют длины волн от нескольких метров до сотен километров и используются для передачи информации на большие расстояния. Радиоволны применяются в радио- и телевещании, связи, радиолокации и других сферах.
\subsection{Микроволны}
Микроволны имеют более высокую частоту и меньшую длину волны, чем радиоволны. Их диапазон составляет от нескольких миллиметров до нескольких сантиметров. Микроволны используются в микроволновых печах, радарах, беспроводных сетях и других технологиях.
\subsection{Инфракрасное излучение}
Инфракрасное излучение имеет еще более высокую частоту и меньшую длину волны, чем микроволны. Его диапазон составляет от нескольких микрометров до нескольких миллиметров. Инфракрасное излучение обладает тепловыми свойствами и широко применяется в термографии, ночном видении, медицине и других областях.
\subsection{Видимый свет}
Видимый свет - это узкий диапазон электромагнитных волн, которые способны восприниматься человеческим глазом. Он имеет длины волн от 400 до 700 нанометров и включает в себя все цвета радуги. Видимый свет играет ключевую роль в нашем восприятии окружающего мира и используется в освещении, фотографии, оптике и других областях.
\subsection{Ультрафиолетовое излучение}
Ультрафиолетовое излучение имеет еще более высокую частоту и меньшую длину волны, чем видимый свет. Его диапазон составляет от 10 до 400 нанометров. Ультрафиолетовое излучение может быть опасным для человека и вызывать солнечные ожоги, поэтому требуется использование солнцезащитных средств. Однако оно также имеет применение в медицине, флуоресцентных лампах и других областях.
\subsection{Рентгеновское и гамма-излучение}
Рентгеновское и гамма-излучение имеют самую высокую частоту и самую маленькую длину волны. Они используются в медицине для рентгеновских исследований и лечения рака, а также в ядерной энергетике и других областях.\\
~\\
Частотный диапазон электромагнитных волн охватывает огромный спектр значений, каждый из которых имеет свои особенности и применения. Понимание этого диапазона позволяет нам лучше понять свет как электромагнитную волну и его взаимодействие с окружающей средой.
\subsection{Определение электромагнитных волн}
Электромагнитные волны представляют собой периодические колебания электрического и магнитного поля, распространяющиеся в пространстве со скоростью света. Они возникают в результате взаимодействия электрических и магнитных полей, которые образуются при движении заряженных частиц.\\
~\\
Основными характеристиками электромагнитных волн являются частота и длина волны. Частота определяет количество колебаний электрического и магнитного поля за единицу времени и измеряется в герцах (Гц). Длина волны представляет собой расстояние между двумя соседними точками, в которых поля достигают максимальной амплитуды, и измеряется в метрах (м).\\
~\\
Электромагнитные волны охватывают широкий частотный диапазон, который включает в себя различные виды излучения, такие как радиоволны, микроволны, инфракрасное излучение, видимый свет, ультрафиолетовое излучение, рентгеновское излучение и гамма-излучение. Каждый вид излучения имеет свою уникальную частоту и длину волны, что определяет его свойства и воздействие на окружающую среду.\\
~\\
Электромагнитные волны играют важную роль в нашей жизни. Например, видимый свет является формой электромагнитного излучения, которое мы воспринимаем глазами. Радиоволны используются для передачи информации по радио и телевидению. Микроволны применяются в микроволновых печах для нагрева пищи. Рентгеновское излучение используется в медицине для диагностики и лечения различных заболеваний.\\
~\\
Важно отметить, что человеческая способность воспринимать свет ограничена определенным диапазоном частот, который называется видимым спектром. Видимый спектр включает в себя цвета от красного до фиолетового и имеет частотный диапазон примерно от \\$4.3 \times 10^{14}\\$ Гц до \\$7.5 \times 10^{14}\\$ Гц. Это означает, что мы можем воспринимать только часть электромагнитных волн, а остальные частоты остаются невидимыми для нашего глаза.\\
~\\
В заключение, электромагнитные волны представляют собой периодические колебания электрического и магнитного поля, которые распространяются в пространстве со скоростью света. Они охватывают широкий частотный диапазон и играют важную роль в нашей жизни. Однако, человеческая способность воспринимать свет ограничена видимым спектром, который включает только определенный диапазон частот.
\subsection{Структура электромагнитного спектра}
Электромагнитный спектр представляет собой непрерывный диапазон электромагнитных волн, отличающихся по частоте и длине волны. Спектр охватывает все возможные значения этих параметров, начиная от очень низких частот и длин волн до очень высоких.\\
~\\
Структура электромагнитного спектра можно разделить на несколько основных областей:
\begin{enumerate}
\item \textbf{Радиоволны} - это область спектра с наибольшей длиной волны и наименьшей частотой. Радиоволны используются для передачи информации на большие расстояния, например, в радио- и телекоммуникационных системах.
\item \textbf{Микроволны} - это область спектра с более высокой частотой и более короткой длиной волны, чем радиоволны. Микроволны используются в радиовещании, радиолокации, микроволновых печах и других технологиях.
\item \textbf{Инфракрасное излучение} - это область спектра с еще более высокой частотой и еще более короткой длиной волны, чем микроволны. Инфракрасное излучение используется в тепловизорах, пультовых устройствах, системах безопасности и других приложениях.
\item \textbf{Видимый свет} - это узкий диапазон частот и длин волн, который человеческий глаз способен воспринимать. Видимый свет включает в себя все цвета радуги, от красного до фиолетового. Он играет ключевую роль в нашем восприятии окружающего мира и используется в освещении, оптике и других областях.
\item \textbf{Ультрафиолетовое излучение} - это область спектра с еще более высокой частотой и еще более короткой длиной волны, чем видимый свет. Ультрафиолетовое излучение имеет как положительные, так и отрицательные эффекты на живые организмы и широко используется в медицине, науке и промышленности.
\item \textbf{Рентгеновское излучение} - это область спектра с очень высокой частотой и очень короткой длиной волны. Рентгеновское излучение используется в медицине для диагностики и лечения, а также в научных исследованиях и промышленности.
\item \textbf{Гамма-излучение} - это область спектра с самой высокой частотой и самой короткой длиной волны. Гамма-излучение является самым энергетически интенсивным и опасным типом излучения. Оно используется в медицине для лечения рака и в ядерной энергетике.
\end{enumerate}
Структура электромагнитного спектра позволяет использовать различные области спектра для различных целей, от коммуникации и освещения до научных исследований и медицинских процедур. Понимание этой структуры является важным для разработки и применения технологий, связанных с электромагнитными волнами.
\subsection{Частотный диапазон световых волн}
Световые волны являются частью электромагнитного спектра и имеют определенный частотный диапазон. Частотный диапазон световых волн включает в себя узкую область электромагнитного спектра, которая воспринимается человеческим глазом как свет.\\
~\\
Световые волны имеют частоты от приблизительно \\$4 \times 10^{14}\\$ герц (Гц), что соответствует фиолетовому цвету, до приблизительно \\$7.5 \times 10^{14}\\$ Гц, что соответствует красному цвету. Этот диапазон частот называется видимым спектром.\\
~\\
Видимый спектр света состоит из различных цветов, которые можно увидеть в радуге или при пропускании света через призму. Цвета видимого спектра в порядке возрастания частоты включают в себя красный, оранжевый, желтый, зеленый, голубой, синий и фиолетовый.\\
~\\
Однако, помимо видимого спектра, существуют и другие частоты световых волн, которые не воспринимаются человеческим глазом. Например, ультрафиолетовые волны имеют частоты выше \\$7.5 \times 10^{14}\\$ Гц и не видны для человеческого глаза, но могут быть замечены некоторыми животными, такими как пчелы. Инфракрасные волны, с частотами ниже \\$4 \times 10^{14}\\$ Гц, также не видимы для человеческого глаза, но могут быть обнаружены с помощью специальных приборов, таких как инфракрасные камеры.\\
~\\
Таким образом, частотный диапазон световых волн охватывает узкую область электромагнитного спектра, которая воспринимается человеческим глазом как свет. Видимый спектр света включает в себя различные цвета, начиная от красного и заканчивая фиолетовым. Однако, помимо видимого спектра, существуют и другие частоты световых волн, которые не видны для человеческого глаза, но могут быть обнаружены с помощью специальных приборов или восприняты некоторыми животными.\\
~\\

\newpage
\section{Шкала электромагнитных волн}
Шкала электромагнитных волн представляет собой упорядоченный набор различных типов волн, которые отличаются по длине волны и частоте. Эта шкала позволяет классифицировать электромагнитные волны и определить их положение в спектре.
\subsection{Спектр электромагнитных волн}
Спектр электромагнитных волн включает в себя все возможные типы волн, от самых коротких до самых длинных. Он охватывает широкий диапазон длин волн и частот, начиная от гамма-лучей с очень короткими длинами волн и высокими частотами, и заканчивая радиоволнами с очень длинными длинами волн и низкими частотами.
\subsection{Классификация электромагнитных волн}
Электромагнитные волны классифицируются на основе их длины волны и частоты. Существует несколько основных типов волн, которые включаются в шкалу электромагнитных волн:
\begin{itemize}
\item Гамма-лучи: это самые короткие волны в спектре электромагнитных волн. Они имеют очень высокую частоту и используются в медицине и научных исследованиях.
\item Рентгеновские лучи: они имеют более длинные длины волн и ниже частоты, чем гамма-лучи. Рентгеновские лучи используются в медицине для обнаружения и изучения внутренних органов и структур.
\item Ультрафиолетовые лучи: они имеют еще более длинные длины волн и ниже частоты, чем рентгеновские лучи. Ультрафиолетовые лучи присутствуют в солнечном излучении и могут вызывать повреждение кожи и глаз.
\item Видимый свет: это самый узкий диапазон электромагнитных волн, который может быть воспринят человеческим глазом. Он включает в себя все цвета радуги, от фиолетового до красного.
\item Инфракрасные лучи: они имеют более длинные длины волн и ниже частоты, чем видимый свет. Инфракрасные лучи используются в тепловизорах и других приборах для обнаружения теплового излучения.
\item Радиоволны: это самые длинные волны в спектре электромагнитных волн. Они имеют очень низкую частоту и используются для передачи радио- и телевизионных сигналов.
\end{itemize}
\subsection{Человеческая способность восприятия света}
Человеческий глаз способен воспринимать только ограниченный диапазон электромагнитных волн, который называется видимым спектром. Видимый спектр включает в себя длины волн от приблизительно 400 до 700 нанометров, что соответствует различным цветам радуги.\\
~\\
Человеческое восприятие цвета основано на способности глаза различать разные длины волн света. Когда свет попадает на сетчатку глаза, специальные клетки, называемые конусами, реагируют на разные длины волн и передают информацию о цвете мозгу.\\
~\\
Однако, помимо видимого спектра, человеческий глаз также может воспринимать некоторые инфракрасные и ультрафиолетовые лучи, но в значительно меньшей степени. Это объясняет, почему мы не можем видеть некоторые объекты, которые излучают только инфракрасное или ультрафиолетовое излучение.
\subsection{Значение шкалы электромагнитных волн}
Шкала электромагнитных волн имеет большое значение в науке и технологии. Она позволяет ученым классифицировать и изучать различные типы электромагнитных волн, а также разрабатывать новые технологии и приборы для их использования.\\
~\\
Например, знание о спектре электромагнитных волн позволяет разрабатывать новые методы лечения рака с использованием гамма-лучей и рентгеновских лучей, а также создавать новые виды световых источников для освещения и дисплеев.\\
~\\
Также шкала электромагнитных волн имеет практическое применение в области коммуникаций. Различные типы электромагнитных волн используются для передачи информации по радио, телевидению, сотовой связи и другим системам связи.\\
~\\
В целом, шкала электромагнитных волн играет важную роль в нашем понимании света и его взаимодействия с окружающим миром. Она помогает нам лучше понять природу электромагнитных волн и использовать их в различных областях науки и технологии.
\subsection{Определение электромагнитных волн}
Электромагнитные волны представляют собой периодические колебания электрического и магнитного поля, распространяющиеся в пространстве со скоростью света. Они возникают в результате взаимодействия электрических и магнитных полей, которые образуются при движении заряженных частиц.\\
~\\
Основными характеристиками электромагнитных волн являются частота и длина волны. Частота определяет количество колебаний электрического и магнитного поля за единицу времени и измеряется в герцах (Гц). Длина волны представляет собой расстояние между двумя соседними точками, в которых поля достигают максимальной амплитуды, и измеряется в метрах (м).\\
~\\
Электромагнитные волны могут иметь различные частоты и длины волн, что определяет их положение в шкале электромагнитных волн. Шкала включает в себя различные виды волн, такие как радиоволны, микроволны, инфракрасное излучение, видимый свет, ультрафиолетовое излучение, рентгеновское излучение и гамма-излучение.\\
~\\
Видимый свет является узким диапазоном электромагнитных волн, который способен восприниматься человеческим глазом. Он имеет частоты от приблизительно \\$4.3 \times 10^{14}\\$ Гц (фиолетовый цвет) до \\$7.5 \times 10^{14}\\$ Гц (красный цвет) и длины волн от приблизительно 400 нм до 700 нм.\\
~\\
Определение электромагнитных волн является важным для понимания природы света и его взаимодействия с окружающей средой. Электромагнитные волны играют ключевую роль во многих областях науки и техники, таких как радио- и телекоммуникации, оптика, медицина и другие.
\subsection{Структура электромагнитных волн}
Электромагнитные волны представляют собой периодические колебания электрического и магнитного поля, распространяющиеся в пространстве со скоростью света. Они обладают определенной структурой, которая определяется их частотой и длиной волны.\\
~\\
Структура электромагнитных волн включает в себя следующие основные элементы:\\
~\\
1. \textbf{Электрическое поле} - это физическое поле, создаваемое заряженными частицами, которое оказывает воздействие на другие заряженные частицы. В электромагнитных волнах электрическое поле изменяется со временем и пространством, создавая периодические колебания.\\
~\\
2. \textbf{Магнитное поле} - это физическое поле, создаваемое движущимися заряженными частицами, которое оказывает воздействие на другие заряженные частицы. В электромагнитных волнах магнитное поле также изменяется со временем и пространством, синхронно с изменениями электрического поля.\\
~\\
3. \textbf{Частота} - это количество колебаний электромагнитной волны, происходящих за единицу времени. Она измеряется в герцах (Гц). Частота определяет энергию и цвет электромагнитной волны.\\
~\\
4. \textbf{Длина волны} - это расстояние между двумя соседними точками на электромагнитной волне, которые находятся в фазе. Она измеряется в метрах (м) или других единицах длины. Длина волны обратно пропорциональна частоте и определяет спектр электромагнитных волн.\\
~\\
5. \textbf{Амплитуда} - это максимальное значение электрического или магнитного поля в точке пространства, находящейся на пути электромагнитной волны. Она определяет интенсивность и яркость электромагнитной волны.\\
~\\
6. \textbf{Фаза} - это смещение электрического и магнитного поля относительно начального положения. Фаза определяет положение точки на волне в определенный момент времени.\\
~\\
Структура электромагнитных волн может быть представлена в виде графика, называемого \textbf{волновым образом}. Волновой образ показывает изменение электрического и магнитного поля в зависимости от времени и пространства.\\
~\\
В заключение, структура электромагнитных волн включает электрическое и магнитное поле, частоту, длину волны, амплитуду и фазу. Понимание структуры электромагнитных волн является важным для изучения света как электромагнитной волны и его восприятия человеком.
\subsection{Свойства электромагнитных волн}
Электромагнитные волны обладают рядом характерных свойств, которые определяют их поведение и взаимодействие с окружающей средой. Некоторые из основных свойств электромагнитных волн включают:
\begin{enumerate}
\item \textbf{Интерференция и дифракция}: Электромагнитные волны могут взаимодействовать друг с другом и с преградами, что приводит к явлениям интерференции и дифракции. Интерференция возникает при наложении двух или более волн, что может приводить как к усилению, так и к ослаблению их амплитуды. Дифракция, в свою очередь, проявляется в изменении направления распространения волны при ее прохождении через отверстия или преграды.
\item \textbf{Отражение и преломление}: При переходе электромагнитной волны из одной среды в другую происходит отражение и преломление. Отражение представляет собой отклонение волны от поверхности раздела сред, при котором волна отражается обратно в исходную среду. Преломление, в свою очередь, происходит при изменении скорости распространения волны при переходе из одной среды в другую, что приводит к изменению ее направления.
\item \textbf{Поляризация}: Электромагнитные волны могут быть поляризованными, то есть колебания электрического и магнитного полей происходят в определенной плоскости. Существуют различные типы поляризации, такие как линейная, круговая и эллиптическая поляризация, которые определяются направлением и фазой колебаний полей.
\item \textbf{Интенсивность}: Интенсивность электромагнитной волны определяет ее энергию, переносимую через единицу площади в единицу времени. Интенсивность волны зависит от амплитуды колебаний полей и частоты волны.
\item \textbf{Скорость распространения}: Электромагнитные волны распространяются со скоростью света в вакууме, которая составляет приблизительно \\$3 \times 10^8\\$ м/с. Скорость распространения волны зависит от свойств среды, через которую она проходит, и может быть меньше скорости света в вакууме.
\item \textbf{Длина волны и частота}: Длина волны электромагнитной волны определяется расстоянием между двумя соседними точками с одинаковой фазой колебаний. Частота волны, в свою очередь, определяет количество колебаний, происходящих за единицу времени. Длина волны и частота связаны между собой соотношением \\$\lambda = \frac{c}{f}\\$, где \\$\lambda\\$ - длина волны, \\$c\\$ - скорость света, \\$f\\$ - частота волны.
\end{enumerate}
Эти свойства электромагнитных волн играют важную роль в их взаимодействии с окружающей средой и восприятии человеком. Понимание этих свойств позволяет более глубоко изучить природу света и его влияние на нашу жизнь.\\
~\\

\newpage

\section{Восприятие света человеком}
\begin{center}
    \textbf{
        Спасибо, что воспользовались Scribot! Надеюсь, Вам понравилась курсовая работа!\\
        Для получения полной версии отправьте 99 рублей по ссылке:\\
        https://pay.cloudtips.ru/p/7a822105\\
        Или по QR-коду:\\
    }
\end{center}
\begin{figure}[h]
    \center{\includegraphics[width=\linewidth/2]{qrCode}}
    \caption{QR-код на оплату работы.}
    \label{ris:image}
\end{figure}
\newpage
\begin{center}
    \textbf{
        Спасибо, что воспользовались Scribot! Надеюсь, Вам понравилась курсовая работа!\\
        Для получения полной версии отправьте 99 рублей по ссылке:\\
        https://pay.cloudtips.ru/p/7a822105\\
        Или по QR-коду:\\
    }
\end{center}
\begin{figure}[h]
    \center{\includegraphics[width=\linewidth/2]{qrCode}}
    \caption{QR-код на оплату работы.}
    \label{ris:image}
\end{figure}
\newpage

\section{Физиология зрения}
\begin{center}
    \textbf{
        Спасибо, что воспользовались Scribot! Надеюсь, Вам понравилась курсовая работа!\\
        Для получения полной версии отправьте 99 рублей по ссылке:\\
        https://pay.cloudtips.ru/p/7a822105\\
        Или по QR-коду:\\
    }
\end{center}
\begin{figure}[h]
    \center{\includegraphics[width=\linewidth/2]{qrCode}}
    \caption{QR-код на оплату работы.}
    \label{ris:image}
\end{figure}
\newpage
\begin{center}
    \textbf{
        Спасибо, что воспользовались Scribot! Надеюсь, Вам понравилась курсовая работа!\\
        Для получения полной версии отправьте 99 рублей по ссылке:\\
        https://pay.cloudtips.ru/p/7a822105\\
        Или по QR-коду:\\
    }
\end{center}
\begin{figure}[h]
    \center{\includegraphics[width=\linewidth/2]{qrCode}}
    \caption{QR-код на оплату работы.}
    \label{ris:image}
\end{figure}
\newpage

\section{Чувствительность глаза к различным длинам волн}
\begin{center}
    \textbf{
        Спасибо, что воспользовались Scribot! Надеюсь, Вам понравилась курсовая работа!\\
        Для получения полной версии отправьте 99 рублей по ссылке:\\
        https://pay.cloudtips.ru/p/7a822105\\
        Или по QR-коду:\\
    }
\end{center}
\begin{figure}[h]
    \center{\includegraphics[width=\linewidth/2]{qrCode}}
    \caption{QR-код на оплату работы.}
    \label{ris:image}
\end{figure}
\newpage
\begin{center}
    \textbf{
        Спасибо, что воспользовались Scribot! Надеюсь, Вам понравилась курсовая работа!\\
        Для получения полной версии отправьте 99 рублей по ссылке:\\
        https://pay.cloudtips.ru/p/7a822105\\
        Или по QR-коду:\\
    }
\end{center}
\begin{figure}[h]
    \center{\includegraphics[width=\linewidth/2]{qrCode}}
    \caption{QR-код на оплату работы.}
    \label{ris:image}
\end{figure}
\newpage

\section{Влияние света на организм человека}
\begin{center}
    \textbf{
        Спасибо, что воспользовались Scribot! Надеюсь, Вам понравилась курсовая работа!\\
        Для получения полной версии отправьте 99 рублей по ссылке:\\
        https://pay.cloudtips.ru/p/7a822105\\
        Или по QR-коду:\\
    }
\end{center}
\begin{figure}[h]
    \center{\includegraphics[width=\linewidth/2]{qrCode}}
    \caption{QR-код на оплату работы.}
    \label{ris:image}
\end{figure}
\newpage
\begin{center}
    \textbf{
        Спасибо, что воспользовались Scribot! Надеюсь, Вам понравилась курсовая работа!\\
        Для получения полной версии отправьте 99 рублей по ссылке:\\
        https://pay.cloudtips.ru/p/7a822105\\
        Или по QR-коду:\\
    }
\end{center}
\begin{figure}[h]
    \center{\includegraphics[width=\linewidth/2]{qrCode}}
    \caption{QR-код на оплату работы.}
    \label{ris:image}
\end{figure}
\newpage

\section{Заключение}
\begin{center}
    \textbf{
        Спасибо, что воспользовались Scribot! Надеюсь, Вам понравилась курсовая работа!\\
        Для получения полной версии отправьте 99 рублей по ссылке:\\
        https://pay.cloudtips.ru/p/7a822105\\
        Или по QR-коду:\\
    }
\end{center}
\begin{figure}[h]
    \center{\includegraphics[width=\linewidth/2]{qrCode}}
    \caption{QR-код на оплату работы.}
    \label{ris:image}
\end{figure}
\newpage
\begin{center}
    \textbf{
        Спасибо, что воспользовались Scribot! Надеюсь, Вам понравилась курсовая работа!\\
        Для получения полной версии отправьте 99 рублей по ссылке:\\
        https://pay.cloudtips.ru/p/7a822105\\
        Или по QR-коду:\\
    }
\end{center}
\begin{figure}[h]
    \center{\includegraphics[width=\linewidth/2]{qrCode}}
    \caption{QR-код на оплату работы.}
    \label{ris:image}
\end{figure}
\newpage

\section{Список использованных источников}
\begin{center}
    \textbf{
        Спасибо, что воспользовались Scribot! Надеюсь, Вам понравилась курсовая работа!\\
        Для получения полной версии отправьте 99 рублей по ссылке:\\
        https://pay.cloudtips.ru/p/7a822105\\
        Или по QR-коду:\\
    }
\end{center}
\begin{figure}[h]
    \center{\includegraphics[width=\linewidth/2]{qrCode}}
    \caption{QR-код на оплату работы.}
    \label{ris:image}
\end{figure}
\newpage
\begin{center}
    \textbf{
        Спасибо, что воспользовались Scribot! Надеюсь, Вам понравилась курсовая работа!\\
        Для получения полной версии отправьте 99 рублей по ссылке:\\
        https://pay.cloudtips.ru/p/7a822105\\
        Или по QR-коду:\\
    }
\end{center}
\begin{figure}[h]
    \center{\includegraphics[width=\linewidth/2]{qrCode}}
    \caption{QR-код на оплату работы.}
    \label{ris:image}
\end{figure}
\end{document}
