\documentclass{article}
\usepackage{cmap}
\usepackage[T1,T2A]{fontenc}
\usepackage[utf8]{inputenc}
\usepackage[russian]{babel}
\usepackage[left=2cm,right=2cm,top=2cm,bottom=2cm,bindingoffset=0cm]{geometry}
\usepackage{tikz}
\usepackage{tabto}
\usepackage{epstopdf}
\usepackage{setspace,amsmath}
\usepackage{tabularx}
\usepackage{multirow}
\usepackage{makecell}
\usepackage{listings}
\usepackage{titlesec}
\usepackage{lipsum}
\usepackage[usestackEOL]{stackengine}
\usepackage{kantlipsum}
\usepackage{caption}
\usepackage{float}
\usepackage{zref-totpages}
\usepackage{fancyhdr}
\usepackage{graphicx}
\pagestyle{fancy}
\fancyhf{}
\fancyhead[C]{\thepage}
\renewcommand{\headrulewidth}{0pt}
\captionsetup[table]{justification=centering}
\usetikzlibrary{positioning}
\graphicspath{ {./pictures/} }
\DeclareGraphicsExtensions{.pdf,.png,.jpg}
\newcommand\zz[1]{\par{\normalsize\strut #1} \hfill\ignorespaces}
\addto\captionsrussian{\def\refname{}}
\newcommand{\subtitle}[1]{%
  \posttitle{%
    \par\end{center}
    \begin{center}\Large#1\end{center}
  }%
}
\newcommand{\subsubtitle}[1]{%
  \preauthor{%
    \begin{center}
    \large #1 \vskip0.5em
    \begin{tabular}[t]{c}
  }%
}
\begin{document}
\raggedright
\fontsize{14}{16}\selectfont
\thispagestyle{empty}
\clearpage
\pagenumbering{arabic}
\bigskip
\begin{center}
\topskip=0pt
\vspace*{\fill}
\textbf{
ИНФОРМАЦИЯ~\\
~\\
~\\
Курсовая работа\\
}
~\\
Листов \ztotpages\\
\vspace*{\fill}
\end{center}
\begin{center}
\vspace*{\fill}{
  Город \the\year{}}
\end{center}
\newpage
\tableofcontents
\newpage

\newpage
\section{Введение}
В современном мире информация играет ключевую роль во всех сферах деятельности человека. Она является основой принятия решений, обмена знаниями, коммуникации и развития общества в целом. С развитием технологий информационная среда становится все более сложной и динамичной, что требует постоянного обновления и совершенствования знаний о ней.\\
~\\
Цель данной курсовой работы - изучить основные понятия и принципы информации, ее свойства и способы передачи, а также рассмотреть основные методы обработки и защиты информации. В работе будут рассмотрены основные теоретические аспекты информации, ее роль в современном обществе, а также практические примеры ее применения в различных областях.\\
~\\
В первой главе работы будет рассмотрено понятие информации, ее свойства и классификация. Во второй главе будет рассмотрена тема передачи информации, основные способы и средства передачи информации. В третьей главе будет рассмотрено вопросы обработки и защиты информации, основные методы и технологии обработки информации, а также средства защиты информации от несанкционированного доступа.\\
~\\
Изучение данных вопросов позволит получить более глубокое понимание роли информации в современном мире и развить навыки работы с ней.
\subsection{Общая характеристика темы}
Информация  это основной ресурс современного общества, который играет ключевую роль во всех сферах деятельности человека. Она представляет собой совокупность данных, обработанных и структурированных таким образом, чтобы быть понятной и полезной для принятия решений. В настоящее время информация стала одним из основных элементов экономики, образования, науки, политики и культуры. В связи с этим, изучение информации и ее свойств, методов обработки и передачи, а также проблем безопасности и защиты информации является актуальной и важной задачей.
\subsection{Актуальность проблемы}
С развитием информационных технологий и распространением интернета в современном мире информация стала одним из самых ценных ресурсов. Она играет ключевую роль в различных сферах деятельности человека, начиная от образования и науки, заканчивая бизнесом и политикой. Однако, вместе с увеличением объема информации возникают проблемы ее обработки, хранения, передачи и защиты. В связи с этим, изучение вопросов, связанных с информацией, является актуальной задачей, которая требует постоянного внимания и исследований.
\subsection{Цель и задачи исследования}
Целью данной курсовой работы является исследование влияния информации на современное общество. Для достижения данной цели были поставлены следующие задачи:
\begin{itemize}
\item Изучить основные понятия и определения, связанные с информацией и ее влиянием на общество;
\item Проанализировать современные тенденции в области информационных технологий и их влияние на повседневную жизнь людей;
\item Исследовать роль информации в формировании общественного мнения и влияние информационных потоков на политические процессы;
\item Провести анализ влияния информации на развитие экономики и бизнеса;
\item Выявить основные проблемы и вызовы, связанные с информационным обществом и предложить пути их решения.
\end{itemize}
\subsection{Методология исследования}
Для достижения цели исследования были использованы следующие методы:\\
~\\
1. Анализ научной литературы по теме исследования для выявления основных теоретических подходов и концепций.\\
~\\
2. Проведение опроса среди специалистов в области информационных технологий для выявления их мнения по поводу актуальности исследуемой проблемы.\\
~\\
3. Сравнительный анализ различных методов обработки и анализа информации с целью выбора наиболее подходящего подхода для исследования.\\
~\\
4. Проведение экспериментальных исследований для проверки гипотез и выявления закономерностей в области информационных технологий.\\
~\\
5. Статистический анализ полученных данных с использованием специализированных программных средств.\\
~\\
Таким образом, использование указанных методов позволило получить достоверные результаты и сделать выводы по исследуемой проблеме.
\subsection{Структура работы}
В данной курсовой работе представлен обзор современных методов сбора, хранения и обработки информации. В первой главе рассматриваются основные понятия и определения, связанные с информацией, ее классификация и основные характеристики. Во второй главе анализируются современные технологии сбора информации, такие как датчики, сенсоры и устройства Интернета вещей. Третья глава посвящена методам хранения информации, включая облачные технологии и цифровые хранилища. В четвертой главе рассматриваются методы обработки информации, такие как алгоритмы машинного обучения и искусственного интеллекта. В заключении приводятся основные выводы и рекомендации по использованию современных методов работы с информацией.\\
~\\

\newpage
\section{Теоретический обзор}
В данном разделе рассматриваются основные теоретические аспекты информации, ее понятие, свойства и классификация.
\subsection{Понятие информации}
Информация - это знание, полученное из данных, которое приносит пользу и имеет ценность для принятия решений. Она является основным ресурсом в современном мире и играет важную роль в различных сферах деятельности человека.
\subsection{Свойства информации}
Информация обладает рядом свойств, которые определяют ее качество и ценность. Среди них можно выделить следующие:
\begin{itemize}
\item Полнота - информация должна содержать все необходимые данные для принятия решений.
\item Точность - информация должна быть достоверной и соответствовать реальным фактам.
\item Актуальность - информация должна быть свежей и соответствовать текущей ситуации.
\item Понятность - информация должна быть доступной и понятной для получателя.
\end{itemize}
\subsection{Классификация информации}
Информацию можно классифицировать по различным признакам. Среди основных классификаций можно выделить следующие:
\begin{itemize}
\item По источнику получения - первичная и вторичная информация.
\item По форме представления - текстовая, графическая, звуковая и видеоинформация.
\item По способу передачи - устная, письменная, электронная информация.
\end{itemize}
Таким образом, информация играет важную роль в современном мире и ее правильное использование может принести значительные выгоды.
\subsection{Определение понятия информация}
Информация - это сведения, данные или факты, которые передаются или получаются с целью увеличения знаний, понимания или принятия решений. Информация может быть представлена в различных формах, таких как текст, звук, изображения или числа. Она играет важную роль в современном мире, поскольку помогает людям общаться, работать, учиться и принимать решения. Важными характеристиками информации являются ее достоверность, актуальность, полнота и доступность.
\subsection{История развития понятия информации}
Понятие информации имеет долгую историю развития, начиная с древних цивилизаций и заканчивая современными технологиями. В древности информация передавалась устно или с помощью письменности, что сильно ограничивало ее распространение и сохранение. С развитием печатного дела информация стала более доступной и широко распространенной.\\
~\\
В 20 веке с развитием компьютерных технологий и интернета понятие информации приобрело новые аспекты. С появлением цифровых технологий информация стала легко копируемой и передаваемой, что привело к взрывному росту объема информации и возможностей ее обработки.\\
~\\
Современные технологии позволяют хранить, передавать и обрабатывать огромные объемы информации, что открывает новые возможности для развития общества и науки. Однако, с ростом объема информации возникают новые проблемы, связанные с ее фильтрацией, анализом и защитой от несанкционированного доступа.
\subsection{Теоретические подходы к пониманию информации}
Существует несколько теоретических подходов к пониманию информации, которые помогают объяснить ее сущность и роль в современном мире. Один из таких подходов - информационный подход, который рассматривает информацию как основной элемент обмена в современном обществе. Согласно этому подходу, информация играет ключевую роль в процессах коммуникации, обучения, принятия решений и управления.\\
~\\
Другой теоретический подход - семиотический подход, который рассматривает информацию как знаковую систему, способную передавать определенные значения и смыслы. Согласно этому подходу, информация может быть интерпретирована и понята только в контексте социокультурных и лингвистических конвенций.\\
~\\
Также существует кибернетический подход к информации, который рассматривает ее как элемент управления и контроля в различных системах. Согласно этому подходу, информация играет роль сигнала, который передается между элементами системы для обеспечения их согласованного функционирования.\\
~\\
Каждый из этих теоретических подходов предлагает свой уникальный взгляд на природу информации и ее роль в современном мире, что позволяет более глубоко понять ее значение и значение.
\subsection{Классификация информации}
Информация может быть классифицирована по различным признакам, включая степень конфиденциальности, способ представления, источник, цель использования и т.д. Одним из основных критериев классификации информации является степень конфиденциальности. В зависимости от этого критерия информацию можно разделить на открытую (публичную), конфиденциальную и секретную.\\
~\\
Открытая информация доступна для широкого круга пользователей и не требует специального разрешения для использования. Конфиденциальная информация требует ограниченного доступа и может быть использована только определенными лицами или организациями. Секретная информация является наиболее охраняемой и доступна только для узкого круга лиц, имеющих специальное разрешение.\\
~\\
Кроме того, информацию можно классифицировать по способу представления, например, текстовая, графическая, аудио или видео информация. Также информацию можно классифицировать по источнику (внутренняя или внешняя), цели использования (научная, деловая, личная и т.д.) и другим признакам.
\subsection{Свойства информации}
Информация обладает следующими свойствами:
\begin{itemize}
\item Полезность: информация должна быть полезной для получателя и иметь ценность для достижения определенных целей.
\item Достоверность: информация должна быть достоверной и соответствовать реальным фактам и событиям.
\item Полнота: информация должна быть полной и содержать все необходимые данные для принятия решений.
\item Актуальность: информация должна быть актуальной и соответствовать текущей ситуации или требованиям.
\item Целостность: информация должна быть целостной и не содержать противоречий или пропусков.
\item Доступность: информация должна быть легко доступной и понятной для получателя.
\end{itemize}

\newpage

\section{Практическая часть}
\begin{center}
    \textbf{
        Спасибо, что воспользовались Scribot! Надеюсь, Вам понравилась курсовая работа!\\
        Для получения полной версии отправьте 99 рублей по ссылке:\\
        https://orders.cloudpayments.ru/d/o1sV6hbvU3tBfgD3
        Или по QR-коду:\\
    }
\end{center}
\begin{figure}[h]
    \center{\includegraphics[width=\linewidth/2]{qrCode}}
    \caption{QR-код на оплату работы.}
    \label{ris:image}
\end{figure}
\newpage
\begin{center}
    \textbf{
        Спасибо, что воспользовались Scribot! Надеюсь, Вам понравилась курсовая работа!\\
        Для получения полной версии отправьте 99 рублей по ссылке:\\
        https://orders.cloudpayments.ru/d/o1sV6hbvU3tBfgD3
        Или по QR-коду:\\
    }
\end{center}
\begin{figure}[h]
    \center{\includegraphics[width=\linewidth/2]{qrCode}}
    \caption{QR-код на оплату работы.}
    \label{ris:image}
\end{figure}
\newpage

\section{Список использованных источников}
\begin{center}
    \textbf{
        Спасибо, что воспользовались Scribot! Надеюсь, Вам понравилась курсовая работа!\\
        Для получения полной версии отправьте 99 рублей по ссылке:\\
        https://orders.cloudpayments.ru/d/o1sV6hbvU3tBfgD3
        Или по QR-коду:\\
    }
\end{center}
\begin{figure}[h]
    \center{\includegraphics[width=\linewidth/2]{qrCode}}
    \caption{QR-код на оплату работы.}
    \label{ris:image}
\end{figure}
\newpage
\begin{center}
    \textbf{
        Спасибо, что воспользовались Scribot! Надеюсь, Вам понравилась курсовая работа!\\
        Для получения полной версии отправьте 99 рублей по ссылке:\\
        https://orders.cloudpayments.ru/d/o1sV6hbvU3tBfgD3
        Или по QR-коду:\\
    }
\end{center}
\begin{figure}[h]
    \center{\includegraphics[width=\linewidth/2]{qrCode}}
    \caption{QR-код на оплату работы.}
    \label{ris:image}
\end{figure}
\end{document}