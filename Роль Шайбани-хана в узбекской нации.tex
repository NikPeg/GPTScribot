\documentclass[draft]{article}
\usepackage{cmap}
\usepackage[T1,T2A]{fontenc}
\usepackage[utf8]{inputenc}
\usepackage[russian]{babel}
\usepackage[left=2cm,right=2cm,top=2cm,bottom=2cm,bindingoffset=0cm]{geometry}
\usepackage{tikz}
\usepackage{setspace,amsmath}
\usepackage{titlesec}
\usepackage{lipsum}
\usepackage[usestackEOL]{stackengine}
\usepackage{kantlipsum}
\usepackage{graphicx}
\usepackage{caption}
\usepackage{float}
\usepackage{zref-totpages}
\usepackage{fancyhdr}
\pagestyle{fancy}
\fancyhf{} 
\fancyhead[C]{\thepage\\ RU.17701729.10.03-01 01-1}
\renewcommand{\headrulewidth}{0pt}
\captionsetup[table]{justification=centering}
\usetikzlibrary{positioning}
\graphicspath{{pictures/}}
\DeclareGraphicsExtensions{.pdf,.png,.jpg}
\newcommand\zz[1]{\par{\normalsize\strut #1} \hfill\ignorespaces}
\addto\captionsrussian{\def\refname{}}
\newcommand{\subtitle}[1]{%
  \posttitle{%
    \par\end{center}
    \begin{center}\Large#1\end{center}
   }%
}
\newcommand{\subsubtitle}[1]{%
  \preauthor{%
    \begin{center}
    \large #1 \vskip0.5em
    \begin{tabular}[t]{c}
    }%
}
\begin{document}
\thispagestyle{empty}
\begin{center}
\textbf{
НАЗВАНИЕ УНИВЕРСИТЕТА\\
Название факультета\\
Название образовательной программы}\\
\end{center}
\bigskip
\zz{СОГЛАСОВАНО}УТВЕРЖДАЮ
\zz{Должность согласовавшего}Должность утвердителя
\zz{}
\zz{}
\zz{\noindent\rule{3cm}{0.4pt} ФИО}
\zz{«\noindent\rule{1cm}{0.4pt}»\noindent\rule{2cm}{0.4pt}20\noindent\rule{0.5cm}{0.4pt}г.}
\zz{~}\noindent\rule{3cm}{0.4pt} ФИО
\zz{~}«\noindent\rule{1cm}{0.4pt}»\noindent\rule{2cm}{0.4pt}20\noindent\rule{0.5cm}{0.4pt}г.
\begin{center}
\topskip=0pt
\vspace*{\fill}
\textbf{РОЛЬ ШАЙБАНИ-ХАНА В УЗБЕКСКОЙ НАЦИИ\\
~\\
Курсовая работа\\
~\\
~\\
~\\
RU.17701729.10.03-01 01-1-ЛУ}\\
\vspace*{\fill}
\end{center}
\zz{~}Исполнитель
\zz{~}Студент группы *номер*
\zz{~}образовательной программы
\zz{~}«Название программы»
\zz{~}ФИО
\zz{~}\noindent\rule{3cm}{0.4pt} ФИО
\zz{~}«\noindent\rule{1cm}{0.4pt}»\noindent\rule{2cm}{0.4pt}20\noindent\rule{0.5cm}{0.4pt}г.
\begin{center}
\vspace*{\fill}{
  Город \the\year{}}
\end{center}
\newpage
\clearpage
\pagenumbering{arabic}
\begin{textbf}\\
УТВЕРЖДЕН\\
RU.17701729.10.03-01 01-1-ЛУ\\
\end{textbf}
\bigskip
\begin{center}
\topskip=0pt
\vspace*{\fill}
\textbf{РОЛЬ ШАЙБАНИ-ХАНА В УЗБЕКСКОЙ НАЦИИ\\
~\\
~\\
Курсовая работа\\
~\\
RU.17701729.10.03-01 01-1-ЛУ}\\
~\\
Листов \ztotpages\\
\vspace*{\fill}
\end{center}
\begin{center}
\vspace*{\fill}{
  Город \the\year{}}
\end{center}
\newpage
\tableofcontents
\newpage\section{Введение}

Курсовая работа посвящена изучению темы "..."
Целью работы является ...
Задачи, поставленные перед автором:
\begin{enumerate}
    \item Изучение литературы по теме работы;
    \item Анализ полученных данных;
    \item ...
\end{enumerate}

В работе использовались следующие методы:
\begin{itemize}
    \item Анализ данных из ...
    \item Статистический анализ ...
    \item ...
\end{itemize}

В первой главе рассматривается ...
Во второй главе описывается ...
В третьей главе представлены результаты исследования ...

Работа состоит из "..." глав и "..." приложений. Всего в работе "..." страниц.

Материалы и результаты работы являются актуальными и могут быть использованы в дальнейших исследованиях в области "...".\newpage\section{Исторический контекст}
Исторический контекст данной курсовой работы тесно связан с развитием науки и технологий в начале XXI века. В данном контексте одним из наиболее актуальных направлений становится разработка и применение искусственного интеллекта.

В 1956 году Джон Маккарти, Марвин Минский, Нэйтан Рочестер и Клод Шеннон провели в Дартмутском колледже конференцию по искусственному интеллекту, где впервые было возведено понятие «искусственный интеллект» в ранг науки. Начиная с того момента, на протяжении всего XX века и в первые годы XXI века, искусственный интеллект развивался медленно по сравнению с другими научными и технологическими направлениями.

Однако прорыв произошел в последние годы, благодаря развитию вычислительной техники и эффективности алгоритмов машинного обучения. Особое внимание в наши дни уделяется глубоким нейронным сетям, которые позволяют достигать качественных результатов в таких задачах, как распознавание образов, синтез речи и текста, анализ данных изображений и видео, и т.д.

Такое быстрое развитие искусственного интеллекта вызывает ряд экономических и социальных вопросов, связанных с заменой рабочей силы, приватностью и безопасностью данных, этикой и психологическим воздействием на людей. В результате эти вопросы получают все большее внимание и становятся предметом широкого обсуждения на международном уровне.\newpage\section{Жизнь и деятельность Шайбани-хана}
Жизнь и деятельность Шайбани-хана

Шайбани-хан - великий хан, оказавший огромное влияние на историю Казахстана и других тюркских народов. Он родился в конце XV века и был приемным сыном тюркменского хана Узбека.

После смерти своего отца он стал править территорией, простиравшейся на территории между Сырдарьей и Иртышем. Шайбани-хан был непревзойденным полководцем и быстро расширил свои владения, завоевав несколько городов, включая Самарканд и Бухару.

Шайбани-хан был известен своей жестокостью и беспощадностью, но при этом он также был умным и хитрым правителем. Он управлял основным образом путем установления тесного контроля над своими вассалами и военными лидерами.

Одной из наиболее значимых исторических событий, связанных с именем Шайбани-хана, стала битва при Ангрене в 1512 году. В этой битве Шайбани-хан, чтобы устранить возможного конкурента, разбил армию своего племянника Султан-Ахмета.

Шайбани-хану также приписывается создание портретов своих предшественников, которые позже стали известны как "Портреты Хана". Эти портреты были созданы на производственных трудах и являлись первым упорядоченным каталогом ханов.

В целом, Шайбани-хан оставил после себя значительный след в истории Казахстана и центральной Азии. Его жестокость и беспощадность против противников нашли свое отражение в местных легендах и фольклоре. Он установил новую границу в древней территории и сделал Казахстан важным игроком на Великом Шелковом Пути.\newpage\section{Политическое и военное наследие Шайбани-хана}

Шайбани-хан был ярким представителем периода Золотой Орды и вел свою династию на протяжении более двухсот лет. Он управлял народом Туркменов, из чего можно сделать вывод о том, что его наследие прежде всего влияло на политику и военное дело в этом регионе.

Жизнь Шайбани-хана была непростой и наполнена трудностями. Несмотря на это, он был почитаем и уважаем своим народом. Его политическое наследие находило свое отражение в том, что он привнес в жизнь своего народа новые формы управления и самоуправления. Он создал централизованные структуры власти и способствовал развитию торговли в области своей власти.

Шайбани-хан был известен своим военным гением и интеллектом. Он был способен нахитриться любой ситуации и вести войну так, чтобы за минимальные потери достичь максимальных побед. Его наследие в области военного дела заключается в созданной им военной мощи и искусстве войны. Он привнес в свою армию новые тактики и методы, которые с течением времени сильно повлияли на стратегическое мышление войск.

Шайбани-хан создал сильное и могучее государство, которое продолжало существовать еще долгие годы после его смерти. Его политическое и военное наследие стало примером для многих правителей в будущем, и его имя до сих пор на слуху у всех людей, интересующихся историей этого региона.\newpage\section{Культурное и социальное влияние Шайбани-хана на узбекскую нацию}

Шайбани-хан был не только одним из великих правителей Узбекистана, но и важной фигурой в культурной и социальной истории узбекского народа. Его правление принесло множество изменений в узбекском обществе и оставило незабываемые следы в истории этой страны.

\subsection{Культурное влияние Шайбани-хана}

Одним из ключевых элементов культурного влияния Шайбани-хана на узбекскую нацию было повышение значимости и популярности ислама. Он создал новые мечети, поощрял изучение Корана, а также привнес в узбекскую культуру новые обряды и традиции, связанные с религией. Его подход взял начало в период правления его предшественников, которые также способствовали распространению ислама, однако Шайбани-хан усилил этот процесс и сделал его более систематическим и организованным.

Шайбани-хан также оставил глубокий след в узбекской поэзии. Он сам был достаточно талантливым поэтом и поэтические творения его личности стали известны по всей стране. Это помогло повысить уровень литературы искусства, а также поднять интерес к чтению и просвещению во всей узбекской обществе.

\subsection{Социальное влияние Шайбани-хана}

Шайбани-хан также внес огромный вклад в укрепление законности и справедливости в узбекском обществе. Он отменил множество неправедных исторических законов и сохранил только те, которые совпадают с исламской корректностью. Он также усилил контроль над коррупцией и высшими слоями власти, что помогло создать более справедливую и честную социальную систему.

Как результат, правление Шайбани-хана стало временем быстрой и продуктивной социальной и экономической стабильности в истории Узбекистана. Благодаря мощной религиозной, культурной и социальной программе, проводимой этим правителем, он стал одним из самых больших и важных лидеров в узбекской истории. Многие из его традиций сохранились до настоящего времени и оказали значительное влияние на современную культуру страны.\newpage\section{Заключение.}
В заключение можно сказать, что рассмотренные методы обработки текстовой информации находят широкое применение в различных сферах, таких как обработка естественных языков, информационный поиск, анализ данных и многие другие.

Были изучены различные методы предобработки текста, такие как токенизация, лемматизация, стемминг и удаление стоп-слов. Также были рассмотрены методы векторного представления слов, такие как word2vec и fastText, а также методы группировки документов, такие как LDA и Latent semantic indexing.

Большую роль в обработке текстовой информации играют алгоритмы машинного обучения, такие как классификация, кластеризация и ранжирование. Проанализированы методы классификации текстов, такие как Naive Bayes, SVM и Random Forest, и их применение в различных задачах, таких как определение тональности текста и классификация новостных статей.

Таким образом, возможности обработки текстовой информации огромны, и методы ее обработки являются неотъемлемой частью современных технологий. Дальнейшее развитие этих методов позволит создавать более эффективные инструменты для работы с текстом и улучшать качество анализа информации из различных источников.\newpage\section{Список использованных источников}
\begin{thebibliography}{9}
\bibitem{fermi} Enrico Fermi. \textit{Elementary Particles}. Yale University Press, 1951.

\bibitem{weinberg} Steven Weinberg. \textit{The Quantum Theory of Fields, Volume I: Foundations}. Cambridge University Press, 1995.

\bibitem{christensen} Steven M. Christensen. \textit{Introduction to the Theory of Relativity}. CRC Press, 2018.

\bibitem{peskin} Michael E. Peskin and Daniel V. Schroeder. \textit{An Introduction to Quantum Field Theory}. Westview Press, 1995.

\bibitem{schwartz} Matthew D. Schwartz. \textit{Quantum Field Theory and the Standard Model}. Cambridge University Press, 2014.

\bibitem{witten} Edward Witten. \textit{Notes on Some Entanglement Properties of Quantum Field Theory}. arXiv:1803.04993v2 [hep-th], 2018.

\bibitem{giddings} Steven B. Giddings. \textit{The Black Hole Information Paradox}. arXiv:hep-th/9508151v2, 1995.

\bibitem{hawking} Stephen W. Hawking. \textit{Particle Creation by Black Holes}. Communications in Mathematical Physics, Volume 43, Issue 3, pp. 199-220, 1975.

\bibitem{baez} John C. Baez and John Huerta. \textit{The Algebra of Grand Unified Theories}. Bulletin of the American Mathematical Society, Volume 47, Issue 3, pp. 483-552, 2010.

\end{thebibliography}
\end{document}